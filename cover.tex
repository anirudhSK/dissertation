% -*-latex-*-
% 
% For questions, comments, concerns or complaints:
% thesis@mit.edu
% 
%
% $Log: cover.tex,v $
% Revision 1.8  2008/05/13 15:02:15  jdreed
% Degree month is June, not May.  Added note about prevdegrees.
% Arthur Smith's title updated
%
% Revision 1.7  2001/02/08 18:53:16  boojum
% changed some \newpages to \cleardoublepages
%
% Revision 1.6  1999/10/21 14:49:31  boojum
% changed comment referring to documentstyle
%
% Revision 1.5  1999/10/21 14:39:04  boojum
% *** empty log message ***
%
% Revision 1.4  1997/04/18  17:54:10  othomas
% added page numbers on abstract and cover, and made 1 abstract
% page the default rather than 2.  (anne hunter tells me this
% is the new institute standard.)
%
% Revision 1.4  1997/04/18  17:54:10  othomas
% added page numbers on abstract and cover, and made 1 abstract
% page the default rather than 2.  (anne hunter tells me this
% is the new institute standard.)
%
% Revision 1.3  93/05/17  17:06:29  starflt
% Added acknowledgements section (suggested by tompalka)
% 
% Revision 1.2  92/04/22  13:13:13  epeisach
% Fixes for 1991 course 6 requirements
% Phrase "and to grant others the right to do so" has been added to 
% permission clause
% Second copy of abstract is not counted as separate pages so numbering works
% out
% 
% Revision 1.1  92/04/22  13:08:20  epeisach

% NOTE:
% These templates make an effort to conform to the MIT Thesis specifications,
% however the specifications can change.  We recommend that you verify the
% layout of your title page with your thesis advisor and/or the MIT 
% Libraries before printing your final copy.
\title{\textsc{Transport Architectures for an Evolving Internet}}

\author{\textsc{Keith Winstein}}
% If you wish to list your previous degrees on the cover page, use the 
% previous degrees command:
%       \prevdegrees{A.A., Harvard University (1985)}
% You can use the \\ command to list multiple previous degrees
       \prevdegrees{
\begin{tabular}{rll}
Electrical Engineer, & \hspace{-9 pt}Massachusetts Institute of Technology (2014) \\
Master of Engineering, & \hspace{-9 pt}Massachusetts Institute of Technology (2005) \\
Bachelor of Science, & \hspace{-9 pt}Massachusetts Institute of Technology (2004) \\
\end{tabular}}

\department{\mbox{Department of Electrical Engineering and Computer Science}}

% If the thesis is for two degrees simultaneously, list them both
% separated by \and like this:
% \degree{Doctor of Philosophy \and Master of Science}
\degree{Doctor of Philosophy in Computer Science}

% As of the 2007-08 academic year, valid degree months are September, 
% February, or June.  The default is June.
\degreemonth{June}
\degreeyear{2014}
\thesisdate{May 21, 2014}

%% By default, the thesis will be copyrighted to MIT.  If you need to copyright
%% the thesis to yourself, just specify the `vi' documentclass option.  If for
%% some reason you want to exactly specify the copyright notice text, you can
%% use the \copyrightnoticetext command.  
%\copyrightnoticetext{\copyright IBM, 1990.  Do not open till Xmas.}

% If there is more than one supervisor, use the \supervisor command
% once for each.
\supervisor{Hari Balakrishnan}{Fujitsu Professor of Computer Science and Engineering}

% This is the department committee chairman, not the thesis committee
% chairman.  You should replace this with your Department's Committee
% Chairman.
\chairman{Leslie A.~Kolodziejski}{Professor of Electrical Engineering\\ Chair, Department Committee on Graduate Students}

% Make the titlepage based on the above information.  If you need
% something special and can't use the standard form, you can specify
% the exact text of the titlepage yourself.  Put it in a titlepage
% environment and leave blank lines where you want vertical space.
% The spaces will be adjusted to fill the entire page.  The dotted
% lines for the signatures are made with the \signature command.
\maketitle

% The abstractpage environment sets up everything on the page except
% the text itself.  The title and other header material are put at the
% top of the page, and the supervisors are listed at the bottom.  A
% new page is begun both before and after.  Of course, an abstract may
% be more than one page itself.  If you need more control over the
% format of the page, you can use the abstract environment, which puts
% the word "Abstract" at the beginning and single spaces its text.

%% You can either \input (*not* \include) your abstract file, or you can put
%% the text of the abstract directly between the \begin{abstractpage} and
%% \end{abstractpage} commands.

% First copy: start a new page, and save the page number.
\cleardoublepage
% Uncomment the next line if you do NOT want a page number on your
% abstract and acknowledgments pages.
% \pagestyle{empty}
\setcounter{savepage}{\thepage}
\begin{abstractpage}
% $Log: abstract.tex,v $
% Revision 1.1  93/05/14  14:56:25  starflt
% Initial revision
% 
% Revision 1.1  90/05/04  10:41:01  lwvanels
% Initial revision
% 
%
%% The text of your abstract and nothing else (other than comments) goes here.
%% It will be single-spaced and the rest of the text that is supposed to go on
%% the abstract page will be generated by the abstractpage environment.  This
%% file should be \input (not \include 'd) from cover.tex.
Historically, the evolution of network routers was driven primarily by
performance. Recently, owing to the need for better control over network
operations and the constant demand for new features, programmability of routers
has become as important as performance.  However, today's fastest routers,
which have 10--100 ports each running at a line rate of 10--100 Gbit/s, use
fixed-function hardware, which cannot be modified after deployment. This
dissertation describes three router hardware primitives and their
corresponding software programming models that allow network operators to
program specific classes of router functionality on such fast routers.

First, we develop a system for programming stateful packet-processing
algorithms such as algorithms for in-network congestion control, buffer
management, and data-plane traffic engineering. The challenge here is the fact
that these algorithms maintain and update state on the router.  We develop a
small but expressive instruction set for state manipulation on fast routers.
 We then expose this to the programmer through a high-level programming model
and compiler.

Second, we develop a system to program packet scheduling: the task of picking
which packet to transmit next on a link. Our main contribution here is the
finding that many packet scheduling algorithms can be programmed using one
simple idea: a priority queue of packets in hardware coupled with a software
program to assign each packet's priority in this queue.

Third, we develop a system for programmable and scalable measurement of network
statistics. Our goal is to allow programmers to flexibly define what they want
to measure for each flow and scale to a large number of flows. We formalize
a class of statistics that permit a scalable
implementation and show that it includes many useful statistics (\eg
moving averages and counters).

These systems show that it is possible to program several packet-processing
functions at speeds approaching today's fastest routers. Based on these systems, we
distill two lessons for designing fast and programmable routers in the future.
First, specialized designs that program only specific classes of router
functionality improve packet processing throughput by 10--100x relative to a general-purpose
solution. Second, joint design of
hardware and software provides us with more leverage relative to designing only
one of them while keeping the other fixed.

% Try and mention compiler and vertical integration here (and if so), then
% also bring them up in the main text.

\end{abstractpage}

% Additional copy: start a new page, and reset the page number.  This way,
% the second copy of the abstract is not counted as separate pages.
% Uncomment the next 6 lines if you need two copies of the abstract
% page.
% \setcounter{page}{\thesavepage}
% \begin{abstractpage}
% % $Log: abstract.tex,v $
% Revision 1.1  93/05/14  14:56:25  starflt
% Initial revision
% 
% Revision 1.1  90/05/04  10:41:01  lwvanels
% Initial revision
% 
%
%% The text of your abstract and nothing else (other than comments) goes here.
%% It will be single-spaced and the rest of the text that is supposed to go on
%% the abstract page will be generated by the abstractpage environment.  This
%% file should be \input (not \include 'd) from cover.tex.
Historically, the evolution of network routers was driven primarily by
performance. Recently, owing to the need for better control over network
operations and the constant demand for new features, programmability of routers
has become as important as performance.  However, today's fastest routers,
which have 10--100 ports each running at a line rate of 10--100 Gbit/s, use
fixed-function hardware, which cannot be modified after deployment. This
dissertation describes three router hardware primitives and their
corresponding software programming models that allow network operators to
program specific classes of router functionality on such fast routers.

First, we develop a system for programming stateful packet-processing
algorithms such as algorithms for in-network congestion control, buffer
management, and data-plane traffic engineering. The challenge here is the fact
that these algorithms maintain and update state on the router.  We develop a
small but expressive instruction set for state manipulation on fast routers.
 We then expose this to the programmer through a high-level programming model
and compiler.

Second, we develop a system to program packet scheduling: the task of picking
which packet to transmit next on a link. Our main contribution here is the
finding that many packet scheduling algorithms can be programmed using one
simple idea: a priority queue of packets in hardware coupled with a software
program to assign each packet's priority in this queue.

Third, we develop a system for programmable and scalable measurement of network
statistics. Our goal is to allow programmers to flexibly define what they want
to measure for each flow and scale to a large number of flows. We formalize
a class of statistics that permit a scalable
implementation and show that it includes many useful statistics (\eg
moving averages and counters).

These systems show that it is possible to program several packet-processing
functions at speeds approaching today's fastest routers. Based on these systems, we
distill two lessons for designing fast and programmable routers in the future.
First, specialized designs that program only specific classes of router
functionality improve packet processing throughput by 10--100x relative to a general-purpose
solution. Second, joint design of
hardware and software provides us with more leverage relative to designing only
one of them while keeping the other fixed.

% Try and mention compiler and vertical integration here (and if so), then
% also bring them up in the main text.

% \end{abstractpage}

