\chapter{Conclusion}
\label{chap:concl}

To conclude this dissertation, we mention broader impact that this dissertation
has had, outline areas for future work, and discuss the broader implications of
this work.

\section{Broader impact}
\label{s:impact}
Several ideas from \pktlanguage have now found their way into P4~\cite{p4}, an
emerging language for programming network devices with considerable industry
momentum behind it~\cite{p4org}. We describe enhancements to P4 informed by
\pktlanguage below.
\begin{CompactEnumerate}
\item When we began work on \pktlanguage in 2015, P4 was still relatively
low-level and close to the targets it was programming. For instance, to match
the underlying router pipeline, P4 expected programmers to specify a
packet-processing program as a sequence of lookups in a set of match-action
tables. Each action within a match-action table was made up of a set of
primitive actions, defined by the P4 language. These primtiive actions within
an action were expected to execute in parallel. This caused considerable
confusion to programmers~\cite{p4-semantics} because the P4 program had serial
semantics between table lookups, while it had parallel semantics within the
action half of a table lookup. Based on \pktlanguage's sequential semantics, P4
adopted a more uniform semantics, where primitive actions within a larger
action also executed sequentially~\cite{p4_sequential_pr, p4_sequential_issue}.
\item \pfs (the version of P4 released in 2016)~\cite{p4_16} adopted many
higher-level language constructs first introduced in \pktlanguage such as
C-style expressions and the C-style ternary operator.
\item \pfs also allows programmers to specify atomic blocks of
code~\cite{p4_atomic_pr, p4_atomic_issue}, which have the same transactional
semantics as packet transactions. This is especially useful when writing P4
code for algorithms similar to the ones considered by \pktlanguage, which
manipulate router state on a per-packet basis.
\end{CompactEnumerate}

The impact of PIFOs and performance queries remains to be seen. We have talked
to hardware engineers in industry about the possibility of adding PIFOs to a
router's scheduler. Many of them seem interested in a programmable scheduler
because they can now expressing scheduling algorithms as firmware, and not
hardware. However, one major concern expressed by them is whether the PIFO
hardware would scale to a multi-Tbit/s router, which typically features
multiple ingress and egress pipelines separated by the packet buffer
(\S\ref{ss:multiple}).

We have also talked to hardware engineers about the possibility of adding a
multiply-accumulate instruction to support linear-in-state measurement queries
efficiently. Based on our conversations, we understand that this is well within
reach, although there might be limitations on the bit-width of the operands and
outputs of the multiply-accumulate instruction.

\section{Future work}
\label{s:future}

Besides addressing the limitations described earlier (\S\ref{chap:limitations}),
there are a number of avenues for future work, described below.

\Para{Instruction set design for routers.} The first generation of programmable
router instruction sets, whether those in commercial chips~\cite{xpliant,
flexpipe, tofino, rmt} or those proposed by \pktlanguage, were designed
manually through a process of trial and error.

In particular, when designing atoms for \pktlanguage, we use trial and error to
first guess atoms that we think might be useful and then use our compiler to
check if those atoms can support useful algorithms.  Formalizing this design
process and automating it into an atom-design tool would be useful when
designing router instruction sets. For instance, given a corpus of data-plane
algorithms, this tool would automatically mine the corpus for recurring motifs
of state and packet header modification. A router hardware engineer could then
design hardware for atoms that capture these motifs.

 Once programmers start writing P4 programs more broadly, we have the
opportunity to design these instruction sets in a more empirical manner,
optimizing instruction sets for actual use cases, as opposed to what a router
engineer thinks the use cases will be. The continued evolution of the x86
instruction set over the last four decades suggests that there is considerable
opportunity for workload-driven instruction set design.

\Para{Understanding the x86 tax\footnote{We thank Adam Belay for coming up with
this expression.} of networking.} This dissertation has focused entirely on
routers, which have historically been architected as fixed-function hardware
devices. It leaves out a large number of networking devices with lower
aggregate speeds that are built on top of commodity x86 processors (\eg network
interface cards, the Linux kernel, middleboxes, proxies, and front-end
servers). Until now, the steady improvement in x86 performance has allowed
these devices to be programmable, while still providing sufficient forwarding
performance for their needs. Going forward, as processor clock frequencies and
core counts stall, it seems natural to turn to specialized
hardware~\cite{dark_silicon, four_horsemen}. Can we profile repeatedly used and
relatively mature networking motifs that run on x86 chips today and measure
their ``x86 tax'' relative to a pure silicon implementation? Such motifs (\eg
SSL encryption and data compression) can be hardened as accelerators in
silicon, providing energy and performance benefits over running on a
general-purpose processor.

\Para{Approximate semantics for packet processing.} \pktlanguage provides
transactional semantics for packet processing. These semantics are
straightforward, but end up rejecting programs for which transactional
semantics are not feasible while running at the router's line rate.  Are weaker
semantics sensible? One possibility is approximating transactional semantics by
only processing a sampled packet stream.  This provides an increased time
budget for each packet in the sampled stream, potentially allowing the packet
to be {\em recirculated} through the pipeline for further packet processing.

\Para{A middle plane for networking.} There will always be algorithms that
cannot run in the data plane because they cannot sustain packet processing at
the router's line rate. Today, such algorithms can only run on the much slower
control plane, leading to a performance cliff once an algorithm crosses a
threshold of complexity. Can we develop a programmable ``middle plane'' that
sits between the control and data planes? This middle plane would provide
greater programmability than the data plane, but at lower performance. It could
be used to run algorithms either on a sample of packets across all ports or at
full line rate on a subset of ports. For instance, an operator might require
programmable measurement on a sample of all packets or line-rate scheduling
support only on the congested ports.

\Para{Average case designs for routers.} Routers have historically been
designed to handle worst-case traffic patterns (\ie the smallest packet size at
100\% utilization on all ports). This worst-case design mindset has several
advantages. For instance, it obviates the need for performance profiling
because routers guarantee line-rate performance regardless of packet size or
utilization. It also guards routers against denial-of-service attacks that flood
the router with small packets.

But it also leads to overengineering. A recent study found that the average
packet size in datacenters is around 850 bytes (relative to a minimum packet
size of 64 bytes)~\cite{theo_dc}, while the utilization can be as low as
30\%~\cite{theo_dc}. Factoring both average packet size and utilization, this
implies that routers are designed to handle as much as 44 $\times$ more traffic
than the average. Clearly, there are substantial gains to be had from adopting
an average case mindset.

\Para{Software engineering for programmable routers.} The development
environment for programmable router chips is still quite rudimentary. This
gives us an opportunity to revisit many traditional software engineering
questions in the context of programmable routers. To name a few, developing
better verifiers, debuggers, test generators, and exception handling
mechanisms. While prior work addresses this in the context of a programmable
control plane~\cite{test_gen, hsa}, more work is needed to extend these ideas
to a programmable data plane.

\Para{Evaluating ideas end-to-end on programmable router chips.} Programmable
router chips are just becoming available~\cite{tofino}. While they do not
directly support the kinds of programmability described in this dissertation,
they provide a platform to program different router algorithms on a physical
router within a real network at aggregate speeds exceeding a Tbit/s.  They also
allow us to evaluate the benefits of our ideas end-to-end. For instance, if we
could program a new active queue management scheme on a programmable router,
what would be the effect on the completion time of a data analytics job?

\Para{Design-space exploration of router architectures.} To evaluate the area
overheads of our hardware designs in this dissertation, we have relied on
anectodal estimates of a baseline router chip's area based on private
conversations with engineers in industry~\cite{gibb_parsing}. Ideally, we would
have an open-source hardware design for a router chip that is competitive with
industry-standard router chips. We could then run synthesis experiments on this
open-source chip to provide a rigorous baseline area estimate for our
evaluations.  When adding new hardware functionality, we could add it to this
open-source chip, and rerun synthesis to estimate the new area of the chip.

%TODO: Replace reference to netfpga with reference to reference router design
While the NetFPGA platform provides a reference router design~\cite{netfpga},
the NetFPGA reference router design is designed for an FPGA platform, not an
ASIC. An open-source chip would also enable design-space exploration of router
architectures. For instance, given an overall silicon budget, what is the
Pareto frontier that trades off exact-match table capacity for ternary-match
table capacity? How does this Pareto frontier move with changes in the
transistor size (\eg moving from 32 nm to 16 nm)?.

\Para{Network architecture in the age of programmable routers.} We have shown
that it is technically feasible to build fast and programmable routers,
allowing us to program routers with functionality that was traditionally on end
hosts (\eg congestion control, measurement, and load balancing). Now that we
have this capability to program the network's internals, and assuming we want
to push our networks to the limits of their performance, what functionality
belongs on the end hosts and what belongs within the network?  Does this answer
change if we only want basic correctness (and not optimal performance) from our
networks?

\section{Towards a world of programmable networks}

The results in this dissertation point towards a world of programmable
networks, where network operators \textit{tell} routers what to do, without
being \textit{told} that this is all that the router can do. In such a setting,
operators could customize networks as they see fit.  They could program
additional features that give them performance benefits or better visibility
into their networks.  More interestingly, they could remove needless features
from their router, allowing them to simplify their routers' feature sets, which
in turn could ease troubleshooting when things go wrong.

Besides the benefits to network operators, a programmable router chip has
benefits for router vendors as well. Router vendors can now add new features in
firmware and sell different versions of the firmware to different market
segments. Further, when bugs arise, it is much easier to fix these bugs in
firmware, as opposed to redoing the hardware design for the router, which could
easily take years. It also allows vendors to respond to new requests from
network operators in a period of days--not years.

Whether the classes of router programmability described here will be sufficient
remains to be seen, but we are encouraged by the fact that the hardware designs
proposed here support a wide range of existing use cases and some new use cases
that we had not anticipated initially. We hope the results of this dissertation
provide guidance to router chip manufacturers when designing hardware for
programmable routers.
