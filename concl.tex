\chapter{Conclusion}
\label{chap:concl}

We conclude by mentioning broader impact that this dissertation has had,
outlining areas for future work, and reflecting on lessons learned.

%TODO: Completeness theorem for programmability
\section{Broader impact}
Several ideas from \pktlanguage have now found their way into P4~\cite{p4}, an
emerging language for programming network devices with considerable industry
momentum behind it~\cite{p4org}. We describe enhancements to P4 informed by
\pktlanguage below.
\begin{CompactEnumerate}
\item When we began work on \pktlanguage in 2015, P4 was still relatively
low-level and close to the targets it was programming. For instance, to mirror
the underling router pipeline, P4 expected programmers to specify a
packet-processing program as a sequence of lookups in a set of match-action
tables. Each action within a match-action table was made up of a set of
primitive actions, defined by the P4 language. These primtiive actions within
an action were expected to execute in parallel. This caused considerable
confusion to programmers because the P4 program had serial semantics between
table lookups, while it had parallel semantics within the action half of a
table lookup. Based on \pktlanguage's sequential semantics, P4 adopted a more
uniform semantics, where primitive actions within a larger action also executed
sequentially~\cite{p4_sequential_pr, p4_sequential_issue}.
\item P4-16 (the version of P4 released in 2016)~\cite{p4_16} adopted many
higher-level language constructs first introduced in \pktlanguage: C-style
expressions and the C-style ternary operator.
\item P4-16 also allows programmers to specify atomic blocks of code~\cite{p4_atomic_pr, p4_atomic_issue}, which
have the same transactional semantics as packet transactions. This is
especially useful when writing P4 code for algorithms similar to the ones
considered by \pktlanguage, which manipulate router state on a per-packet
basis.
\end{CompactEnumerate}

The impact of PIFOs and Performance Queries remains to be seen. We have talked
to hardware engineers in industry about the possibility of adding PIFOs to a
router's scheduler. While they seem interested in a programmable scheduler
because they can now expressing scheduling algorithms as firmware, and not
hardware, one major concern expressed by them is whether the PIFO hardware
would scale to a multi-Tbps router, which typically features multiple ingress
and egress pipelines separated by the packet buffer. Our current PIFO design is
intended for a single-pipeline router, and extending our design to
multi-pipeline routers is an interesting area for future work.

We have also talked to hardware engineers about the possibility of adding a
multiply-accumulate instruction to support linear-in-state measurement queries
efficiently. Based on our conversations, we understand that this is well within
reach, although there might be limitations on the bit-width of the operands and
outputs from the multiply-accumulate instruction.

\section{Future work}
\label{s:future}

Besides addressing the limitations described earlier (\S\ref{chap:limitations},
there are a number of avenues for future work, described below.

\Para{Instruction set design for routers.} The first generation of programmable
router instruction sets, whether those in commercial chips~\cite{xpliant,
flexpipe, tofino, rmt} or those proposed by \pktlanguage, were designed
manually through a process of trial and error. Once we have programmers writing
P4 programs more broadly, we have the ability to design these instruction sets
in a more empirical manner, optimizing them for actual use cases, as opposed to
what the router engineer thinks the use cases will be. The continued evolution
of the x86 instruction set over the last four decades suggests that there is
considerable opportunity for workload-driven instruction set design.

\Para{Understanding the x86 tax\footnote{We thank Adam Belay for coming up with
this expression.} of networking.} This dissertation has focused entirely on
routers, which have historically been architected as fixed-function hardware
devices. It leaves out a large number of networking devices with lower
aggregate speeds that are built on top of commodity x86 processors (\eg Network
Interface Cards, the Linux kernel, middleboxes, proxies, front-end servers). So
far the steady improvement in x86 performance has allowed these devices to be
both programmable, while providing sufficient forwarding performance for their
needs. Going forward, as improvements in processor clock frequencies and core
counts stall, it seems natural to turn to specialized hardware~\cite{dark_silicon, four_horsemen}.  One example of this is the Microsoft Catapult
project~\cite{catapult}, which offloads many networking functions, such as TLS
encryption and decryption to an FPGA that interposes between the NIC and the
network. Can we profile repeatedly used and relatively mature networking motifs
that target x86 chips today and measure their “x86 tax” relative to a pure
silicon implementation? Such motifs (e.g., SSL encryption and data compression)
can be hardened as accelerators in silicon, providing energy and performance
benefits over running on a general-purpose processor.

\Para{Approximate semantics for packet processing.} \pktlanguage provides
transactional semantics for packet processing. These semantics are
straightforward, but end up rejecting programs for which transactional
semantics are not feasible while running at the router's line rate.  Are weaker
semantics sensible? One possibility is approximating transactional semantics by
only processing a sampled packet stream.  This provides an increased time
budget for each packet in the sampled stream, potentially allowing the packet
to be {\em recirculated} through the pipeline multiple times for packet
processing.

\Para{A middle plane for networking.} There will always be algorithms that
cannot run in the data plane because they cannot sustain packet processing at
the router’s line rate. Today, such algorithms can only run on the much slower
control plane, leading to a performance cliff once an algorithm crosses a
threshold of complexity. Can we develop a programmable “middle plane” that sits
between the control and data planes? This middle plane would provide greater
programmability than the data plane but at lower performance. It could be used
to run algorithms either on a sample of packets across all ports or at full
line rate on a few ports. For instance, an operator might require programmable
measurement on a sample of all packets or line-rate scheduling support only on
the congested ports.

\Para{Average case designs for routers.} Routers have historically been
designed to handle worst-case traffic patterns (\ie the smallest packet size at
100\% utilization on all ports). This worst-case design mindset has several
advantages. For instance, it obviates the need for performance profiling
because routers guarantee line-rate performance regardless of packet size or
utilizatio. It also guards routers against denial-of-service attacks that flood
the router with small packets. But it also leads to overengineering. A recent
study found that the average packet size in datacenters is around 850 bytes
(relative to a minimum packet size of 64 bytes)~\cite{theo_dc}, while the
utilization can be as low as 30\%~\cite{theo_dc}. Factoring both average packet
size and utilization, this implies that routers are designed to handle as much
as 44 $\times$ more traffic than the average. Clearly, there are substantial
gains to be had from adopting a slightly-less-than-worst-case mindset.

\Para{Software engineering for programmable routers.}
The development environment for programmable router chipsets is still quite
rudimentary. This gives us an opportunity to revisit many traditional software
engineering questions in the context of programmable routers. To name a few,
developing better verifiers, debuggers, test generators, and exception handling
mechanisms. While prior work addresses this in the context of a programmable
control plane~\cite{test_gen, hsa}, extending these ideas to a programmable
data plane would considerably ease the programming of these routers, the way
NVIDIA's Compute Unified Device Architecture (CUDA) environment did for GPUs.

\Para{Evaluating ideas end-to-end on programmable router chipsets.}
Programmable router chipsets are just becoming available~\cite{tofino}. While
they don't directly support the kinds of programmability described in this
dissertation, they provide a platform to program different router algorithms on
a physical router within a real network at aggregate speeds exceeding a Tbit/s.
They also allow us to evaluate the benefits of our ideas end-to-end. For
instance, if we could program a new active queue management scheme on a
programmable router, what would be the effect on the completion time of a
data analytics job?

\Para{Design-space exploration of router architectures.} To evaluate the area
overheads of our hardware designs in this dissertation, we have relied on
anectodal evidence of a router chip's area based on private conversations with
engineers in industry~\cite{gibb_parsing}. Ideally, we would have an
open-source hardware platform for a router chips that is competitive with
industry-standard router chips, which we could then run synthesis experiments
on to provide a baseline area estimate for our evaluations. While the NetFPGA
platform provides a reference router design~\cite{netfpga}, this is designed
for an FPGA platform, not an ASIC. This would also enable design-space
exploration of router architectures. For instance, given an overall silicon
budget, what is the Pareto frontier that trades off exact-match table capacity
for ternary-match table capacity? How does this Pareto frontier move with
changes in the transistor size (\eg moving from 32 nm to 16 nm)?.

\Para{Network architecture in the age of programmable routers.} We have shown
that it is technically feasible to build fast and programmable routers,
allowing us to program routers with functionality that was traditionally on end
hosts (\eg congestion control, probe-based measurement, load balancing). Now
that we have this capability to program the network's internals, and assuming
we want to push our networks to the limits of their performance, what
functionality within a network belongs on the end hosts and what belongs within
the network?  Does this answer change if we only want basic correctness (and
not optimal performance) from our networks?

\section{Closing thoughts}
The most important lesson learned through the course of this dissertation is
the idea of targeting hardware and software to specific classes of router
functionality. As the three systems presented here demonstrate, narrowing our
focus to specific router functionality allows us to resolve the
performance-programmability tradeoff that has plagued routers so far. None of
the systems presented here are as general as a CPU, but each covers a large
number of example use cases within specific functionality classes (stateful
data-plane algorithms, scheduling, and measurement queries that are
linear-in-state) without giving up performance.

We are entering an era where Moore's law has slowed down and effectively
stopped. In this era, transistors are no longer guaranteed to get smaller or
faster. Year on year, the greatest performance improvements will come from
human creativity in specializing hardware for specific applications and
figuring out how best to utilize a \textit{limited} budget of transistors for a
specific application. This is already visible in many domains (\eg deep
learning~\cite{tpu}, sensor data processing~\cite{m7}, video
decoding~\cite{qualcomm_video}). Yet, specializing hardware is at odds with
future-proofing the hardware: any hardware that is specialized for one
applications risks being useless when the application changes. We believe the
approach advocated by this dissertation of tailoring programming abstractions
to specific functionality classes provides one way around this by extracting
the most performance possible out of limited hardware while providing future
proofness within that class.
