\chapter{Conclusion}
% \section{Broader lessons}
% Broader lesson: target abstractions to specific functionality
% Packet transactions in P4
% There is considerable value to domain-specific hw/sw co-design with the end of Moore's law

%\section{Future work}
% Future work (use this to talk about future work in general, not just future work as pertains to each of the three projects)
% 1. instruction set design,
% 2. x86 tax of networking,
% 3. exception handling,
% 4. software engineering for routers,
% 5. trying out ideas on emerging prog. routers,
% 6. approximate semantics (ref. UW paper),
%%\begin{CompactEnumerate}
%%\item Packet transactions provide the first transactional semantics
%%for line-rate packet processing. These semantics make it easier to
%%reason about correctness and performance, but they exclude algorithms that cannot run at
%%line rate while respecting these semantics. Are weaker semantics sensible? One possibility is approximating
%%transactional semantics by only processing a sampled packet stream.
%%This provides an increased time budget for each packet in the sampled stream, potentially allowing
%%the packet to be {\em recirculated} through the pipeline multiple times
%%for packet processing.
%%
% 7. average-case design for routers.
% 8. A middle plane for networking.
% 9. A community emerging around P4.
% 10. A platform for design-space exploration of routers, maybe as a platform for router hardware research.
% 11. Network architecture in the age of programmable routers.
% 12. SmartNICs 

%\section{Concluding thoughts}
