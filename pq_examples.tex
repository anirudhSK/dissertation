%% constructs to cover:
%% filter, map, zip, groupby
%% with and without emits
%% show composition
%% show headers as well as performance metadata conditions

\begin{figure*}[!t]
\centering
\resizebox{\textwidth}{!}{%
\begin{tabular}{llllllll}
\hline

{\textbf{Example}} &
{\textbf{Query code}} &
{\textbf{Description}} &
{\textbf{Linear}} &
{\textbf{Scales?}} &
{\textbf{Pipe}} &
{\textbf{Pipe}} & 
{\textbf{\# of}} \\

&
&
&
{\textbf{in state?}} &
&
{\textbf{depth}} &
{\textbf{width}} &
{\textbf{atoms}} \\

\hline
\hline

Packet counts &
{\ct def count([cnt], []): cnt = cnt + 1; emit() } &
Count packets per source IP. &
Yes &
Yes &
5 &
2 &
7 \\

&
{\ct result = groupby(\pktlog, [srcip], count);} &
&
&
&
&
&
\\

\hline

EWMA over latencies &
{\ct def ewma([avg], [tin, tout]):} &
Maintain a moving EWMA over &
Yes &
Yes &
6 &
4 &
11 \\

&
{\ct \ \ avg = (1-alpha)*avg + (alpha)*(tout-tin)} &
packet latencies per flow. &
&
&
&
\\

&
{\ct ewma\_q = groupby(\pktlog, [\codeftuple{}], ewma);} &
&
&
&
&
\\

\hline

TCP out-of-sequence &
{\ct def oos([lastseq, cnt], [tcpseq, payload\_len]):} &
Count the number of packets per &
Yes &
Yes &
7 &
4 & 
14 \\

&
{\ct \ \ if lastseq != tcpseq: cnt = cnt + 1} &
connection arriving with a sequence &
&
&
&
\\

&
{\ct \ \ lastseq = tcpseq + payload\_len } &
number that is non-consecutive with &
&
&
&
\\

&
{\ct oos\_q = groupby(\pktlog, [\codeftuple{}], oos);} &
the last packet. &
&
&
&
\\

\hline

TCP non-monotonic &
{\ct def nonmt([maxseq, cnt], [tcpseq]):} &
Count the number of packets per &
No &
No &
5 &
2 &
6 \\

&
{\ct \ \ if maxseq > tcpseq: cnt = cnt + 1} &
connection with sequence numbers &
&
&
&
\\

&
{\ct \ \ else: maxseq = tcpseq} &
lower than the maximum so far. &
&
&
&
\\

&
{\ct nm\_q = groupby(\pktlog, [\codeftuple{}], nonmt);} &
&
&
&
&
\\

\hline

Flowlet size &
{\ct def fl\_detect([last\_time, size], [tin]):} &
Compute a histogram over the &
Yes &
No &
11 &
6 & 
31 \\

histogram &
{\ct \ \ if tin - last\_time > delta:} &
lengths of flowlets. This statistic is &
&
&
&
\\

&
{\ct \ \ \ \ emit(); size = 1 } &
 useful to evaluate network load &
&
&
&
\\

&
{\ct \ \ else: size = size + 1} &
balancing schemes, \eg\cite{conga}. &
&
&
&
\\

&
{\ct \ \ last\_time = tin} &
&
&
&
&
\\

&
{\ct R1 = groupby(\pktlog, [\codeftuple{}], fl\_detect);} &
&
&
&
&
\\

&
{\ct fl\_hist = groupby(R1, [size], count);} &
&
&
&
&
\\

\hline

High E2E latency &
{\ct def sum\_lat([e2e\_lat], [tin, tout]): } &
Capture packets experiencing high &
Yes &
Yes &
5 &
3 &
8 \\

&
{\ct \ \ e2e\_lat = e2e\_lat + tout - tin} &
end-to-end queueing latency, by &
&
&
&
\\

&
{\ct e2e = groupby(\pktlog, [uid], sum\_lat);} &
adding time spent in the queue at &
&
&
&
\\

&
{\ct high\_e2e = filter(e2e, e2e\_lat > 10);} &
each hop. &
&
&
&
\\

\hline

Count concurrently &
{\ct def new\_flow([cnt], []):} &
Count the number of active &
No &
No &
4 &
3 & 
10 \\

active connections &
{\ct \ \ if cnt == 0: emit(); cnt = 1} &
connections in a queue over a &
&
&
&
\\

&
{\ct R1 = map(\pktlog, [tin/128], [epoch]);} &
period of time (``epoch''). &
&
&
&
\\

&
{\ct R2 = groupby(R1, [\codeftuple, epoch], new\_flow);} &
&
&
&
&
\\

&
{\ct num\_conns = groupby(R2, [epoch], count);} &
&
&
&
&
\\

\hline

TCP incast &
{\ct R3 = zip(num\_conns, \pktlog);} &
Detect when many connections use &
No &
No &
7 &
4 &
14 \\

&
{\ct ic\_q = filter(R3, qin > 100 and cnt < 25);} &
a long queue. Uses the query above. &
&
&
&
\\

%&
%{\ct } &
%queue. Builds on the query above. &
%&
%&
%&
%\\

\hline

Lossy connections &
{\ct total = groupby(\pktlog, [\codeftuple{}], count);} &
Compute packet loss rate per &
Yes &
No &
8 &
4 &
19 \\

&
{\ct R1 = filter(\pktlog, tout == infinity);} &
connection, reporting connections &
&
&
&
\\

&
{\ct lost = groupby(R1, [\codeftuple{}], count);} &
with packet drop rate higher than &
&
&
&
\\

&
{\ct Z = zip(total, lost);} &
a threshold {\ct p.} &
&
&
&
\\

&
{\ct lc\_q = filter(Z, lost.cnt > p*total.cnt);}
&
&
&
&
\\

\hline

TCP timeouts &
{\ct def timeout([cnt], [last\_time, tin]):} &
Count the number of timeouts for &
Yes &
Yes &
8 &
3 &
15 \\

&
{\ct \ \ timediff = tin - last\_time } &
each TCP connection, by checking &
&
&
&
\\

&
{\ct \ \ if timediff > 280ms and timediff < 320ms:} &
for packet inter-arrival times &
&
&
&
\\

&
{\ct \ \ \ \ cnt = cnt + 1} &
around 300 ms (retransmission &
&
&
&
\\

&
{\ct \ \ last\_time = tin} &
timer). &
&
&
&
\\

&
{\ct to\_q = groupby(\pktlog, [\codeftuple{}], timeout);} &
&
&
&
&
\\

\hline

%% %% template left behind to create new rows quickly
%% &
%% {\ct } &
%% &
%% \\

\end{tabular}
}
\caption{Examples of performance queries. We
report that a query {\em scales} to a large number of keys either if (1) there
are no stateful updates involved, or (2) all its stateful updates are
linear-in-state {\em and} there are no {\ct emit()}s. We use
Domino~\cite{domino_sigcomm} to report the hardware resources, \ie atom count and pipeline depth and width. Linear-in-state queries use the
multiply-accumulate atom (\Sec{aggregation}); others use a NestedIf
atom~\cite{domino_sigcomm} that supports updates predicated on the state value
itself.}
\vspace{-0.1in}
%
%We label the query
%{\em linear-in-state} if all its stateful updates are linear-in-state. 
\label{fig:example-perf-queries}
\end{figure*}
