%% constructs to cover:
%% filter, map, zip, groupby
%% with and without emits
%% show composition
%% show headers as well as performance metadata conditions

\begin{figure*}[!t]
\centering
\resizebox{\textwidth}{!}{%
\begin{tabular}{lllllll}
\hline

{\textbf{Example}} &
{\textbf{Query code}} &
{\textbf{Description}} &
{\textbf{Linear}} &
{\textbf{Scales?}} &
{\textbf{Pipe}} &
{\textbf{Pipe}} \\

&
&
&
{\textbf{in state?}} &
&
{\textbf{depth}} &
{\textbf{width}} \\

\hline
\hline

Packet counts &
{\ct result = groupby(\pktlog, srcip, count);} &
Count packets per source IP. &
Yes &
Yes &
5 &
2 \\

\hline

EWMA over latencies &
{\ct def ewma(avg, tin, tout):} &
Maintain a moving EWMA over &
Yes &
Yes &
6 &
4 \\

&
{\ct \ \ avg = (1-alpha)*avg + (alpha)*(tout-tin)} &
packet latencies per flow. &
&
&
&
\\

&
{\ct ewma\_q = groupby(\pktlog, \codeftuple, ewma);} &
&
&
&
&
\\

\hline

TCP out-of-sequence &
{\ct def oos(lastseq, count, tcpseq, payload\_len):} &
Count the number of packets per &
Yes &
Yes &
7 &
4 \\

&
{\ct \ \ if lastseq != tcpseq: count = count + 1} &
connection arriving with a sequence &
&
&
&
\\

&
{\ct \ \ lastseq = tcpseq + payload\_len } &
number that is non-consecutive with &
&
&
&
\\

&
{\ct oos\_q = groupby(\pktlog, \codeftuple, oos);} &
the last packet. &
&
&
&
\\

\hline

TCP non-monotonic &
{\ct def nonmt(maxseq, count, tcpseq):} &
Count the number of packets per &
No &
No &
5 &
2 \\

&
{\ct \ \ if maxseq > tcpseq: count = count + 1} &
connection with sequence numbers &
&
&
&
\\

&
{\ct \ \ else: maxseq = tcpseq} &
lower than the maximum so far. &
&
&
&
\\

&
{\ct nm\_q = groupby(\pktlog, \codeftuple, nonmt);} &
&
&
&
&
\\

\hline

Flowlet size &
{\ct def fl\_detect(last\_time, size, tin):} &
Compute a histogram over the &
Yes &
No &
11 &
6 \\

histogram &
{\ct \ \ if tin - last\_time > delta:} &
lengths of flowlets. This statistic is &
&
&
&
\\

&
{\ct \ \ \ \ emit(); size = 1 } &
 useful to evaluate network load &
&
&
&
\\

&
{\ct \ \ else: size = size + 1} &
balancing schemes, \eg\cite{conga}. &
&
&
&
\\

&
{\ct last\_time = tin} &
&
&
&
&
\\

&
{\ct R1 = groupby(\pktlog, \codeftuple, fl\_detect);} &
&
&
&
&
\\

&
{\ct fl\_hist = groupby(R1, size, count);} &
&
&
&
&
\\

\hline

High E2E latency &
{\ct def sum\_lat(count, tin, tout): } &
Capture packets experiencing high &
Yes &
Yes &
5 &
3\\

&
{\ct \ \ count = count + tout - tin} &
end to end latency. &
&
&
&
\\

&
{\ct e2e = groupby(\pktlog, uid, sum\_lat);} &
&
&
&
&
\\

\hline

Count concurrently &
{\ct def new\_flow(count):} &
Count the number of active &
No &
No &
4 &
3 \\

active connections &
{\ct \ \ if count == 0: emit(); count = 1} &
connections in a queue over a &
&
&
&
\\

&
{\ct R1 = map(\pktlog, [tin/128], [epoch]);} &
period of time (``epoch''). &
&
&
&
\\

&
{\ct R2 = groupby(R1, [\codeftuple, epoch], new\_flow);} &
&
&
&
&
\\

&
{\ct num\_conns = groupby(R2, epoch, count);} &
&
&
&
&
\\

\hline

TCP incast &
{\ct R4 = zip(num\_conns, \pktlog);} &
Detect scenarios with many &
No &
No &
7 &
4 \\

&
{\ct ic = filter(R4, qin > 100 and count < 25);} &
connections fanning into a long &
&
&
&
\\

&
{\ct } &
queue. Builds on the query above. &
&
&
&
\\

\hline

Lossy connections &
{\ct total = groupby(\pktlog, \codeftuple, count);} &
Compute packet loss rate per &
Yes &
No &
8 &
4 \\

&
{\ct R1 = filter(\pktlog, tout == infinity);} &
connection, reporting connections &
&
&
&
\\

&
{\ct lost = groupby(R1, \codeftuple, count);} &
with packet drop rate higher than &
&
&
&
\\

&
{\ct R4 = zip(total, lost);} &
a threshold {\ct p.} &
&
&
&
\\

&
{\ct lc = filter(R4, loss.count > p*total.count);}
&
&
&
&
\\

\hline

TCP timeouts &
{\ct def timeout(count, last\_time, tin):} &
Count the number of timeouts for &
Yes &
Yes &
8 &
3 \\

&
{\ct \ \ timediff = tin - last\_time } &
each TCP connection, by checking &
&
&
&
\\

&
{\ct \ \ if timediff > 280ms and timediff < 320ms:} &
for packet inter-arrival times &
&
&
&
\\

&
{\ct \ \ \ \ count = count + 1} &
around 300 ms (retransmission &
&
&
&
\\

&
{\ct \ \ last\_time = tin} &
timer). &
&
&
&
\\

&
{\ct to\_q = groupby(\pktlog, \codeftuple, timeout);} &
&
&
&
&
\\

\hline

%% %% template left behind to create new rows quickly
%% &
%% {\ct } &
%% &
%% \\

\end{tabular}
}
\caption{Some example performance queries in \TheSystem. We label the query {\em
    linear-in-state} if all its stateful updates are linear-in-state.  We report
  that a query {\em scales} to a large number of keys either if (i) there are no
  stateful updates involved, or (ii) all stateful updates are linear-in-state
  {\em and} terminate the flow of tuples. Using the Domino compiler (\S\ref{s:compiler}), we report the hardware resources, \ie pipeline depth and width,
  required to map the queries to a pipeline. Linear-in-state queries use the
  multiply-accumulate atom (\Sec{aggregation}); others use a {\em nested-if}
  atom (Table~\ref{tab:templates}) that supports updates predicated on the state value
  itself.}
\label{fig:example-perf-queries}
\end{figure*}
