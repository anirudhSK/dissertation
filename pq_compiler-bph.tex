\subsection{Handling linear-in-state aggregations}
\label{sec:linear-in-state-compilation}

The \TheSystem compiler recognizes linear-in-state updates, and keeps
the necessary auxiliary state to merge these updates correctly with the backing
store. 

Concretely, the compiler should figure out state updates which have the form $S
\gets A \cdot S + B$, where $A$ and $B$ are functions of bounded packet
history. Suppose $A$ and $B$ are functions of the last $k$ packet fields. Then,
the switch should track (i) the relevant fields of the first and last $k$
packets, and (ii) the value of $\prod_{i} A_i$ across all state updates after
the first $k$ packets, to be able to merge each key later. %% Once we determine
%% states $S$ which are updated in a linear-in-state fashion along with the
%% corresponding coefficients $A$ and $B$, the rest is straightforward.

%% How to get \TheSystem aggregation functions to run on switch hardware? These
%% functions involve updates to state on the switch, where the updates are encoded
%% in the form of assignments (\ie {\ct x = ...})  and branches (\ie {\ct if pred
%%   {...} else {...}}). Prior work as part of the Domino
%% language~\cite{domino_sigcomm} studies how to take such programs and configure a
%% pipeline of stateful computation elements that finish operation within a clock
%% cycle.

%% However, there are two problems. First, the domino instruction set of stateful
%% computations isn't sufficient, since none of the proposed stateful atoms encodes
%% multiplications involving the state---which is necessary for the linear-in-state
%% updates.

%% Second, adding the multiply-accumulate atom (\Sec{hardware}) alone isn't
%% sufficient to encode the stateful computations in the multiply-accumulate
%% atom. Specifically, the coefficients of the linear update can vary depending on
%% the code paths executed in the aggregation function; this requires a more
%% complex atom that involves a {\em predicated} linear-in-state update.

%% In the \TheSystem compiler, we retain the simplicity of the multiply-accumulate
%% atom on hardware, but transform the function to encode linear-in-state updates
%% explicitly. This enables the Domino compiler to directly implement those state
%% updates using the multiply-accumulate atom. For our running example, the update
%% to {\ct count} in function {\ct oos\_count} will be transformed to {\ct count =
%%   A*count + B}, where {\ct A} and {\ct B} are pre-computed over all code
%% paths. The resulting function looks like this:
%% \begin{lstlisting}
%% def oos_count(count, lastseq, tcpseq, payload_len):
%%   A = 1; B = 0;
%%   if lastseq != tcpseq:
%%     A = 1; B = 1;
%%   lastseq = tcpseq + payload_len;
%%   count = A*count + B;
%% \end{lstlisting}
%% Observe that the {\ct A} and {\ct B} values thus pre-computed can be reused to
%% implement the merge operations for the linear-in-state updates
%% (\Sec{hardware}). Here, we need to 

We solve two problems for each aggregation function: We (i) determine state
variables which are functions of bounded packet history---configuring the switch
to track $k$ packet fields; and (ii) detect linear-in-state updates for other
states---configuring the switch to track $\prod_{i} A_i$. The switch tracks
these quantities using additional values for each key.

%% While the canonicalized form of a linear-in-state update may be simple to spot,
%% there are query examples where this form is not apparent. For example, consider
%% the aggregation function for the TCP out-of-sequence example in \Sec{language}:

%% \begin{lstlisting}
%% def outofseq(lastseq, oos_count):
%%   if lastseq != tcpseq:
%%     oos_count = oos_count + 1
%%   lastseq = tcpeq + payload_len
%% \end{lstlisting}

%% The update to {\ct oos\_count} is indeed linear-in-state, where $A$ and $B$ are
%% functions of the last two packets: $A = 1$ always; $B = 1$ if the last packet's
%% sequence number (stored in {\ct lastseq}) is equal to the current packet's
%% sequence number, and $0$ otherwise. Note that this inference relies on
%% determining that {\ct lastseq}---a state variable in {\ct outofseq}---is itself
%% just a function of the last two packets.

%% We first show that it is possible to detect when a state is just a function of
%% bounded packet history, up to a maximum bound. For other stateful updates, we
%% show how the compiler checks the linearity of the state update, and transforms
%% the program to precompute the linear update coefficients $A$ and $B$.

\Para{Determining states of bounded packet history.} First, we observe that any
state {\em all of whose assignments} only depend on values which are themselves
functions of bounded packet history is also a function of bounded packet
history. In our running example (\Fig{running-example-code}), {\ct lastseq} is a
function of bounded packet history because it only depends on packet fields {\ct
  tcpseq} and {\ct payload\_len}.

However, handling branching in the code, \ie {\ct if (cond) \{ ...\}}
statements, requires us to generalize the above observation. We check two
conditions: (i) a state should be bounded in its (implicit or explicit)
assignments in {\em all} branches of the program; (ii) each branching condition
{\ct cond} itself should be a function of bounded packet history.

%% For example, since the only assignment to {\ct lastseq} in the example
%% above involves {\ct tcpseq} and {\ct payload\_len} which are both fields in the
%% current packet, we conclude that {\ct lastseq} depends only on the current
%% packet fields.

%% Generalizing this observation to all aggregation functions requires the handling
%% of branching in the code correctly. We further restrict the conditions for
%% bounded packet history as follows: (i) variables that aren't assigned in a
%% branch {\em implicitly retain} their prior value, hence they are not functions
%% of bounded packet history (\eg see the {\ct timeout} function in
%% \Fig{example-perf-queries}); (ii) the branching predicate itself should be a
%% function of bounded packet history (\eg see the TCP non-monotonic packets
%% example in \Fig{example-perf-queries}).

%% Consider this code snippet that detects whether a flow has
%% ever experienced a TCP timeout:
%% \begin{lstlisting}
%% def detect_timeout(timeout, last_time, tin):
%%   if tin - last_time > 200ms and tin - last_time < 300ms:
%%     timeout = 1
%%   last_time = tin
%% \end{lstlisting}
%% Intuitively, it is clear that the {\ct timeout} state variable must be a
%% function of unbounded packet history, since its value depends on whether any
%% packet in the connection appeared after a timeout. Even though its assignments
%% in the program source only use constants (which are degenerate functions of
%% bounded history), we must also consider the {\em implicit} assignment in the
%% {\ct else} branch of the code which makes {\ct timeout} retain its prior value.

%% Further, the enclosing sequence of branches to reach an assignment are also
%% relevant to determine history. For example, consider a code snippet that detects
%% whether the {\em last} packet of a TCP connection has a value smaller than the
%% maximum sequence number seen so far:
%% \begin{lstlisting}
%% def last_nm(maxseq, nonmonotonic, tcpseq, payload_len):
%%   if tcpseq < maxseq:
%%     nonmonotonic = 1 // flag last packet non-monotonic
%%   else:
%%     nonmonotonic = 0
%%     maxseq = tcpseq + payload_len
%% \end{lstlisting}
%% Here each assignment to the state variable {\ct nonmonotonic} only uses
%% constants, but the enclosing branch is taken based on the value of {\ct maxseq},
%% which is itself not a function of bounded packet history. Hence, {\ct
%%   nonmonotonic} cannot be correctly determined based on a small bounded history
%% of packets.

These observations are codified in \Alg{bounded-packet-history}. The algorithm
assigns each state a {\em history value} corresponding to an upper bound on the
number of past packets that the state value depends on. We track the history
separately for each branching {\em context}, \ie the sequence of branches
enclosing any statement.\footnote{\TheSystem disallows multiple {\ct if...else}
  statements at the same nesting level, so the enclosing branches uniquely
  identify code paths.} The algorithm starts with a default large pessimistic
history value for each state (line \ref{line:curr-hist-init}), and performs a
fixed-point computation (line \ref{line:fixed-point}).
%% iterating over the statements of the aggregation function.
%% For example, the history value of {\ct lastseq} in {\ct oos\_count} is 1,
%% since it only depends on the last one packet. The history value of a state is
%% specific to the sequence of branches taken during code execution: we term the
%% most specific predicate enclosing a statement its {\em context.}\footnote{Our
%% grammar permits only nested branching, disallowing multiple {\ct if...else}
%% statements at the same nesting level. So the enclosing predicates uniquely
%% identify code paths.}
For each assignment to a state in the aggregation function, the algorithm
updates the history of that state in the current branching context (lines
\ref{line:stmt-iteration}-\ref{line:stmt-history-update}). For each branch in
the aggregation function, the algorithm maintains a new branching context---and
a history value for the branching context itself
(lines \ref{line:encounter-branching}-\ref{line:branch-ctx-update-end}).
  
%% The algorithm assigns a default maximum bound for all state histories, and
%% performs a fixed-point computation iterating over the statements of the
%% aggregation function\footnote{The fixed-point iteration converges because the
%%   history values can only decrease, and histories are bounded from below at 0.}
%% (lines \ref{line:curr-hist-init}-\ref{line:fixed-point}). For each statement
%% that updates a state, the algorithm updates the history value for the state in
%% the context enclosing the statement through the {\sc update} function (lines
%% \ref{line:stmt-iteration}-\ref{line:stmt-history-update}). When the algorithm
%% encounters a branching statement (line \ref{line:encounter-branching}), it
%% updates the enclosing predicate context as well as the history value for the
%% context itself (lines
%% \ref{line:branch-ctx-update-start}-\ref{line:branch-ctx-update-end}).

%% The history value of a state that gets assigned to a field from the current
%% packet is 1, since it only depends on the last one packet, \eg the {\ct
%%   last\_time} variable in {\ct detect\_timeout()} above. History values are
%% specific to the sequence of branches encountered in the program until that
%% point. We term the most specific predicate enclosing a statement its {\em
%%   context}. The history value of a state may change through the course of the
%% function, as assignments update the state, and a state may have different
%% history values in different contexts.

%% We initialize the history values to a default maximum bound, {\ct MAX\_BOUND},
%% and a `new' empty history that the algorithm will update (lines
%% \ref{line:curr-hist-init} and \ref{line:prev-hist-init}).

%% The algorithm iterates through the statements in the function until it reaches a
%% fixed-point, \ie the history values from the last two iterations match up for
%% all states and contexts (line \ref{line:fixed-point}). The algorithm is
%% guaranteed to converge to a fixed point since, as we show later
%% (\Alg{update-hist}), the history for any state in a given context can only
%% reduce after each iteration, and there is a lower bound of $0$ on the history
%% values, \ie when the state has no dependence on packets.

%% In each iteration, the algorithm iterates through the statements of the program
%% (line \ref{line:stmt-iteration}). For each statement which assigns to a state,
%% the {\ct update} computes a new history value for the state (line
%% \ref{line:stmt-history-update}). 

\begin{algorithm}
  \caption{Detecting states of bounded packet history.}
  \label{alg:bounded-packet-history}
  \begin{algorithmic}[1]
    \State hist = \{state = \{true: max\_bound\}\} \Comment{Init
      histories.} \label{line:curr-hist-init}
    \State prevHist = $\{\}$ \Comment{Previous iteration
      history.} \label{line:prev-hist-init}
    \Function{detectHistory}{{\cta fun}}
    \While{hist != prevHist} \label{line:fixed-point}
    \State prevHist $\gets$ hist \Comment{Store previous iter history.}
    \State hist $\gets$ \{\}
    \State ctx $\gets$ true \Comment{Set up enclosing context.}
    \State ctxHist $\gets$ 0
    \For{stmt in {\cta fun}}\label{line:stmt-iteration}
    \If{stmt == {\cta state = f(x1,...,xn)}}
    \State {\sc update}(state, ctx, x1, ..., xn)\label{line:stmt-history-update}
    \ElsIf{stmt == {\cta if predicate}}\label{line:encounter-branching}
    \State save context info (restore on branch exit)\label{line:branch-ctx-update-start}
    \State newCtx $\gets$ ctx \textbf{and} predicate
    \State ctxHist $\gets$ {\sc getCtxHist}(ctx, newCtx)
    \State ctx $\gets$ newCtx\label{line:branch-ctx-update-end}
    \EndIf
    \EndFor
    \EndWhile
    \EndFunction
  \end{algorithmic}
\end{algorithm}

Now we show precisely how the history values are computed from each statement of
the aggregation function, in function {\sc update} in
\Alg{update-hist}. Consider an assignment statement {\ct x = f(xi, ...,
  xn)}. If any of the {\ct xi} is a function of the last $k$ packets, then {\ct
  x} is a function of at least the last $k$ packets (if not more). Hence, {\sc
  update} sets the state history to the maximum of the histories of the
variables and the context (lines
\ref{line:used-var-start}-\ref{line:assign-max-hist}). For example, the history
value for {\ct lastseq} after its assignment in {\ct oos\_count} is the maximum
of 1 ({\ct tcpseq} and {\ct payload\_len} are functions of the current packet),
and 0 (for the enclosing outermost context {\ct true}).

To look up the history for a given variable, the function {\sc getCtxHist} in
\Alg{update-hist} implements a simple dictionary lookup for a state in the
history of the current iteration. However, a state history may not always be
found in the current iteration history. For example, at the branching condition
{\ct lastseq != tcpseq} in the second fixed-point iteration, the state {\ct
  lastseq} is actually a function of the last {\em two} packets. So, {\sc
  getCtxHist} increments the history value from the previous iteration (line
\ref{line:older-history-returned}).

The fixed-point iterations converge because state histories in a context are
non-increasing. The algorithm returns a bound $k$ on the number of historical
packets that each state depends on, including possibly max\_bound (line
\ref{line:curr-hist-init}, \Alg{bounded-packet-history}).

%% Now we turn our attention to the {\sc update} and {\sc getCtxHist} functions,
%% which are described in \Alg{update-hist}. The {\sc update} function assigns the
%% maximum of the history values of all used variables ({\ct x1, ..., xn}) and the
%% statement's context (ctx) to the history value of the state in the enclosing
%% context (lines \ref{line:used-var-start}-\ref{line:assign-max-hist}). For each
%% used variable {\ct xi}, the function {\sc getCtxHist} returns a stored history
%% either from the current iteration (line \ref{line:current-history-returned}) if
%% the variable has a history in the current iteration. Otherwise, {\sc getCtxHist}
%% returns a history from the previous iteration (line
%% \ref{line:older-history-returned}), for which the stored history must be
%% incremented by one, as the stored value corresponds to the previous packet.

%% The higher-level algorithm (\Alg{bounded-packet-history}) maintains the
%% enclosing predicate context by updating it every time a branch ({\ct if} or {\ct
%%   else}) is taken or exited from (lines
%% \ref{line:branch-ctx-update-start}-\ref{line:branch-ctx-update-end}). We show
%% just the {\ct if} branch of the algorithm, but the {\ct else} branch is handled
%% similarly.

%% At the fixed point of \Alg{update-hist}, if the history value for a state is
%% smaller than the max\_bound, then the state is a function of bounded packet
%% history. All other states are not functions of bounded packet history.

\begin{algorithm}
  \caption{Updating history values from assignment.}
  \label{alg:update-hist}
  \begin{algorithmic}[1]
    \Function{update}{state, ctx, x1, ..., xn}
    \For{xi $\in$ x1, ..., xn}\label{line:used-var-start}
    \State hi = {\sc getCtxHist}(ctx, xi)
    \EndFor\label{line:used-var-end}
    \State hist[state][ctx] $\gets$ max(ctxHist, h1, ...,
    hn) \label{line:assign-max-hist}
    \EndFunction
    \Function{getCtxHist}{ctx, var}
    \If{var $\in$ hist}
    \State \textbf{return} hist[var][ctx] \label{line:current-history-returned}
    \ElsIf{var $\in$ prevHist}
    \State \textbf{return} prevHist[var][ctx] + 1 \Comment{One pkt
      older.} \label{line:older-history-returned}
    \EndIf
    \EndFunction
  \end{algorithmic}
\end{algorithm}

\Para{Determining linear-in-state coefficients.} For each state $S$ that is not
a function of bounded packet history, we check whether the state updates are
linear-in-state as follows:\footnote{These checks are sound
  but incomplete, as they miss updates that {\em simplify} to linear-in-state,
  \eg $(S^2-1)/(S-1)$.} (i) each update to $S$ is syntactically affine, \ie $S
\gets A \cdot S + B$ with either $A$ or $B$ possibly zero; and (ii) $A$, $B$ and
every branch predicate are functions of bounded packet history.

%% \Para{Encoding linear-in-state updates explicitly.} For each state $S$ which is
%% not a function of bounded packet history, we perform the following checks to
%% determine if it is updated in a linear-in-state fashion:
%% \begin{CompactEnumerate}
%%   \item Every update to $S$ follows a {\em syntactic} affine form: $S \gets B$,
%%     $S \gets A \cdot S$ or $S \gets A \cdot S + B$, where $A$ and $B$ are
%%     expressions of only bounded packet history variables.
%%   \item The enclosing predicate context for an assignment to $S$ involves only
%%     bounded packet history variables.
%%   \item Any read from $S$ only updates $S$.
%% \end{CompactEnumerate}

%% These conditions are conservative, as the syntactic checking condition (1) may
%% fail to consider state updates that {\em simplify} to linear-in-state updates,
%% \eg $S \gets (S^2 - 1) / (S + 1)$. Further, there may be other unbounded states
%% which {\em read} from $S$, which are missed because of condition (3). We do not
%% consider vector linear-in-state updates in our current compiler. All the
%% linear-in-state example programs we wrote are recognized by these conservative
%% checks.

If state $S$ satisfies these conditions, we initialize a new value $S_A$ to 1
and a packet counter $c$ on inserting a new key. When the counter $c$ crosses
the packet history bound $k$, we update $S_A$ each time $S$ is updated, as
follows: $S_A \gets S_A \times A$.\footnote{This stateful update itself can be
  implemented through a multiply-accumulate atom.} When the key is evicted,
$S_A$ contains the value of $\prod_i A_i$. The \TheSystem compiler emits the
code necessary to determine the coefficients from \TheSystem statements that
update $S$.

%% For our running example, the
%% accumulation of $S_A$ is performed through the following code fragment:

%% \begin{lstlisting}
%% def oos_count(count, lastseq, tcpseq, payload_len):
%%   A = 1; B = 0;
%%   if lastseq != tcpseq:
%%     A = 1; B = 1;
%%   lastseq = tcpseq + payload_len;
%%   count = A*count + B;
%% \end{lstlisting}

%% If state $S$ satisfies these conditions, we pull out the linear coefficients of
%% each update to $S$. We initialize $S_A$ to 1 and $S_B$ to 0, and for each update
%% $S \gets A \cdot S + B$, we add two statements:
%% \begin{lstlisting}
%% S_A = S_A * A
%% S_B = (A * S_B) + B
%% \end{lstlisting}
%% At the end of the program, $S$ is updated in one linear assignment, namely {\ct
%%   S = S\_A * S + S\_B}. The overall effect of this transformation is that the
%% Domino compiler can now recognize the linear-in-state updates and use the
%% multiply-accumulate atom to encode them.

