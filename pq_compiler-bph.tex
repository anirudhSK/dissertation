\subsection{Handling linear-in-state aggregations}
\label{sec:linear-in-state-compilation}

We now consider the problems of detecting if an aggregation function is
linear-in-state and setting up auxiliary state for such linear-in-state
aggregation functions (Figure~\ref{fig:compiler-steps}). Recall that an
aggregation function is linear-in-state if the updates to all state variables
within the aggregation function can be written as $S =
\boldsymbol{A}(\mathbf{p}) \cdot S + \boldsymbol{B}(\mathbf{p})$).  A general
solution to this problem is challenging because the aggregation function can
take varied forms. For instance, the assignment $S = \frac{S^2 - 1}{S -1}$ is
linear-in-state, but detecting that it is linear-in-state needs the compiler to
perform algebraic simplifications.

We take a pragmatic approach and sacrifice completeness, but still cover useful
functions. Specifically, we only detect linear-in-state state updates through
simple syntactic pattern matching in the compiler (\ie without any algebraic
transformations).  Despite these simplifications, the \TheSystem compiler
correctly identifies all the linear-in-state aggregations in
Table~\ref{tab:example-perf-queries} and targets the multiply-accumulate atom
that we added to the Banzai pipeline.

To describe how linear-in-state detection works, we introduce some terminology.
Recall that an aggregation function takes two arguments (\S\ref{sec:language}):
a list of state variables (\eg a counter) and a list of tuple fields (\eg the
TCP sequence number). We use the term variable in this subsection to refer to
either a state variable or a tuple field. These are the only variables that can
appear within the body of the aggregation function.\footnote{\TheSystem
supports local variables within the function body, but the more general
algorithm is not materially different from the simpler version we present
here.}

\begin{figure}
\centering
% Figure from
% https://docs.google.com/drawings/d/1Y-AC8YxCGkOJsaPILN_Fr1jKed1BKs0UdEab86r9T3g/edit?usp=sharing
\includegraphics[width=\columnwidth]{pq_compiler-steps.pdf}
\caption{Steps for compiling linear-in-state updates.}
\label{fig:compiler-steps}
\end{figure}

We carry out a three-step procedure, summarized in
Figure~\ref{fig:compiler-steps}.  First, for each variable in an aggregation
function, we assign a {\em history}.  This history tells us how many previous
packets we need to look at to determine a variable's value accurately (history
= 1 means the current packet). For instance, for the value of a byte counter,
we need to look back to the beginning of the packet stream (history =
$\infty$), while for a variable that tracks the last TCP sequence number we
need to only look back to the previous packet (history = 2). Consistent with
the definition of history, constants are assigned a history of 0, and variables
in the tuple field list are assigned a history of 1. For state variables, we
use \Alg{history} to determine each variable's history.

Second, once each variable has a history, we look at the history of each state
variable $s$. If the history of $s$ is a finite number $k$, then $s$ only
depends on the last $k$ packets. In this case, the state update for $s$ is
trivially linear-in-state, by setting $A$ to 0 and $B$ to the aggregation
function itself.\footnote{More precisely, the parts of the aggregation function
that update $s$.} If $s$ has an infinite history, we use syntactic pattern
matching to check if the update to $s$ is linear-in-state.

Third, if all state variables have linear-in-state state updates, the
aggregation function is linear-in-state. For such linear-in-state aggregations,
we generate the auxiliary state that permits merging of the aggregation
function (\S\ref{sec:aggregation}). If not, we use the set of stateful atoms
developed in Domino (Table~\ref{tab:templates}) to implement the aggregation
function. We now describe each of the three steps in detail.

\Para{Determining history of state variables.} We now describe \Alg{history}
that assigns a history to every state variable. To motivate this algorithm,
observe that if all assignments to a state variable only use variables that
have a finite history, then the state variable itself has a finite history. For
instance, in Figure~\ref{fig:running-example-code}, right after it is assigned,
{\ct lastseq} has a history of 1 because it only depends on the current
packet's fields {\ct tcpseq} and {\ct payload\_len}. To handle branching in the
code, \ie {\ct if (predicate) \{...\}} statements, we generalize this
observation. A state variable has finite history if  (1) it has finite history
in all its assignments in all branches of the program, and (2) each branching
condition {\ct predicate} itself only depends on variables with a finite
history.

Concretely, {\sc computeHistory} (line \ref{line:computeHistoryStart}) assigns
each variable a history corresponding to an upper bound on the number of past
packets that the state variable depends on. We track the history separately for
each branching context, \ie the sequence of branches enclosing any
statement.\footnote{Currently, \TheSystem forbids multiple {\ctfoot if ...
else} statements at the same nesting level; hence, the enclosing branches
uniquely identify a code path through the function. This restriction is not
fundamental; the more general form can be transformed into this form.} The
algorithm starts with a default large pessimistic history (\ie an approximation
to $\infty$) for each state variable (line \ref{line:curr-hist-init}), and
performs a fixed-point computation (lines
\ref{line:fixed-point}--\ref{line:fixed-point-end}), repeatedly iterating over
the statements in the aggregation function (line
\ref{line:fp-agg-begin}--\ref{line:fp-agg-end}).

For each assignment to a state variable in the aggregation function, the
algorithm updates the history of that state variable in the current branching
context (lines \ref{line:stmt-iteration}--\ref{line:stmt-history-update}). For
each branch in the aggregation function, the algorithm maintains a new
branching context and a history for the branching context itself (lines
\ref{line:encounter-branching}--\ref{line:branch-ctx-update-end}).  At the end
of each iteration, the algorithm increments each variable's history to denote
that the variable is one packet older (line \ref{line:inc}).  The algorithm
returns a conservative history $k$ for each state variable, including possibly
max\_bound (line \ref{line:curr-hist-init}) to reflect an infinite history.

\begin{small}
\begin{algorithm}
  \caption{Determining history of all state variables}
  \label{alg:history}
  \begin{algorithmic}[1]
    \State hist = \{state = \{true: max\_bound\}\} \Comment{Init.
      hist. for all state vars.} \label{line:curr-hist-init}
    \Function{computeHistory}{{\cta fun}}\label{line:computeHistoryStart}
    \While{hist is still changing} \Comment{Run to fixed point.}\label{line:fixed-point}
    \State hist $\gets$ \{\}
    \State ctx $\gets$ true \Comment{Set up outermost context.}
    \State ctxHist $\gets$ 0 \Comment{History value of ctx.}
    \For{stmt in {\cta fun}} \label{line:stmt-iteration} \label{line:fp-agg-begin}
    \If{stmt == {\cta state = expr}}
    \State hist[state][ctx] $\gets$ {\sc getHist}(ctx, expr,
    ctxHist) \label{line:stmt-history-update}
    \ElsIf{stmt == {\cta if predicate}}\label{line:encounter-branching}
    \State save context info (restore on branch exit)\label{line:branch-ctx-update-start}
    \State newCtx $\gets$ ctx \textbf{and} predicate
    \State ctxHist $\gets$ {\sc getHist}(ctx, newCtx, ctxHist)
    \State ctx $\gets$ newCtx\label{line:branch-ctx-update-end}
    \EndIf
    \EndFor \label{line:fp-agg-end}
    \For{ctx, var in hist}\Comment{Make history one pkt older.}
    \State hist[var][ctx] $\gets$ min(hist[var][ctx] + 1, max\_bound)\label{line:inc}
    \EndFor
    \EndWhile \label{line:fixed-point-end}
    \EndFunction
    \Function{getHist}{ctx, {\cta ast}, ctxHist}
    \For{{\cta xi} $\in$ {\sc leafNodes}({\cta ast})}\label{line:used-var-start}
    \State hi = hist[{\cta xi}][ctx]
    \EndFor\label{line:used-var-end}
    \State \textbf{return} max(h1, ... , hn, ctxHist) \label{line:assign-max-hist}
    \EndFunction
  \end{algorithmic}
\end{algorithm}
\end{small}

%TODO: Removing the stuff below.
%%Now we show precisely how the histories are updated as each statement of the
%%aggregation function is processed using the helper function {\sc getHist}.
%%Consider a statement assigning a variable to an expression, {\ct x = expr},
%%within a branching context {\ct ctx}. Then the history of {\ct x} is the
%%maximum of the history of the predicates in {\ct ctx} and the history of the
%%expression {\ct expr}. This is because if either is a function of the last $k$
%%packets, then {\ct x} is a function of at least the last $k$ packets.  To
%%determine the history of {\ct expr}, suppose the AST of {\ct expr} contains the
%%variables {\ct x1, x2, ..., xn} as its leaves. Then, the history of {\ct expr}
%%is the maximum of the histories of the {\ct xi}.  For example, the history for
%%{\ct lastseq} after its assignment in {\ct oos\_count} is the maximum of 1
%%({\ct tcpseq} and {\ct payload\_len} are functions of the current packet), and
%%0 (for the enclosing outermost context {\ct true}).

%The fixed-point iterations converge because the assigned state histories in a
%context are non-increasing.
%

\Para{Determining if a state variable's update is linear-in-state.} For each
state variable $S$ with an infinite history, we check whether the state updates
are linear-in-state as follows: (1) each update to $S$ is syntactically affine,
\ie $S \gets A \cdot S + B$ with either $A$ or $B$ possibly zero; and (2) $A$,
$B$ and every branch predicate depend on variables with a finite history. This
approach is sound, but incomplete: it misses linear-in-state updates such as $S
= \frac{S^2 - 1}{S - 1}$.

\Para{Determining auxiliary state.} For each state variable with a
linear-in-state update, we initialize four pieces of auxiliary state for a
newly inserted key:\footnote{This can happen either when a key first appears or
reappears following an eviction.} (1) a running product $S_A = 1$; (2) a packet
counter $c = 0$; (3) an entry log, consisting of relevant fields from the first
$k$ packets following insertion; and (4) an exit log, consisting of relevant
fields from the last $k$ packets seen so far.  After the counter $c$ crosses
the packet history bound $k$, we update $S_A$ to $A \cdot S_A$ each time $S$ is
updated. This update to $S_A$ can also be implemented using the same
multiply-accumulate atom. When the key is evicted, we send $S_A$ along with the
entry and exit logs to the backing store for merging (see
Appendix~\ref{app:merge}).
