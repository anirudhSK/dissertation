\subsection{Emitting code in P4 and Domino}
\label{sec:emitting-p4-domino-code}

The last pass of the compiler emits P4 and Domino code through a common
intermediate representation consisting of a list of imperative statements, where
each statement takes one of the following forms: (i) ternary assignment ({\ct
  result = predicate ?  expression : expression}), (ii) predicate assignment
({\ct result = predicate}); or (iii) expression assignment ({\ct result =
  expression}). Here the {\ct result} could be a state or a packet variable, and
the expressions and predicates are written over states and packets. It is
conceptually straightforward to print code from this intermediate representation
into P4 actions and Domino programs.

Each operator emits some operator-specific code, along with a {\em valid bit}
that denotes whether the result it produces is considered valid for the next
operator to consume. For each operator in the pipeline from
\Sec{pipeline-layout}, we emit code in the intermediate representation above, as
follows.

For {\ct filter} operators, the compiler emits a predicate assignment that
assigns the valid bit to the conjunction of the filter predicate and the valid
bit of its operand. The {\ct map} operator results in one expression assignment
corresponding to each expression computed by the {\ct map}, and a valid bit
assignment to the operand's valid bit. The {\ct zip} operator results in
emitting a predicate assignment that sets the conjunction of the validity bits
of its operands to its valid bit. The {\ct groupby} operator proceeds only if
its operand is valid, but in addition {\em flattens} the aggregation function
into straight-line code consisting of ternary assignments through a standard
if-conversion procedure~\cite{if-conversion}.

\ngs{figure containing some p4 output for the running example; domino pipeline.}

%% Within each stage of the pipeline, code for the query operator is emitted in a
%% form that is amenable to compilation by a backend compiler to target the p4
%% behavioral model and the Banzai machine. Abstractly, this emitted code has an
%% syntactic form corresponding to simple assignment statements in P4\_16 and
%% domino, along with use of the ternary operator ({\ct ?}) to mimic the effect of
%% branches. The call to {\sc codeGen} in line \ref{line:config-gen-Q} in
%% \Alg{pipeline-construction} emits this code for each query operator.

%% Each stage also emits code to the P4 behavioral model or Banzai through the {\sc
%%   codeGen} function (line \ref{line:config-gen-Q}), which generates an {\em
%%   imperative program} as an action for each stage. The imperative program
%% evaluates a predicate (for {\ct filter}s and {\ct zip}s) or computes a new
%% expression (for {\ct map}s). These {\em stateless} programs are easily
%% implemented through switch match-action configurations. We defer a detailed
%% discussion of the code generation to \Sec{emitting-p4-domino-code}.

%% In \Alg{stage-config-gen}, we show how {\sc codeGen} outputs code for each
%% query operator. It is necessary to detect when the output of any stage is {\em
%%   valid}, in addition to performing the actions associated with the
%% operator. The presence of a {\em valid bit} on a packet for each query operand
%% ensures that the stateful aggregations do not update the state for packets that
%% are previously filtered out.

%% The {\ct filter} operator emits code that sets a valid bit on the packet only if
%% (1) its operand itself provides a valid result, and (2) the packet matches the
%% {\ct filter} predicate (line \ref{line:filter-both-valid}). Setting the valid
%% bit is its only action. Similarly, {\ct zip} sets the valid bit based on the
%% validity of its inputs alone (line \ref{line:zip-both-valid}).

%% The {\ct map} and {\ct groupby} operators perform more computation in addition
%% to setting valid bits, but only if the operand is valid (lines
%% \ref{line:map-compute-only-if-valid} and
%% \ref{line:groupby-compute-only-if-valid}). While the {\ct map} operation always
%% produces a valid result when the operand is valid, the {\ct groupby} is also
%% restricted by the presence of {\ct emit()} in the encountered code paths during
%% execution (line \ref{line:groupby-emit}). The function {\sc FLATTEN} processes
%% the code supplied in the query aggregation function {\ct fun} to a form that is
%% permissible for the P4 backend compiler and domino. We elide the details.

%% \begin{algorithm}
%%   \caption{Emit query code for single operator.}
%%   \label{alg:stage-config-gen}
%%   \begin{algorithmic}[1]
%%     \Function{codeGen}{{\cta Q}}
%%     \State pipe $\gets$ [] \Comment{New pipe stage with no instructions.}
%%     \If{{\cta Q} == {\cta filter(R, pred)}}
%%       \State pipe += ({\cta Q\_valid} $\gets$ {\cta pred} \textbf{and} {\cta R\_valid})
%%       \label{line:filter-both-valid}
%%     \ElsIf{{\cta Q} == {\cta map(R, expr, field)}}
%%       \State pipe += ({\cta Q\_valid} $\gets$ {\cta R\_valid})
%%       \State pipe += (\textbf{if} {\cta R\_valid} \textbf{then} {\cta field} $\gets$ {\cta expr})
%%       \label{line:map-compute-only-if-valid}
%%     \ElsIf{{\cta Q} == {\cta groupby(R, aggFields, fun)}}
%%       \State pipe += (\textbf{if} {\cta R\_valid} \textbf{then} {\sc flatten}({\cta fun}))
%%       \label{line:groupby-compute-only-if-valid}
%%       \State pipe += ({\cta Q\_valid} $\gets$ {\cta R\_valid} \textbf{and} {\ct emit()})
%%       \label{line:groupby-emit}
%%     \ElsIf{{\cta Q} == {\cta zip(R, S)}}
%%       \State pipe += ({\cta Q\_valid} $\gets$ {\cta R\_valid} \textbf{and} {\cta S\_valid})
%%       \label{line:zip-both-valid}
%%     \EndIf
%%     \State \textbf{return} pipe
%%     \EndFunction
%%   \end{algorithmic}
%% \end{algorithm}
