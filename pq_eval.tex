\section{Evaluation}
\label{sec:eval}

%TODO: Add a scalable column to table
\input{examples}
%The performance of Marple in practice depends on three factors: hardware feasibility, result accuracy, and ease of use.
%Section~\ref{s:eval:hardware} discusses the required hardware atoms required to perform aggregations at line rate. We show that Marple's silicon requirements are modest and realizable with current technology.
%Section~\ref{s:eval:traces} presents a Marple aggregation run on traces of both core Internet and datacenter traffic, to demonstrate how the size of the in-memory key-value store affects the number of flows Marple can accurately track.
%Finally, Section~\ref{s:eval:mininet} presents a case study using Marple to
%diagnose the root cause of irregular HTTP traffic.

We evaluate \TheSystem along three dimensions. In \Sec{eval:hardware}, we show
the switch compute resources used for some candidate \TheSystem queries, while
in \Sec{eval:traces}, we measure the memory-bandwidth tradeoff for the
key-value store. In \Sec{eval:mininet}, we show an end to end use case of
\TheSystem by debugging a performance problem, {\em microbursts,} on the P4
behavioral model.

\subsection{Hardware compute resources}
\label{s:eval:hardware}
\label{sec:eval:hardware}

Table~\ref{tab:example-perf-queries} shows several \TheSystem queries.  Next to
each query, we show (1) whether all its aggregations are linear-in-state, (2)
whether it can be scaled by merging correctly with a backing store, and (3) the
router resources required, measured through the pipeline depth (number of
stages), pipeline width (maximum number of parallel computations per stage),
and the number of Banzai atoms (total number of computations).

Table~\ref{tab:example-perf-queries} shows that many useful queries contain
only linear-in-state aggregations, and many of them can be implemented scalably
(\Sec{linear-in-state-description}). Notably, the flowlet size histogram and
lossy connection queries are not scalable despite being linear-in-state, since
they contain {\ct emit()} statements.  In \Sec{workaround-nonscalable}, we
showed how to rewrite some of these queries (\eg lossy connections) to scale,
at the cost of losing some accuracy.

We compute the pipeline's depth and width by compiling each query to Banzai
using the Domino compiler. When compiling each query, Banzai is supplied with a
single stateless atom type, which perform binary operations (arithmetic, logic,
and relational) on pairs of packet fields, and a stateful atom type depending
on the type of the query.  For the linear-in-state queries, we use the
multiply-accumulate atom as the stateful atom, while for the other operations,
we use the NestedIf atom (Table~\ref{tab:templates}). As expected, all the
linear-in-state queries compile to a pipeline with the multiply-accumulate
atom; for all the queries that are not linear-in-state, the NestedIf atom turns
out to be sufficiently expressive.

The computational resources required for \TheSystem queries are modest.  All
queries in Table~\ref{tab:example-perf-queries} require a pipeline shorter than
11 stages.  This is feasible, \eg the RMT architecture offers 32
stages~\cite{rmt}. Further, functionality other than measurement can run in
parallel in each stage because the number of atoms required per stage is at
most 6, while programmable routers provide a few 100 parallel instructions per
stage (\eg RMT provides 224~\cite{rmt}).

\subsection{Memory and bandwidth overheads}
\label{s:eval:traces}
\label{sec:eval:traces}

In this section, we answer the following questions:
\begin{CompactEnumerate}
\item What is a good size for the on-chip key value store?
\item What are the eviction rates to the backing store?
\item How accurate are queries that are not mergeable?
\end{CompactEnumerate}

\Para{Experimental setup.}  We simulate a \TheSystem query over three unsampled
packet traces: two traces from \tenglink core Internet routers, one from
Chicago (\textasciitilde{}150M packets) from 2016~\cite{caida2016} and one from
San Jose (\textasciitilde{}189M packets) from 2014~\cite{caida2014}; and a 2.5
hour university data-center trace (\textasciitilde{}100M packets) from
2010~\cite{bensonDC}. We refer to these traces as Core16, Core14, and DC
respectively.

We evaluate the impact of memory size on cache evictions for a \TheSystem query
that aggregates by \txtftuple.  As discussed in
\S\ref{sec:hardware-feasibility}, our hardware design uses an 8-way LRU cache.
We also evaluate two other geometries for the cache: a hash table, which evicts
the incumbent key upon a collision, and a fully associative LRU. Comparing our
8-way LRU with other hardware designs demonstrates the tradeoff between
hardware complexity and eviction rate.

\Para{Eviction ratios.}
Each evicted key-value pair is streamed to a backing store.  This requires the
backing store to process packets as quickly as they are evicted, which depends
on the incoming packet rate and the eviction ratio, \ie the ratio of evicted
packets to incoming packets.  The eviction ratio depends on the geometry of the
on-chip cache, the packet trace, and the cache size (\ie the number of
key-value pairs it stores). Hence, we measure eviction ratios over (1) the
three geometries for the Core16 trace (Figure~\ref{fig:eviction-geo}), (2) the
three traces using the 8-way LRU geometry (Figure~\ref{fig:eviction-traces}),
and (3) for caches sizes between $2^{16}$ (65K) and $2^{21}$ (2M) key-value
pairs.

% resume here

\begin{figure}[!t]
\centering
\begin{subfigure}[t]{0.48\columnwidth}
\raggedright
\includegraphics[width=\linewidth]{pq_eviction-rate-alltraces.pdf}
\vspace{-0.2in}
\caption{By trace}
\label{fig:eviction-traces}
\end{subfigure}
\begin{subfigure}[t]{0.48\columnwidth}
\raggedleft
\includegraphics[width=\linewidth]{pq_eviction-rate-geo-core16.pdf}
\vspace{-0.2in}
\caption{By cache geometry (Core16)}
\label{fig:eviction-geo}
\end{subfigure}
\vspace{-0.1in}
\caption{Eviction ratios to the backing store.}
\vspace{-0.25in}
\label{fig:eviction-ratios}
\end{figure}

\Fig{eviction-geo} shows that a full LRU has the lowest eviction ratios, since
the entire LRU must be filled before an eviction occurs. However, the 8-way
associative cache is a good compromise: it avoids the hardware complexity of a
full LRU while coming within 2\% of its eviction ratio.  \Fig{eviction-traces}
shows that the DC trace has the lowest eviction ratios.  This is because it has
much fewer unique keys than the other two traces and these keys are less likely
to be evicted.

The reciprocal of the eviction ratio (as a fraction) is the reduction in server
data collection load relative to a {\em per-packet collector} that processes
per-packet information from routers. For example, for the Core14 trace with a
$2^{19}$ key-value pair cache, the server load reduction is 25$\times$
(corresponding to an eviction ratio of 4\%).

\Para{Eviction rates.}
Eviction ratios are agnostic to specifics of the router, such as link speed,
link utilization, and on-chip cache size in bits. To translate
eviction ratios (evictions per packet) to eviction \emph{rates} (evictions per
second), we first compute the average packet size (700 Bytes) and link utilization
(30\%) from the Core16 trace.

Next, we estimate the on-chip cache size for a \tengrouter router and a
\hundredgrouter router.
On a \tengrouter router, SRAM
densities are $\approx$ 3--4 $\nicefrac{Mbit}{mm^2}$~\cite{sram_45nm_wiki}, and
the smallest router chips occupy 200
\si{\milli\metre\squared}~\cite{gibb_parsing}.  Therefore, a 64 Mbit cache in
SRAM costs around 10\% additional area, which we believe is reasonable.
For recent \hundredgrouter routers~\cite{tomahawk2},
 SRAM densities are $\approx7
\nicefrac{Mbit}{mm^2}$~\cite{sram_estimate}, and the routers occupy
$\approx$ 500 \si{\milli\metre\squared},\footnote{S. Chole. Cisco Systems. Private
communication.  June 2017.} making a 256 Mbit cache (7.3\% area
overhead) a reasonable target.

For a given query, we divide these cache sizes by the size of
the aggregation's key-value pair to get
the number of key-value pairs. We then look up this number in \Fig{eviction-ratios}
to get the eviction ratio for that query, which we translate to an eviction rate
using the network utilization and packet size mentioned earlier.

Eviction rates for some sample queries are shown in \Fig{evt-rate-queries}.
 For a \tengrouter router with a 64 Mbit cache, we observe eviction rates of
$\approx$ 1M packets per second.  For a \hundredgrouter router with a 256 Mbit
cache and the same average packet size and utilization, the eviction rates can
reach
 7.17M packets per second. Relative to a per-packet collector,
\TheSystem reduces the server load by 25--80$\times$.

The eviction rates for both the \tenglink and \hundredglink routers are
under 10M packets per second, well within the capabilities of multi-core
scale-out key-value
stores~\cite{redis_benchmark, memcached_benchmark, redis_vs_memcached_update},
which typically handle 100K--1M operations per second per core.  For instance,
for a single \tengrouter router running an aggregation with a 64-bit
state size, a single 8-core server is sufficient to handle the eviction rate of
1.08M packets per second. For a single \hundredgrouter router running
the same aggregation, the eviction rate goes up to 5.18M packets per
second, requiring four such servers.

\begin{figure}[!t]
\centering
\resizebox{0.5\textwidth}{!}{
%%\begin{tabular}{p{0.15\textwidth} p{0.06\textwidth} p{0.093\textwidth} p{0.093\textwidth} }\hline
\begin{tabular}{llll}\hline
%% {\small \textbf{Query}} & {\small \textbf{State size (bits)}} & {\small
%%   \textbf{Eviction rate at \tenglink (pkts/s)}} & {\small \textbf{Eviction rate
%%     at \hundredglink (pkts/s)}}\\ \hline \hline
  {\textbf{Query}} &
  {\textbf{State size}} &
  {\textbf{Eviction rate at}} &
  {\textbf{Eviction rate at}} \\

  &
  {\textbf{(bits)}} &
  {\textbf{\tengrouter}} &
  {\textbf{\hundredgrouter}} \\

  &
  &
  {\textbf{(packets/s)}} &
  {\textbf{(packets/s)}} \\

  \hline
  \hline
  
Packet count & 32 & 1.0M (34$\times$) & 4.29M (81$\times$) \\ \hline %2.91,1.24 
Lossy connections\nop{\footnote{Lossy Connections requires two stages that count
    by \txtftuple, which we consider as a single key-value store with a concatenated value.}}& 64  & 1.08M (32$\times$) & 5.18M (66$\times$) \\ \hline % 3.14, 1.51 
TCP out-of-sequence & 128 & 1.21M (28$\times$) & 6.72M (52$\times$) \\ \hline % 3.57,1.93
Flowlet size & 160 & 1.26M (27$\times$) & 7.17M (48$\times$) \\ 
histogram (Stage 1) & & & \\ \hline %3.66, 2.08
\end{tabular}
}
\vspace{-0.05in}
\caption{Eviction rates and reduction in collection server
  load for queries from \Fig{example-perf-queries}.
Each key-value pair occupies the listed state size plus 104 bits for a
\txtftuple key.
The 10-Gbit/s and 100-Gbit/s routers have a 64 Mbit and 256 Mbit cache, respectively.}
\label{fig:evt-rate-queries}
\vspace{-0.15in}
\end{figure}

\Para{Generalizing to other scenarios.}
\Fig{eviction-ratios} also generalizes to multiple aggregations and aggregations
of different state sizes. 
First, coarsening the
aggregation key by picking a subset of the \txtftuple reduces the eviction
ratio, since there are fewer keys in the working set. We believe that the
\txtftuple may well be the most fine-grained and still practically useful
aggregation level; hence, our results show the worst-case eviction ratios for a
single {\ct groupby}.  Second, variations in the size of the {\ct groupby}
value simply result in a different number of key-value pairs for a given memory
size. Third, running multiple {\ct groupby} queries with the same number of
key-value pairs, and aggregating by the same key, results in synchronized
evictions across all queries. Hence, the eviction rate can be read off
\Fig{eviction-ratios} at the correspondingly reduced memory size.
%TODO: Maybe the above can be improved.

%%Hence, given an aggregation function, the corresponding eviction rate
%%(in packets per second) can be determined by using the total key-value size to
%%determine the number of available key-value pairs and then interpolating the
%%curves in \Fig{eviction-traces}

\Para{Accuracy of non-mergeable queries.}
Queries that are neither linear-in-state nor associative cannot be merged in
the backing store. If a key from such a query is evicted multiple times,
\TheSystem cannot guarantee its correctness and marks it as invalid. However,
these keys' values are still valid if they are either never evicted or are
evicted once and never reappear. We quantify a query's accuracy as the
fraction of keys with valid values over the query's lifetime.
\Fig{accuracy-traces} shows query accuracy using the three traces, with the DC
trace being near-perfect since it has fewer unique keys, and hence, evictions.
If the query is run over a shorter time interval, its accuracy is typically
higher, since the cache may not be full and a smaller fraction of keys are
evicted.  \Fig{accuracy-time} shows this tradeoff for a range of cache sizes
and geometries using the Core16 trace.  Shortening the query from 5 minutes to
1 minute boosts accuracy by 10\%.

\begin{figure}[ht]
\centering
\vspace{-0.1in}
\begin{subfigure}[t]{0.48\columnwidth}
\raggedright
\includegraphics[width=\linewidth]{pq_accuracy-alltraces.pdf}
\caption{By trace}
\label{fig:accuracy-traces}
\end{subfigure}
\begin{subfigure}[t]{0.48\columnwidth}
\raggedleft
\includegraphics[width=\linewidth]{pq_accuracy-core-geo.pdf}
\caption{By query duration (Core16)}
\label{fig:accuracy-time}
\end{subfigure}
\vspace{-0.1in}
\caption{Query accuracy for non-mergeable aggregations.}
\end{figure}


\subsection{Debugging case study with Mininet}
\label{s:eval:mininet}
\label{sec:eval:mininet}


To demonstrate \TheSystem's use in practice, we present a case study using
Mininet~\cite{mininet}.
%with a topology shown in Figure~\ref{fig:mininet-topo}.
Our topology consists of 4 hosts ({\ct h1, h2, h3, h4}) and 2 routers in a dumbbell topology.
One router is connected to {\ct h1} and {\ct h3}
and the other, to {\ct h2} and {\ct h4}.
The routers are connected via a single link and
programmed in \pfs~\cite{p4-bmv2} with queries compiled by \TheSystem.

Host {\ct h2} repeatedly downloads a 1MB objects over TCP from {\ct h1}.
Meanwhile, {\ct h3} sends {\ct h4} sporadic bursts of UDP traffic, which
{\ct h4} acks.  Suppose a network admin notices the irregular latency
spikes for the TCP traffic (\Fig{mininet-latency}). She suspects a queue buildup
in the routers and measures the queue depths seen by the traffic by writing:
{\ct result = filter(\pktlog, srcip == h1 and dstip == h2).}

% Can remove this figure if we need space.
%\begin{figure}[ht]
%\centering
%\includegraphics[width=0.4\columnwidth]{mininet-topo.pdf}
%\caption{Mininet topology used for the case study.}
%\label{fig:mininet-topo}
%\end{figure}

\begin{figure}[!t]
\centering
\vspace{-0.1in}
\begin{subfigure}[t]{0.48\columnwidth}
\raggedright
\includegraphics[width=\linewidth]{pq_fetch_latency.pdf}
\vspace{-0.2in}
\caption{TCP request latency}
\label{fig:mininet-latency}
\end{subfigure}
\begin{subfigure}[t]{0.48\columnwidth}
\raggedleft
\includegraphics[width=\linewidth]{pq_queue_sizes.pdf}
\vspace{-0.2in}
\caption{Queue depth at egress port 2}
\label{fig:mininet-qin}
\end{subfigure}
\vspace{0.05in}
\caption{Mininet case study measurements.}
\end{figure}
\begin{table}[t]
\centering
\small
\begin{tabular}{|c|c|c|c|c|} \hline
\bf{src $\rightarrow$ dst} & \bf{protocol} & \bf{\# Bursts} & \bf{Time ($\mu$s)} & \bf{\# Packets} \\ \hline
h3:34573 $\rightarrow$ h4:4888 & UDP & 19 & 8969085 & 6090 \\
h4:4888 $\rightarrow$ h3:34573 & UDP & 18 & 10558176 & 5820 \\
h1:1777 $\rightarrow$ h2:58439 & TCP & 1 & 72196926 & 61584 \\
h2:58439 $\rightarrow$ h1:1777 & TCP & 1 & 72248749 & 33074 \\ \hline
\end{tabular}
\caption{Per-flow burst statistics from \TheSystem.}
\label{t:mininet-flowstats}
\vspace{-0.1in}
\end{table}

The results are streamed out on each packet to a collection server. After
plotting the queue latencies, she notices spikes in queue size at egress port 3
on the router (\Fig{mininet-qin}) matching the periodicity of the latency
spikes. To isolate the responsible flow(s), she divides the traffic into
``bursts'', which she defines as a series of packets separated by a gap of at
least 800ms, as determined from the gap between latency spikes. She issues the
following \TheSystem query:

\begin{small}
\begin{lstlisting}
def burst_stats(last_time, nburst, time, pkts, tin):
    if tin - last_time > 800000:
        nbursts++;
        emit();
    else:
        time = time + tin - last_time;
    pkts = pkts + 1;
    last_time = tin;
result = groupby(R1, (*\codeftuple{}*), burst_stats)
\end{lstlisting}
\end{small}

%% NG->Vikram: The burst table is moved to related work to bring it to the head
%% of the next column.

%%\vspace{-0.1in}
She runs the query for 72 seconds and sees the result in
Table~\ref{t:mininet-flowstats}. She concludes, correctly, that UDP traffic
between {\ct h3} and {\ct h4} is responsible for the latency spikes.
There are 18 UDP bursts, with an average size of 320 packets and
average duration of 472 ms, which matches our emulation setup.

\TheSystem's power and flexibility make this diagnosis simple. End host solutions
 are blind to queue contention on the router, and flexible
aggregations expose flow statistics customized for the problem:
packet counts alone would have disguised the bursty nature of
the offending UDP traffic.



%\subsection{[Reach] Redis benchmarking}

%% % Maybe have a list of questions we want to answer to motivate this section
%% % such as FSCQ, section 7 at SOSP15.
%% \subsection{Expressiveness}
%% % TODO: Table for this.
%% % Which queries can be expressed?
%% % Which are linear in state?

%% \subsection{Hardware resources for running queries}
%% % TODO: Write a script for this.
%% % What's the pipeline depth and width?
%% % What atoms beyond linear-in-state are required?

%% \subsection{What is the memory requirement?}
%% % TODO: Write a script for this.
%% % Run trace-driven simulations on CAIDA + Theo Benson's traces.
%% % Maybe run on FB data as well?
%% % Show how linear-in-state significantly reduces memory requirement.
%% % Alternatively show how it reduces load on the backend server.

%% \subsection{How much traffic does the backend server need to process?}

%% % Bombard REDIS with transactions from our corpus at a certain rate.
%% % See how many servers we need to process these transactions.
%% % Exploit the fact that transaction latency can be high so long as transaction throughput is high

%% \subsection{How many servers does this cost in total?}

%% % Get some back of the envelope numbers based on level of aggregation, typical query, REDIS results, exact query etc.
%% % Want a bottomline number like: for x% additional servers, we can have these useful measurement abilities.
%% % Maybe elevate to introduction.

%% \subsection{Mininet evaluations}

%% % ex1: TCP client, server.
%% % UDP workload
%% % Latency spike

%% % ex2: flowlet size histogram

%% % ex3: how long is a high prio packet blocked because a low prio is currently in transmission?

%% % show graphs for each one of these scenarios.
