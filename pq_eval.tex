\section{Evaluation}
\label{sec:eval}

%TODO: Add a scalable column to table
\input{examples}
%The performance of Marple in practice depends on three factors: hardware feasibility, result accuracy, and ease of use.
%Section~\ref{s:eval:hardware} discusses the required hardware atoms required to perform aggregations at line rate. We show that Marple's silicon requirements are modest and realizable with current technology.
%Section~\ref{s:eval:traces} presents a Marple aggregation run on traces of both core Internet and datacenter traffic, to demonstrate how the size of the in-memory key-value store affects the number of flows Marple can accurately track.
%Finally, Section~\ref{s:eval:mininet} presents a case study using Marple to
%diagnose the root cause of irregular HTTP traffic.

We evaluate \TheSystem along three dimensions. In \Sec{eval:hardware}, we show
the switch compute resources used for some candidate \TheSystem queries, while
in \Sec{eval:traces}, we measure the memory-bandwidth tradeoff for the
key-value store. In \Sec{eval:mininet}, we show an end to end use case of
\TheSystem by debugging a performance problem, {\em microbursts,} on the P4
behavioral model.

\input{eval-hardware}
\input{eval-traces}
\input{eval-mininet}

%\subsection{[Reach] Redis benchmarking}

%% % Maybe have a list of questions we want to answer to motivate this section
%% % such as FSCQ, section 7 at SOSP15.
%% \subsection{Expressiveness}
%% % TODO: Table for this.
%% % Which queries can be expressed?
%% % Which are linear in state?

%% \subsection{Hardware resources for running queries}
%% % TODO: Write a script for this.
%% % What's the pipeline depth and width?
%% % What atoms beyond linear-in-state are required?

%% \subsection{What is the memory requirement?}
%% % TODO: Write a script for this.
%% % Run trace-driven simulations on CAIDA + Theo Benson's traces.
%% % Maybe run on FB data as well?
%% % Show how linear-in-state significantly reduces memory requirement.
%% % Alternatively show how it reduces load on the backend server.

%% \subsection{How much traffic does the backend server need to process?}

%% % Bombard REDIS with transactions from our corpus at a certain rate.
%% % See how many servers we need to process these transactions.
%% % Exploit the fact that transaction latency can be high so long as transaction throughput is high

%% \subsection{How many servers does this cost in total?}

%% % Get some back of the envelope numbers based on level of aggregation, typical query, REDIS results, exact query etc.
%% % Want a bottomline number like: for x% additional servers, we can have these useful measurement abilities.
%% % Maybe elevate to introduction.

%% \subsection{Mininet evaluations}

%% % ex1: TCP client, server.
%% % UDP workload
%% % Latency spike

%% % ex2: flowlet size histogram

%% % ex3: how long is a high prio packet blocked because a low prio is currently in transmission?

%% % show graphs for each one of these scenarios.
