\pagebreak
\section{Discussion}
\label{s:real_discussion}

Packet scheduling allocates scarce link capacity across contending flows.  This
allocation services an application or a network-wide objective, e.g., max-min
fairness (WFQ) or minimum flow completion time (SRPT).  Past work has
demonstrated significant performance benefits resulting from switch support
for flexible allocation~\cite{pFabric, baraat, numfabric, nosilverbullet}.
However, these benefits have remained unrealized, because today
there isn't a pathway to implementing such schemes.
PIFOs provide that pathway. They express many scheduling
algorithms and are implementable at line rate.

How a programmable scheduler will be used is still unclear, but at the very
least PIFOs present the network as an additional surface for deploying
resource allocation schemes. No longer will transport protocol designers need
to restrict themselves to end-host/edge-based solutions to datacenter
transport.  As an example of a immediate use case for PIFOs, one could run HPFQ
for traffic isolation, where the classes corresponds to different tenants in a
datacenter and the flows correspond to all source-destination VM pairs
belonging to a tenant. This kind of isolation is much harder to provide with
end-host/edge-based solutions today.
%The network can be an equal participant in transport protocols
%rather than a dumb pipe~\cite{e2e}. 

Looking forward, a programmable scheduler on switches could
simplify packet transport. For instance, pFabric~\cite{pFabric}
minimizes flow completion times by coupling a simple switch
scheduler (shortest remaining processing time) with a simple end-host protocol
(line rate transmission with no congestion control). Other transport mechanisms
~\cite{packetpair, numfabric} leverage fair queueing in the network to simplify
transport and make it more predictable.

That said, our current design is only a first step and can be improved in
several ways.
\begin{CompactEnumerate}
%\item Can we develop an abstraction for output rate limiting that provides
%sensible semantics within the context of a PIFO tree?
% We considered implementing output rate limiting by providing a small
%ALU that computes the next time to dequeue from a flow. 
\item A scheduling tree is more convenient than directly configuring a
PIFO mesh, but it is still a low-level abstraction. Are there higher
level abstractions?
\item Now that we have shown it is feasible, how will programmable scheduling
be used in practice? This would involve surveying network operators to
understand how programmable scheduling could benefit them. In turn, this would
provide valuable design guidance for setting various parameters in our hardware design.
\item Beyond a few counter examples, we lack a formal characterization of
algorithms that cannot be implemented using PIFOs. For instance, is there a
simple, checkable property separating algorithms that can and cannot be
implemented using PIFOs? Given an algorithm specification, can we
automatically check if the algorithm can be programmed
using PIFOs?
\item \an{Our current design scales to 2048 flows. If allocated evenly across
64 ports, we could program scheduling across 32 flows at each port. This
permits per-port scheduling across traffic aggregates (e.g., fair queueing
across 32 tenants within a server), but not a finer granularity (e.g.,
5-tuples). Ideally, to schedule at the finest granularity, our design would
support 60K flows: the physical limit of one flow for each packet. We
currently support up to 2048. Can we bridge this gap?}
\end{CompactEnumerate}
