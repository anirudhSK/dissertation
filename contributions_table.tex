\begin{table}
\textbf{Stateful data-plane algorithms (Chapter~\ref{chap:domino})}
\\[-7pt]\rule{\textwidth}{1pt}\\[-7pt]\rule{\textwidth}{1pt} \\
\textbf{Examples:} in-network congestion control (XCP~\cite{xcp}, RCP~\cite{rcp}), active
queue management (RED~\cite{red}, BLUE~\cite{blue}, CoDel~\cite{codel}) \\
\textbf{Technical challenge:} How do we allow programmable router state modification at
the router's line rate, where a new packet can be received every nanosecond? \\
\textbf{Programming model:} Packet transactions (\S\ref{s:transactions})\\
\textbf{Hardware primitive:} Atoms (\S\ref{s:absmachine}) \\
\textbf{New finding:} A small set of atoms are simultaneously (1) expressive enough to
serve as the instruction set for many stateful algorithms, and (2) simple
enough to implement in high-speed hardware. \\ \\

\textbf{Scheduling algorithms (Chapter~\ref{chap:pifo})}
\\[-7pt]\rule{\textwidth}{1pt}\\[-7pt]\rule{\textwidth}{1pt} \\
\textbf{Examples:} Fair queueing~\cite{wfq}, priority scheduling~\cite{srpt} \\
\textbf{Technical challenge:} Can we find an abstraction that unifies many disparate
scheduling algorithms? \\
\textbf{Programming model:} Scheduling trees (\S\ref{s:pifo}) \\
\textbf{Hardware primitive:} A priority queue data structure we call Push-In First-Out
Queues (PIFOs) (\S\ref{s:design}) \\
\textbf{New finding:} A priority queue of packets with a program to set each packet's
priority can express many scheduling algorithms, and is feasible in high-speed
hardware. \\\\

\textbf{Programmable and scalable per-flow statistics (Chapter~\ref{chap:perf_query})}
\\[-7pt]\rule{\textwidth}{1pt}\\[-7pt]\rule{\textwidth}{1pt} \\
\textbf{Examples:} Per-flow measurements of moving averages, counters, and loss rates \\
\textbf{Technical challenge:} Can we allow programmers to flexibly define the per-flow
statistics they want to measure, while scaling to a large number of flows?\\
\textbf{Programming model:} Performance queries (\S\ref{sec:language}) \\
\textbf{Hardware primitive:} Programmable hardware key-value store. Keys correspond to
flows and values to statistics. (\S\ref{sec:aggregation}) \\
\textbf{New finding:} A class of statistics measurements (linear-in-state) can be scaled
to a large number of flows without losing accuracy.\\
\caption{Contributions of this dissertation}
\label{tab:contributions}
\end{table}

