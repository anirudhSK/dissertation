\section{Related work}
\label{s:related}
\medskip
\noindent
\textbf{Abstract machines for line-rate routers.}
NetASM~\cite{netasm} is an abstract machine and intermediate representation
(IR) for programmable data planes that is portable across network
devices: FPGAs, virtual routers, and line-rate routers.  \absmachine is a
low-level machine model for line-rate routers alone and can be used as a
NetASM target. Because of its role as a low-level machine model, \absmachine
models practical constraints required for line-rate operation (\S\ref{s:atomConstraints}) that an IR like
NetASM doesn't have to. For instance, \absmachine machines don't permit sharing
state between atoms and use atom templates to limit computations that can
happen at line rate.

\medskip
\noindent
\textbf{Programmable data planes.}
Eden~\cite{eden} provides a programmable data plane using commodity routers by
programming end hosts alone. \pktlanguage targets programmable routers that
provide more flexibility relative to an end-host-only solution. For instance,
\pktlanguage allows us to program in-network congestion control and AQM
schemes, which are beyond Eden's capabilities.  Tiny Packet Programs
(TPP)~\cite{tpp} allow end hosts to embed small programs in packet headers,
which are then executed by the router. TPPs use a restricted instruction set to
facilitate router execution; we show that router instructions must and can be
substantially richer (Table~\ref{tab:templates}) for stateful data-plane
algorithms.

Software routers~\cite{routebricks, click} and network processors~\cite{ixp4xx}
are flexible, but at least 10$\times$--100$\times$ slower than programmable
routers~\cite{xpliant, tofino}.  FPGA-based platforms like the Corsa DP
6440~\cite{corsa}, which supports an aggregate capacity of 640 Gbit/s, are
faster, but still 5$\times$--10$\times$ slower than programmable
routers~\cite{tofino, xpliant}.

\medskip
\noindent
\textbf{Programming languages for networks.} Many programming languages target the network control plane~\cite{frenetic, maple}.
\pktlanguage focuses on the data plane, which requires different
programming constructs and compilation techniques.

Several DSLs target the data plane. Click~\cite{click} uses C++ for packet
processing on software routers. packetC~\cite{packetc}, Intel's
auto-partitioning C compiler~\cite{intel_uiuc_pldi}, and Microengine
C~\cite{microenginec} target network processors. \pktlanguage's C-like syntax
and sequential semantics are inspired by these DSLs. However, because it targets
line-rate routers, \pktlanguage is more constrained. For instance, because
compiled programs run at line rate, \pktlanguage forbids loops, and because
\absmachine has no shared state, \pktlanguage has no synchronization constructs.

% TODO: Not sure if the below paragraph should be here.
Jose et al.~\cite{lavanya_compiler} focus on compiling P4 programs to
programmable data planes such as the RMT and FlexPipe architectures. Their work
focuses only on compiling stateless data-plane tasks such as forwarding and
routing, while \pktlanguage handles stateful data-plane algorithms.

\medskip
\noindent
\textbf{Abstractions for stateful packet processing.}
SNAP~\cite{snap} programs stateful data-plane algorithms using a network
transaction: an atomic block of code that treats the entire network as one
router~\cite{onebigrouter}. It then uses a compiler to translate network
transactions into rules on each router. SNAP needs a compiler to compile
these router-local rules to a router's pipeline, and can use \pktlanguage for
this purpose.

FAST~\cite{fast} provides router
support and software abstractions for state machines. \absmachine's atoms
support more general stateful processing beyond state machines that enable a
much wider class of data-plane algorithms.
