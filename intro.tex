\chapter{Introduction}
\label{chap:intro}

\section{Background}

%TODO: Diagram of net. arch.
Computer networks have two classes of elements: the \textit{end hosts} that
generate packets and the \textit{routers}\footnote{We use the term router to
refer to both switches and routers in this disseration.} that forward these
packets between the end hosts. Historically, the Internet was architected so
that most of the complexity resided in the end hosts, while the routers
themselves were simple. According to Clark~\cite{design_philosophy}, this
architecture was a result of the overarching design goal of the Internet: the
ability to easily interconnect existing networks with disparate network
architectures (\eg long-haul networks, local-area networks, satellite networks,
and radio networks) while providing acceptable end-to-end connectivity. Quoting
Clark, "The Internet architecture achieves this flexibility by making a minimum
set of assumptions about the function which the net will provide."

The minimum functionality assumed of and provided by the network was
best-effort and unreliable packet forwarding. Notably absent from a router's
feature set were reliable packet delivery, packet prioritization, monitoring
features to attribute a router's resource usage to specific end hosts, and
security features to detect network breaches. As a result, the early routers
were singularly dedicated to packet forwarding. A minimal router feature set
made it simpler to design high-speed routers and helped broaden the Internet's
reach by interconnecting existing networks with minimum friction. But, it
sidelined other goals~\cite{design_philosophy} such as improving network
performance, security, and monitoring.
 
Today, four decades after ideas underlying the Internet were first
published~\cite{cerf74}, it is clear that routers need to do much more than
forward packets for at least two reasons. First, once the basic goal of
interconnecting different networks is achieved, other goals like performance,
security, and monitoring rise in prominence.  Second, many large-scale private
networks (\eg datacenters, private wide-area networks, enterprise networks) do
not need to concern themselves with interconnecting disparate networks as the
Internet had to and hence can expect more from their network. As a result, a
typical router today implements many features beyond packet forwarding,
pertaining to security (\eg access control), monitoring (\eg counting the
number of packets belonging to each flow transiting the router), and
performance (\eg priority queues).

However, despite the feature creep in routers, there's little consensus between
network operators and router vendors on a router's feature set. Inevitably,
there are network operators whose needs fall outside their router's feature
set. But because today's fastest routers are built out of specialized
forwarding hardware, they are largely {\em fixed-function}\footnote{The term fixed-function
was first used to describe graphics processing units (GPUs) with limited or no programmability~\cite{gpu_fixed}. We use it in an analogous sense here.} in that their
functionality cannot be changed once the router has been built. In such cases,
the operator has no alternative but to wait two--three years for the next
generation of the router hardware---best illustrated by the lag time between
the standardization and availability of new overlay protocols~\cite{vxlan,
nvgre}.

As a result, the rate of innovation in new router algorithms is outstripping
our ability to get these algorithms into production routers, especially those
routers that run at speeds in excess of a Tbit/s.
Figure~\ref{fig:router_algos} shows a timeline of prominent router algorithms
that have been developed since the 1980s. Of these, only a handful are
available in production routers because there is no way to program a new router
algorithm on a production router.
%TODO: What is a production router?

\begin{figure}
\centering
\includegraphics[width=\columnwidth]{router_alg_timeline.pdf}
\caption{Timeline of prominent router algorithms since the 1980s. Only the ones
shaded in blue are available on production routers today.}
\label{fig:router_algos}
\end{figure}

As an operator who wants to introduce new functionality in their network, what
are the operator's choices? One is to give up on changing routers altogether
and make all the required changes at end hosts as the original Internet did.
However, relying solely on end hosts results in solutions that are cumbersome
or suboptimal. As a first example, imagine measuring the queuing latency at a
particular hop in the network. One could do this by collecting end-to-end ping
measurements between a variety of vantage points and then fusing these
measurements together to estimate per-hop queueing latency. Not only is this
indirect, it is also inaccurate relatively to directly instrumenting the router
at that hop to measure its own queueing latency. As a second example, consider
the problem of congestion control, which divides up a network's capacity fairly
among competing users. There are many in-network solutions to congestion
control~\cite{xcp, rcp}, which outperform the end-host-only approaches to
congestion control used today~\cite{cubic, compound}. But, there is no way to
deploy them. 

Another alternative is to use a \textit{software router}: a catch-all term for
a router built on top of some programmable substrate, such as a general-purpose
CPU~\cite{click, routebricks}, a network processor\footnote{A CPU with an
instruction set tailored to packet processing~\cite{ixp4xx, ixp2800}.}, a
graphics processing unit (GPU)~\cite{packetshader}, or a field-programmable gate array (FPGA)~\cite{netfpga}.
Figure~\ref{fig:router_evolution} tracks the evolution of aggregate capacity of
software routers and compares them to the fastest routers known at that point
in time. The figure shows two trends. First, up until the mid 90s, software
routers were in fact the fastest routers; the early routers~\cite{imp} were
minicomputers loaded with forwarding software. Second, since the mid 90s,
growing demands for higher link speeds, fueled by the Internet's growth, have
meant that the fastest routers are now built out of dedicated hardware,
specialized for packet forwarding.

Hardware specialization gives these routers a 10--100 $\times$ performance
improvement relative to software routers.  This performance improvement is the
result of fully exploiting the abundant parallelism available in packet
processing. First, data parallelism, the ability to simultaneously process
either different parts of the same packet or packets belonging to different
ports. Second, pipeline parallelism, the ability to simultaneously perform
different operations on different packets. But, hardware specialization carries
a cost: because routers are built out of specialized hardware, they are
fixed-function devices that can not be reconfigured in the field.

Recent work in software-defined networking~\cite{openflow} (SDN) and
programmable switching chips~\cite{rmt, xpliant, flexpipe} has endowed fast
routers with limited flexibility. SDN allows operators to program the network
control plane, which is the part of the network that computes a network's
routing tables, by moving route computations out of the routers and on to a
programmable server. Programmable switching chips allow operators to program
parts of the data plane, which is the part of the network that forwards packets
based on the routing tables, such as packet header manipulations that do not
modify router state.  However, these solutions are still not sufficient to
express the grayed-out algorithms shown in Figure~\ref{fig:router_algos}
because (1) these algorithms programmatically manipulate router state and (2)
they require flexibility in packet scheduling, which is untouched by both SDN
and programmable switching chips.

\begin{figure}
\centering
\includegraphics[width=\columnwidth]{router_evolution.pdf}
\caption{Aggregate capacity of routers since the first router on the ARPANET in
1969~\cite{imp}. Until the mid 90s, software routers were sufficient. Since
then, however, the fastest routers have been built out of dedicated hardware.}
\label{fig:router_evolution}
\end{figure}

\section{Primary contributions}
\begin{table}
\textbf{Stateful data-plane algorithms (Chapter~\ref{chap:domino})}
\\[-7pt]\rule{\textwidth}{1pt}\\[-7pt]\rule{\textwidth}{1pt} \\
\textbf{Examples:} in-network congestion control (\eg XCP~\cite{xcp} and
RCP~\cite{rcp}), active queue management (\eg RED~\cite{red}, BLUE~\cite{blue},
and CoDel~\cite{codel}) \\
\textbf{Technical challenge:} How do we allow programmable router state
modification at the router's line rate, when a new packet can be received as
often as every nanosecond? \\
\textbf{Programming model:} Packet transactions (\S\ref{s:transactions})\\
\textbf{Hardware primitive:} Atoms (\S\ref{s:absmachine}) \\
\textbf{Key finding:} A small set of atoms (Table~\ref{tab:templates}) is
simultaneously (1) expressive enough to serve as the instruction set for many
stateful algorithms (Table~\ref{tab:algo_atoms}) and (2) feasible
in high-speed hardware (\S\ref{s:eval}). Further, we find that these atoms
can support several new use cases that were unanticipated at the time the atoms
were designed (Table~\ref{tab:atoms_generalize}).\\ \\

\textbf{Scheduling algorithms (Chapter~\ref{chap:pifo})}
\\[-7pt]\rule{\textwidth}{1pt}\\[-7pt]\rule{\textwidth}{1pt} \\
\textbf{Examples:} Weighted Fair Queueing~\cite{wfq} and priority scheduling~\cite{srpt} \\
\textbf{Technical challenge:} Can we find an abstraction that unifies many disparate
scheduling algorithms? \\
\textbf{Programming model:} Scheduling trees (\S\ref{s:pifo}) \\
\textbf{Hardware primitive:} A priority queue data structure called a Push-In First-Out
Queue (PIFO) (\S\ref{s:design}) \\
\textbf{Key finding:} A priority queue of packets with a program to set each
packet's priority can express many scheduling algorithms
(\S\ref{s:expressive}) and is feasible in high-speed hardware
(\S\ref{s:hardware}). \\\\

\textbf{Scalable per-flow statistics (Chapter~\ref{chap:perf_query})}
\\[-7pt]\rule{\textwidth}{1pt}\\[-7pt]\rule{\textwidth}{1pt} \\
\textbf{Examples:} Per-flow measurements of moving averages, counters, and loss rates \\
\textbf{Technical challenge:} Can we allow programmers to flexibly define the
per-flow statistics they want to measure and also scale to a large number of
flows?\\
\textbf{Programming model:} Performance queries (\S\ref{sec:language}) \\
\textbf{Hardware primitive:} Programmable hardware key-value store. Keys correspond to
flows and values to statistics. (\S\ref{sec:aggregation}) \\
\textbf{Key finding:} A class of statistics measurements, which we call the
linear-in-state class (\S\ref{sec:linear-in-state-description}), can be scaled
to a large number of flows without losing accuracy. This class covers many
practically useful statistics such as counters, moving average filters, and
conditional counters (\S\ref{sec:eval}). \\
\caption{Contributions of this dissertation}
\label{tab:contributions}
\end{table}


This dissertation considers the problem of building routers that approach the
speeds of today's fastest fixed-function routers, while also being
programmable. My thesis is that {\em it is possible to design router hardware
that is both fast and programmable, if we restrict ourselves to programming
specific classes of router functionality}. It is this specificity that allows
us to resolve the programmability-performance tension; indeed, our designs
provide a much more restricted form of programmability than a Turing-complete
processor.  The challenge here is to pick classes of router functionality that
are simultaneously (1) practically useful to network operators, (2) broad
enough to cover a range of current and future use cases within that class, and
yet (3) narrow enough to permit a high-speed hardware implementation. We will
describe high-speed programmable hardware primitives and their corresponding
programming models in software for three classes of router functionality:
stateful data-plane algorithms, packet scheduling, and scalable network
measurement.  Table~\ref{tab:contributions} summarizes our contributions.

%Consider adding a crisp statement of limitations of each class of algorithms
%Packet payload processing, Algorithms with loop bounds only known at run time \\

\subsection{Stateful data-plane algorithms}
We first consider the problem of programming {\em stateful data-plane
algorithms} at high packet processing rates. These are algorithms that operate
on a sequeunce of packets in a streaming manner, doing a bounded amount of work
per packet and manipulating a bounded amount of router state in the process.
They include algorithms for managing the router's buffer (\eg RED~\cite{red},
BLUE~\cite{blue}), load balancing(\eg CONGA~\cite{conga}, Flare~\cite{flare}),
and in-network congestion control (\eg XCP~\cite{xcp}, RCP~\cite{rcp}).

High-speed data-plane programming poses two challenges: (1) what hardware
instructions are required to support programmable state modification at the
router's line rate and (2) what is the right programming model? To address
these challenges, we develop a system for data-plane programming, Domino, which
contains three main components:
\begin{CompactEnumerate}
\item \textit{Atoms} capture a router's instruction set. They specify atomic
units of packet processing provided by the router hardware, \eg an atomic
counter or an atomic test-and-set. Atoms are atomic in the sense that if some
state is updated by an atom as part of processing a packet, the next packet
arriving at that atom will see the updated value of that state.
%TODO: Add a figure of an atom from your slides here.
\item \textit{Packet Transactions} provide a programming model for data-plane
algorithms. A packet transaction is an atomic and isolated block of code
capturing an algorithm's logic written in a domain-specific language (DSL)
called Domino. Packet transactions provide the semantics that any visible state
is equivalent to a serial execution of packet transactions in the order of
packet arrival---akin to an infinitely fast single-threaded router. Packet
transactions are expressive and capture many important data-plane algorithms.
Further, their serial semantics shield programmers from the hardware's
parallelism. Packet transactions have been adopted in P4~\cite{p4}, a
packet-processing language that is emerging as an industry standard for
programming router chips.  P4 programmers can now use an @atomic annotation
around a block of statements to specify that the block must execute atomically.

%TODO: Need to clarify packet transaction semantics better. It's messy right now.
%TODO: Add a figure of a packet transaction from your slides here.
\item \textit{The Domino compiler}, which compiles packet transactions written
in the Domino DSL to a pipeline of atoms provided by the router, and rejects
the code if the router's atoms cannot support the packet transaction. The
compiler provides an {\em all-or-nothing} guarantee: if the compiler compiles
the packet transaction it will run at the line rate of the router; all other
code that cannot run at line rate will be rejected. A conventional compiler for
a general-purpose CPU compiles all programs, but a program's run-time
performance depends on its complexity. Here, only programs that are simple
enough to run at the router's line rate will even be compiled, obviating the
need for any performance profiling.
\end{CompactEnumerate}

%TODO: Add figure of this iterative process from powerpoint slides
Developing atoms for a router is a chicken-and-egg problem. A router's atoms
determine what algorithms the router can support, while the algorithms
determine what atoms are required in the first place. Designing the right atoms
is especially important for a programmable line-rate router because---unlike a
general-purpose CPU---there is no way to ``emulate'' functionality in software
when hardware support is not available.

\begin{figure}
\centering
\includegraphics[width=0.5\columnwidth]{iterative_design_process.pdf}
\caption{Iterative atom design process}
\label{fig:iterative_design}
\end{figure}

To develop atoms, we use the Domino compiler to experiment with different atoms
and iteratively modify the atoms until they support enough algorithms
(Figure~\ref{fig:iterative_design}).  Using this process, we developed seven
atoms of increasing complexity (Table~\ref{tab:templates}) that allow us to
progressively program more and more algorithms from
Figure~\ref{fig:router_algos}. For instance, measurement using Bloom filters
only requires simple read and write operations on state. On the other hand,
heavy-hitter detection~\cite{opensketch} uses a counting
sketch~\cite{count_min} that relies on a atomic counter. Finally, algorithms
like flowlet switching~\cite{flare} require conditional updates to counters.

In two subsequent projects~\cite{hula, perf_query} that we carried out after
the Domino work, we found that we could reuse the same atoms for other use
cases, providing evidence that these atoms generalize and are useful beyond the
algorithms that influenced their design in the first place.

\subsection{Packet scheduling}

Packet scheduling is an important determinant of network performance. The
choice of scheduling algorithm is tied to a network's overarching goals. For
instance, an algorithm that divides link capacity fairly is ideal in a
multi-tenant setting~\cite{wfq}, while the shortest remaining processing time
algorithm is ideal for a single tenant who desires low flow completion
time~\cite{pFabric}. Today's routers provide a fixed set of scheduling
algorithms and do not allow an operator to program scheduling to suit their
needs.

Routers lack programmable scheduling because there is no single abstraction to
express many scheduling algorithms. Push In First Out Queues (PIFOs) provide
such an abstraction. They exploit the fact that in many practical schedulers,
the relative order of packets that are already buffered does not change in
response to new packet arrivals. Put differently, when a packet arrives, it can
be pushed into the right location based on a packet priority (push in), but
packets can always be dequeued from the head (first out). The PIFO abstraction
is simple: a priority queue of packets with a small program to assign each
packet its priority. Yet, by flexibly programming a packet's priority
assignment, a network operator can use the PIFO abstraction to program a
variety of previously proposed scheduling algorithms. So far, these algorithms
could only be run in simulation or on slower speed software routers.

As some examples of the expressivness of PIFOs, a single PIFO can express many
classical scheduling algorithms, \eg token bucket shaping, weighted fair
queueing, and strict priority scheduling. Further, PIFOs can be combined to
express hierarchical schedulers. Finally, PIFOs are feasible in hardware: a
hardware design for a programmable 5-level hierarchical scheduler costs less
than 4\% additional chip area.

\subsection{Scalable network measurement}

%% TODO: Functional query language.
%% TODO: Formally characterize linear-in-state operations here.
%% TODO: Maybe give an example of why merging is hard in general and why there's
%% something non-trivial here.

\TheSystem (Chapter~\ref{chap:perf_query}) provides a query language and
supporting router hardware for network performance measurement queries. As
examples, an operator could ask for TCP flows with a high degree of packet
reordering or a moving average of packet latencies for each flow. Our query
language allows order-dependent aggregation of information across packets
belonging to a single flow (\eg an exponentially weighted moving average over
packet latencies), while traditional query languages (\eg SQL) only support
order-independent aggregates like counts and averages. We designed a
programmable key-value store that runs in the router's ASIC to support these
aggregations.  The keys represent flows, while the values store and
programmatically update per-flow state (\eg counters or a moving average
filter) on every packet.

At the heart of our key-value store is a technique to aggregate information
across packets belonging to a flow, while scaling to a large number of flows.
In order to scale to a large number of flows, we use a split design for our
key-value store with an on-chip cache within the router's ASIC supported by a
backing store sitting outside the router. A traditional cache would fetch the
correct value from the backing store on a miss and incur variable access
latencies in the process. Instead, our cache treats a cache miss as a packet
from a new flow. When this flow is eventually evicted from the cache, it is
merged with the current value for this flow in the backing store.  We
mathematically characterize the set of aggregation functions that permit such
merging, and show that it includes many useful aggregation functions such as
rolling minimums, rolling maximums, counters, predicated counters, statistics
computed over a bounded window of packets, and moving average filters.

\section{Towards a world of programmable networks}

The results in this dissertation point towards a world of programmable
networks, where network operators \textit{tell} routers what to do, without
being \textit{told} that this is all that the router can do. In such a setting,
operators could customize networks as they see fit.  They could program
additional features that give them performance benefits.  More interestingly,
they could remove needless features from their router, allowing them to
simplify their routers' feature sets, which in turn could ease troubleshooting
when things go wrong.

Besides the benefits to network operators, a programmable router chip has
benefits for router vendors as well. Router vendors can now add new features in
firmware and sell different versions of the firmware to different market
segments. Further, when bugs arise, it is much easier to fix these bugs in
firmware, as opposed to redoing the hardware design for the router, which could
easily take years. It also allows vendors to respond to new requests from
network operators in a period of days as opposed to years.

Whether the classes of router programmability described here will be sufficient
and future proof remains to be seen, but we are encouraged by the fact that the
hardware designs proposed here support a wide range of existing use cases and
several new use cases that we had not anticipated initially. We hope these
results provide guidance to router chip manufacturers when designing hardware
for programmable routers.
