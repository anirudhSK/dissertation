%\section{Introduction}
\label{sprout:intro}

Cellular wireless networks have become a dominant mode of Internet
access. These mobile networks, which include LTE and 3G (UMTS and
1xEV-DO) services, present new challenges for network applications,
because they behave differently from wireless LANs and from the
Internet's traditional wired infrastructure.

Cellular wireless networks experience rapidly varying link rates
and occasional multi-second outages in one or both directions,
especially when the user is mobile. As a result, the time it takes to
deliver a network-layer packet may vary significantly, and may include
the effects of link-layer retransmissions. Moreover, these networks
schedule transmissions after taking channel quality into account, and
prefer to have packets waiting to be sent whenever a link is
scheduled. They often achieve that goal by maintaining deep packet
queues. The effect at the transport layer is that a stream of packets
experiences widely varying packet delivery rates, as well as variable,
sometimes multi-second, packet delays.


%To ameliorate these issues, these networks use semi-reliable protocols
%with link-layer retransmissions and enqueue packets waiting to be
%transmitted so that the link, when available, does not starve for
%packets.  As a result, these networks experience high variations in
%achievable transmission rates as well as packet delays.

For an interactive application such as a videoconferencing program
that requires both high throughput and low delay, these conditions are
challenging. If the application sends at too low a rate, it will waste
the opportunity for higher-quality service when the link is
doing well. But when the application sends too aggressively, it
accumulates a queue of packets inside the network waiting to be
transmitted across the cellular link, delaying subsequent
packets. Such a queue can take several seconds to drain, destroying
interactivity (see Figure~\ref{fig:skypevssprout}).

%Furthermore, even if the transport protocol knew the varying rate of
%the network path exactly, a videoconferencing application would not
%benefit unless it could rapidly adapt the quality of its video
%encoding to meet the available rate.  Unfortunately, this task is
%difficult and done poorly by existing videoconferencing
%applications. To match a time-varying rate, the application needs to
%control the coded size of each outgoing video frame. To help the
%client recover from loss, the application needs to ensure that
%transmitted video frames depend only on frames available to the
%receiver, and not frames that have been lost. Current video coder
%libraries generally don't expose this level of control to
%applications.

Our experiments with Microsoft's Skype, Google's Hangout, and Apple's
Facetime running over traces from commercial 3G and LTE networks show
the shortcomings of the transport protocols in use and the lack of
adaptation required for a good user experience.
%(\S\ref{s:problem}). 
The transport protocols deal with rate variations in a reactive
manner: they attempt to send at a particular rate, and if all goes
well, they increase the rate and try again. They are slow to
decrease their transmission rate when the link has deteriorated, and as a
result they often create a large backlog of queued packets in the
network. When that happens, only after several seconds and a
user-visible outage do they switch to a lower rate.
%The
%encoder pre-determines which frames should be sent at any given
%encoding rate, not adapting to what the receiver might (or might not)
%have recently received.

%\subsection{Contributions}

\begin{figure}
\caption{Skype and Sprout on the Verizon LTE downlink trace. For Skype, overshoots in throughput lead to large standing queues. Sprout tries to keep each
packet's delay less than 100~ms with 95\% probability.}
\label{fig:skypevssprout}
\vspace{\baselineskip}
\def\svgwidth{\columnwidth}\input{newfigure1.pdf_tex}
\end{figure}

%\paragraph{Contributions.}
This chapter describes Sprout, a transport protocol designed for
interactive applications on variable-quality networks. Sprout uses the
receiver's observed packet arrival times as the primary signal to
determine how the network path is doing, rather than the packet loss,
round-trip time, or one-way delay. Moreover, instead of the
traditional reactive approach where the sender's window or rate
increases or decreases in response to a congestion signal, the Sprout
receiver makes a short-term forecast (at times in the near future) of
the bottleneck link rate using probabilistic inference.  From this
model, the receiver predicts how many bytes are likely to cross the
link within several intervals in the near future with at least 95\%
probability. The sender uses this forecast to transmit its data,
bounding the risk that the queuing delay will exceed some threshold,
and maximizing the achieved throughput within that constraint.

%This is difficult for existing videoconferencing systems. To match the
%capacity of a varying link, the application needs to control the coded
%size of each outgoing video frame. To help the client recover from
%loss, the application needs to ensure that transmitted video frames
%depend only on frames available to the receiver (and not frames that
%have been lost). Current video coder libraries generally don't expose
%this level of control to calling applications.
%
%In our experiments using trace-based emulation of cellular networks,
%we have found that videoconferencing programs --- including
%Microsoft's Skype, Google's Hangout, and Apple's Facetime --- operate
%poorly over commercial 3G and LTE service. Based on our measurements,
%these programs deal with varying link speeds in a reactive manner.
%They attempt to send at a particular rate. If all goes well, after
%several seconds they increase the rate in a large jump and try
%again. If they are accumulating a large delay, again only after
%several seconds, and a user-visible outage, does the program switch to
%a lower rate.

%\begin{enumerate}a

%\item 


%  In practice, Sprout responds more quickly than existing schemes to
%  increases or decreases in path rates because it does not need to
%  wait for detectable changes in the observed round-trip time or for
%  losses. Scout generally achieves higher throughput and lower delay
%  than competing end-to-end schemes.

%  Sprout sends as much data as it can---subject to
%   the constraint that no outgoing packet will end up waiting behind
%   other packets in a network queue for more than 100
%   milliseconds. Sprout can't guarantee this, but it does its best to
%   bound the risk of violating the constraint at less than 5\% per
%   packet.

%  The Sprout receiver maintains a stochastic model of the cellular
%  link, and uses probabilistic inference to evolve a probability
%  density function on the network's underlying link speed. From this
%  probability distribution and model, Sprout's receiver calculates a
%  conservative packet delivery forecast --- an estimate of how many
%  bytes will cross the wireless link in various intervals in the near
%  future, with at least some minimum probability, such as 5\%. This
%  forecast is sent to the sender.

%  The Sprout sender uses the forecast and transmits as much data as
%  possible, while bounding the risk that any packet will have to wait
%  in a network queue for more than 100 milliseconds (or another
%  threshold for good interactivity). The goal is to achieve as much
%  throughput as possible, while keeping the induced delay below 100~ms
%  with at least 95\% probability.


%\item 

%The second contribution is a video coder/decoder (codec) interface
%that can smoothly vary its quality on a frame-by-frame basis to match
%Sprout's network capacity forecasts, and can recover swiftly from
%losses (\S\ref{s:alf}). Alfalfa treats compressed video not so much as
%a stream to be played back in order, but as a collection of ``diffs''
%between arbitrary frames that may be marshaled to advance the receiver
%to the current video frame in the most efficient manner.

%We propose a new, simple interface to low-delay video codecs, in which
% With this interface, the application can request a ``diff'', of a
% specified size, between any previous frame and coding state serving as
% a reference (one presumed to be available at the receiver) and the
% most recent frame. This ``diff'' can then be applied at the receiver
% to update the screen and serve as a reference for future frames.

%  In
%  contradistinction to other schemes, Alfalfa does not use golden
%  frames, high-quality key frames, or a predetermined frame
%  interdependency relationship (such as an I-P-B picture cadence).

%  Instead, we propose a new interface to low-delay video codecs, in
%  which the application can request a ``diff'' of a specified coded
%  frame size, between any previous frame serving as a reference (one
%  presumed to be available to the receiver) and the most recent
%  frame. This ``diff'' can then be applied at the receiver to update
%  the screen and serve as a reference for future frames.

%  In the event of losses, the sender will base its ``diffs'' on older
%  frames that have been acknowledged by the receiver until a newer
%  frame has been acked, all with the goal of advancing the receiver to
%  the ``current'' frame as quickly as possible. The main benefit of
%  this scheme is that network feedback can immediately influence frame
%  interdependency relationships, so there is less danger of
%  transmitting a frame that is predicated on a previous frame that has
%  been lost.

%  Another advantage is that no frames are more important or need to be
%  larger than other frames. ``Lumpiness'' in coded frame sizes (e.g.,
%  in a stream whose I-frames or golden frames are larger than
%  predicted frames) is unfortunate because it generally increases
%  user-experienced latency, as larger frames take longer to send, or
%  causes underutilization when sending smaller frames.

%\end{enumerate}

%Alfalfa varies its frame rate and bit rate continuously, obeying the
%Sprout algorithm's estimate of how much data the network can bear
%without amassing a queue. Using Sprout's forecasts and the explicit
%modeling of delay, Alfalfa is able to use the available network rate
%even when that rate is highly variable.

%Alfalfa represents an end-to-end, application-layer solution to the
%challenges presented by cellular networks: variable (and
%unpredictable) link speeds and delays caused by packets waiting to
%transit a wireless link with link-layer acknowledgments and
%retransmissions.  We have implemented Alfalfa's codec interface on top
%of existing encoder and decoder implementations for MPEG-2 part 2
%(H.262) video (\S\ref{s:impl}), but our scheme is more generally
%applicable (see \S\ref{s:alf}).

%to any DPCM-based video compression method, including H.264 (MPEG-4
%part 10 / AVC), VP8/WebM, and the forthcoming H.265.

%\subsection{Summary of Results}

% To evaluate Sprout's ability to match network performance, we
% saturated the uplink and downlink of four commercial LTE and 3G
% networks: Verizon Wireless's LTE and 3G (1xEV-DO / eHRPD) services,
% AT\&T's LTE service, and T-Mobile's 3G (UMTS) service. We drove around
% the Boston area while recording the timings of packet arrivals from
% each network, with about 18 minutes from each. Later, we replayed
% these traces to emulate the same network conditions for pairs of
% computers running Skype, Hangout, Facetime, and several bulk-transfer
% protocols. These experiments are described in detail in
% \S\ref{s:eval}.

%\paragraph{Summary of results.}
We conducted a trace-driven experimental evaluation (details in
\S\ref{sprout:eval}) using data collected from four different commercial
cellular networks (Verizon's LTE and 3G 1xEV-DO, AT\&T's LTE, and
T-Mobile's 3G UMTS). We compared Sprout with Skype, Hangout, Facetime,
and several TCP congestion-control algorithms, running in both
directions (uplink and downlink).

Figure~\ref{f:sproutcompe2e} summarizes the average relative throughput
improvement and reduction in self-inflicted queueing
delay\footnote{This metric expresses a lower bound on the amount of
  time necessary between a sender's input and receiver's output, so
  that the receiver can reconstruct more than 95\% of the input
  signal. We define the metric more precisely in \S\ref{sprout:eval}.} for
Sprout compared with the various other schemes, averaged over all four
cellular networks in both directions.

\begin{figure}
\caption{Sprout compared with other end-to-end schemes. Metrics where
  Sprout did not outperform the existing algorithm are highlighted in
  red.}
\label{f:sproutcompe2e}

\begin{center}
\noindent \begin{tabular}{|l|c|c|}
\hline
App/protocol & Avg.~speedup & Delay reduction \\
& \footnotesize{vs.~Sprout} & \footnotesize{(from avg.~delay)}\\
\hline
\hline
\cellcolor{blue!20}Sprout & \cellcolor{blue!20}$1.0\times$ & \cellcolor{blue!20}$1.0\times$ (0.32~s) \\
\hline
Skype & $2.2\times$ & $7.9\times$ (2.52~s) \\
Hangout & $4.4\times$ & $7.2\times$ (2.28~s) \\
Facetime & $1.9\times$ & $8.7\times$ (2.75~s) \\
\hline
Compound & $1.3\times$ & $4.8\times$ (1.53~s) \\
TCP Vegas & $1.1\times$ & $2.1\times$ (0.67~s) \\
LEDBAT & $1.0\times$ & $2.8\times$ (0.89~s) \\
Cubic & \cellcolor{red!20}$0.91\times$ & $79\times$ (25~s)\\
\hline
Cubic-CoDel & \cellcolor{red!20}$0.70\times$ & $1.6\times$ (0.50~s) \\
%CUBIC/CoDel & & \\
%Compound/CoDel & & \\
\hline
\end{tabular}
\end{center}

\end{figure}

Cubic-CoDel indicates TCP Cubic running over the CoDel
queue-management algorithm~\cite{CoDel}, which would be implemented in the
carrier's cellular equipment to be deployed on a downlink, and in
the baseband modem or radio-interface driver of a cellular phone for an
uplink.

We also evaluated a simplified version of Sprout, called Sprout-EWMA,
that estimates the network's future packet-delivery rate with a simple
exponentially-weighted moving average, rather than using its
probabilistic model to make a cautious forecast of future packet
deliveries with 95\% probability.

Sprout and Sprout-EWMA represents different tradeoffs in their
preference for throughput versus delay.  As expected, Sprout-EWMA
achieved greater throughput, but also greater delay, than Sprout. It
outperformed TCP Cubic on both throughput and delay. Despite being
end-to-end, Sprout-EWMA outperformed Cubic-over-CoDel on throughput
and approached it on delay (Figure~\ref{f:sproutvsaqm}).

\begin{figure}
\caption{Sprout and Sprout-EWMA achieve comparable results to Cubic
  assisted by in-network active queue management.}
\label{f:sproutvsaqm}

\begin{center}
\noindent \begin{tabular}{|l|c|c|}
\hline
Protocol & Avg.~speedup & Delay reduction \\
& \footnotesize{vs.~Sprout-EWMA} & \footnotesize{(from avg.~delay)}\\
\hline
\hline
\cellcolor{blue!20}Sprout-EWMA & \cellcolor{blue!20}$1.0\times$ & \cellcolor{blue!20}$1.0\times$ (0.53~s) \\
\hline
Sprout & $2.0\times$ & \cellcolor{red!20}$0.60\times$ (0.32~s) \\
Cubic & $1.8\times$ & $48 \times$ (25~s)\\
\hline
Cubic-CoDel & 1.3 $\times$ & \cellcolor{red!20}$0.95\times$ (0.50~s)\\
\hline
\end{tabular}

\end{center}

\end{figure}

% Alfalfa achieved a bit rate 2.2$\times$ higher than Skype, while
% reducing self-inflicted end-to-end delay by 87\%, or from 2,519~ms to
% 317~ms on average,  while averaging a 118\% increase in achievable bit
% rate. Compared with Facetime, which averaged a 2,752~ms self-inflicted
% delay, Sprout's delay was 88\% lower, and Sprout increased throughput
% by 89\%. Against Hangout, Sprout achieved 4.4x the bit rate while
% reducing delay by 86\%, from 2,279~ms.

% We also compared Sprout against the delay-based bulk transfer
% protocols Compound TCP, LEDBAT, and TCP Vegas. Sprout achieved 79\%
% less delay than Compound TCP (with an average speedup of 26\%), 64\%
% less delay than LEDBAT (with no speedup or slowdown on average), and
% 52\% less delay than TCP Vegas (with a 10\% average speedup).

%Finally, we compared Sprout against Linux's TCP CUBIC default, with
%and without the use of CoDel AQM on the emulated bottleneck
%link. CUBIC's delays were gigantic (25 seconds) over a link that
%queued packets indefinitely, and Sprout achieved a 14\% speedup over
%CUBIC over such links.
%
%But adding AQM to the bottleneck link made CUBIC perform the best of
%any comparator. Sprout (without AQM) still achieved lower delay, but
%only by 36\%. On average, Sprout had 30\% less throughput than
%CUBIC-over-CoDel.

We also tested Sprout as a tunnel carrying competing traffic over a
cellular network, with queue management performed at the tunnel
endpoints based on the receiver's stochastic forecast about future
link speeds. We found that Sprout could isolate interactive and bulk
flows from one another, dramatically improving the performance of
Skype when run at the same time as a TCP Cubic flow.

The source code for Sprout, our wireless network trace capture
utility, our trace-based network emulator, and instructions to
reproduce the experiments in this chapter are available at
\url{http://alfalfa.mit.edu/} .
%\texttt{http://github.com/keithw/alfalfa}.
