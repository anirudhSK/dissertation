\chapter{Limitations}
\label{chap:limitations}

One major limitation shared by all three systems is that we do not have silicon
implementations of any of them. Instead, we evaluated them using a combination
of simulation and microbenchmarks. We briefly considered an FPGA implementation
on the NetFPGA platform~\cite{netfpga}, but decided against it because the
relative areas consumed by router subsystems on an FPGA do not accurately
reflect the relative area consumptions on a router ASIC.  That said, we paid
careful attention to keeping the hardware designs simple, clearly articulated
the interfaces between the different hardware blocks, and evaluated the area
consumed by our hardware designs by synthesizing them to a recent transistor
library. We hope that the results here make a strong case for router chip
manufacturers to design router chips that support the systems in this
dissertation.

\begin{CompactEnumerate}
\item Our compiler doesn't aggressively optimize. For instance, it is possible
to fuse two stateful codelets incrementing two independent counters into the
same instance of the Pairs atom. However, by carrying out a one-to-one mapping
from codelets to the atoms implementing them, our compiler precludes these
optimizations.  Developing an {\em optimizing} compiler for packet transactions
is an area for future work.

\item Supporting multiple packet transactions in \pktlanguage also requires
further work. When a switch executes multiple transactions, there may be
opportunities for inter-procedural analysis~\cite{dragonbook}, which goes
beyond compiling individual transactions and looks at multiple transactions
together.  For instance, the compiler could detect computations common to
multiple transactions and execute them only once.

\item Finally, we have a manual design process for atoms.  Formalizing this
design process and automating it into an atom-design tool would be useful for
switch designers. For instance, given a corpus of data-plane algorithms, can we
automatically mine this corpus for stateful and stateless codelets, and design
an atom (or atoms) that captures the computations required by some (or all) of
them?

\item Beyond a few counter examples, we lack a formal characterization of
algorithms that cannot be implemented using PIFOs. For instance, is there a
simple, checkable property separating algorithms that can and cannot be
implemented using PIFOs? Given an algorithm specification, can we
automatically check if the algorithm can be programmed
using PIFOs?

\item Our current PIFO design scales to 2048 flows. If allocated evenly across
64 ports, we could program scheduling across 32 flows at each port. This
permits per-port scheduling across traffic aggregates (e.g., fair queueing
across 32 tenants within a server), but not a finer granularity (e.g.,
5-tuples). Ideally, to schedule at the finest granularity, our design would
support 60K flows: the physical limit of one flow for each packet. We currently
support up to 2048. Can we bridge this gap?
\end{CompactEnumerate}
