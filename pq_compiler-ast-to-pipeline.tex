\subsection{Query AST to pipeline configuration}
\label{sec:pipeline-layout}

%TODO: We don't actually say how this router-local query AST was generated.
This compiler pass first generates a sequence of operators from the
router-local query AST of \Sec{network-to-router-local}. This sequence of
operators will then be used in the same order to generate a router pipeline
configuration. There are two aspects that require care when constructing a
pipeline structure: (1) the pipeline should respect read-write dependencies
between different operators, and (2) repeated subqueries should not create
additional pipeline stages. We generate a sequence through a post-order
traversal of the query AST, which guarantees that the operands of a node are
added into the pipeline before the operator in the node. Further, we
deduplicate subquery results from the pipeline to avoid repeating stages in the
final output. For the running example, the algorithm produces the sequence of
operators: {\ct tcps} ({\ct filter}) $\rightarrow$ {\ct tslots} ({\ct map})
$\rightarrow$ {\ct joined} ({\ct zip}) $\rightarrow$ {\ct oos} ({\ct groupby}).

%TODO: How do you turn the sequence into a pipeline?
Next, the compiler emits P4 code for a router pipeline
from the operator sequence.  The {\ct filter} and {\ct zip} configuration just
involves checking a predicate and setting a ``valid'' bit on the packet
metadata. The {\ct map} configuration assigns a packet metadata field to the
computed expression. The {\ct groupby} configuration uses a register that is
indexed by the aggregation fields, and is updated through the action specified
in the aggregation function. We transform \TheSystem aggregation functions into
straight-line code consisting of C-style conditional operators through a
standard procedure known as if-conversion~\cite{if-conversion}. This allows us
to fit the aggregation function into the body of a single P4 action.

To target the Banzai router pipeline simulator~\cite{domino_sigcomm}, the
\TheSystem compiler emits Domino code by concatenating C-like code fragments
from all pipeline stages into a single Domino program. The Domino compiler then
takes this program and compiles it to a pipeline of Banzai {\em atoms.} Atoms
are ALUs representing a programmable router's instruction set. Atoms implement
either stateless (\eg incrementing a packet field) or stateful (\eg atomically
incrementing a router counter) computations.

