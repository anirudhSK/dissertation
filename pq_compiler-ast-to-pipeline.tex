\subsection{Query AST to pipeline configuration}
\label{sec:pipeline-layout}

This compiler pass first generates a sequence of operators from the
router-local query AST of \S\ref{sec:network-to-router-local}. This sequence of
operators is then used in the same order to generate a router pipeline
configuration. There are two aspects that require care when constructing a
pipeline structure: (1) the pipeline should respect read-write dependencies
between different streams, and (2) repeated subqueries should not be
re-executed by creating additional pipeline stages.

We generate a sequence of operators through a post-order traversal of the query
AST, which guarantees that the operands of a node are added into the pipeline
before the operator in the node, thereby respecting read-write dependencies.
Further, we deduplicate subquery results from the pipeline to avoid repeating
stages in the final output. For the running example, the algorithm produces the
sequence of operators: {\ct tcps} ({\ct filter}) $\rightarrow$ {\ct tslots}
({\ct map}) $\rightarrow$ {\ct joined} ({\ct zip}) $\rightarrow$ {\ct oos}
({\ct groupby}).

Next, the compiler emits P4 code for a router pipeline from the operator
sequence.  The {\ct filter} and {\ct zip} configuration just involves checking
a predicate and setting a ``valid'' bit on the packet metadata. The {\ct map}
configuration assigns a packet metadata field to the computed expression. The
{\ct groupby} configuration uses a P4 register that is indexed by the
aggregation fields. The state located at a particular register index is updated
through a P4 action representing the aggregation function.

%%We transform
%%\TheSystem aggregation functions into straight-line code consisting of C-style
%%conditional operators through a standard procedure known as
%%if-conversion~\cite{if-conversion}. This allows us to fit the aggregation
%%function into the body of a single P4 action.

To target the Banzai machine model (\S\ref{s:absmachine}), the \TheSystem
compiler emits imperative code fragments for each pipeline stage and then
concatenates these fragments together into a single Domino program. The Domino
compiler then takes this program and compiles it to a pipeline of Banzai atoms.
