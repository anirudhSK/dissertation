\subsection{Query AST to pipeline configuration}
\label{sec:pipeline-layout}

This compiler pass first generates a {\em linear sequence of operators} from the
switch-local query AST of \Sec{network-to-switch-local}. The operator sequence
will then be used in the same order to generate a switch pipeline
configuration. There are two aspects which require care when constructing a
pipeline structure: (i) the pipeline should respect read-write dependencies
between different operators, and (ii) repeated subqueries should not create
additional pipeline stages. We generate a chain through a post-order traversal
of the query AST, which guarantees that the operands of a node are added into
the pipeline before the operator in the node. Further, we deduplicate subquery
results from the pipeline to avoid repeating stages in the final output. For the
running example, the algorithm produces the chain of operators: {\ct tcps} ({\ct
  filter}) $\rightarrow$ {\ct tslots} ({\ct map}) $\rightarrow$ {\ct joined}
({\ct zip}) $\rightarrow$ {\ct oos} ({\ct groupby}).

Next, the compiler emits \pfs configuration for each of the pipeline stages. The
{\ct filter} and {\ct zip} configuration just involves checking a predicate and
setting a ``valid'' bit on the packet metadata. The {\ct map} configuration
assigns a packet metadata field to the computed expression. The {\ct groupby}
configuration uses a register which is indexed by the aggregation fields, and is
updated through the action specified in the aggregation function. We transform
\TheSystem aggregation functions into straight-line code consisting of ternary
statements through a standard procedure known as
if-conversion~\cite{if-conversion}. This allows us to fit the aggregation
function into the body of a single \pfs action.
%% to fit it into an acceptable \pfs action
%%
%%%All
%% query-related temporary packet fields are collected into a packet metadata
%% structure. We report on our experience running \TheSystem queries end-to-end
%% with \pfs on the behavioral model in \S\ref{s:eval:mininet}.

Further, the \TheSystem compiler emits Domino code by collecting C-like code
fragments from all pipeline stages together into a stateful program. The Domino
compiler targets the Banzai switch pipeline simulator.
%%% by compiling this
%%% program. %% Banzai can be configured to run with specific stateless and stateful
%% hardware instructions, and the Domino compiler determines whether the stateful
%% input program can map to a pipeline with those instructions. The \TheSystem
%% compiler can emit Domino code that leverages the new multiply-accumulate
%% instruction; we elide the details.

%% Next, we generate a pipeline layout from a query AST containing the functional
%% operators (\ie {\ct map, groupby, } \etc). We generate match-action stages for
%% {\ct filter, map} and {\ct zip} operators, and key-value store stages for {\ct
%%   groupby}s.

%% Generating a pipeline from a tree requires some care to respect read-write
%% dependencies between operators. The function {\sc getPipe} in
%% \Alg{pipeline-construction} shows how the compiler constructs the overall
%% pipeline structure through a post-order traversal of the query AST (line
%% \ref{line:postorder-traversal}). Since there may be repeated subqueries in the
%% tree, the pipeline constructor is only invoked for nodes that haven't built a
%% pipeline (line \ref{line:operand-unvisited}) through a previous call to {\sc
%%   getPipe.} The final output of the algorithm is just a list of operators
%% respecting read-write dependencies, and without redundancies from repeated
%% subqueries.

%% \begin{algorithm}
%%   \caption{Query AST to pipeline.}
%%   \label{alg:pipeline-construction}
%%   \begin{algorithmic}[1]
%%     \State visited $\gets$ [] \Comment{Empty set of visited queries.}
%%     \Function{getPipe}{{\cta Q}}
%%       \If{{\cta Q} $\notin$ visited}
%%       \State pipeQ $\gets$ [] \Comment{Empty pipeline.}
%%       \For{each operand {\cta R} of {\cta Q}}\label{line:postorder-traversal}
%%         \If{{\cta R} $\notin$ visited}\label{line:operand-unvisited}
%%           \State pipeQ += {\sc getPipe}({\cta R})
%%         \EndIf
%%       \EndFor
%%       \State pipeQ += {\cta Q}\label{line:config-gen-Q}
%%       \State add {\cta Q} to visited
%%       \State \textbf{return} pipeQ
%%       \EndIf
%%     \EndFunction
%%   \end{algorithmic}
%% \end{algorithm}

%% In \Sec{linear-in-state-compilation} and \Sec{emitting-p4-domino-code}, we delve
%% deeper into the feasibility of hardware implementation for the pipeline stages
%% emitted by this compiler pass. We focus mainly on generating code for
%% aggregation functions in \Sec{linear-in-state-compilation};
%% \Sec{emitting-p4-domino-code} discusses code generation for the rest of the
%% operators.

