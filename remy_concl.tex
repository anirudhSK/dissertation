\section{Conclusion}
\label{remy:concl}

This paper asks whether the design of distributed congestion-control
algorithms for heterogeneous and dynamic networks can be done by
specifying the assumptions that such algorithms are entitled to have
and the policy they ought to achieve, and letting computers work out
the details of the per-endpoint mechanisms.

Much future work remains before this question can be answered for
the real-world Internet, but our findings suggest that this approach
has considerable potential.

We developed and evaluated Remy, a program that designs end-to-end
congestion-control algorithms to human-supplied specifications. Remy's
outputs handily outperform the best-known techniques, including ones
that require intrusive in-network changes, in scenarios where network
parameters varied over one or two orders of magnitude.

Our results, and many others in the literature, indicate that there is
no existing single congestion-control method that is the best in all
situations. Moreover, the set of ``all situations'' is rapidly growing
as new subnetworks and link technologies proliferate.  A
computer-generated approach that maximizes an explicit function of the
throughput and delay to generate algorithms may be the right way
forward for the networking community.  Today's informal approach of
hampering lower layers or providing vague advice on how best to
accommodate TCP should be replaced by end-to-end algorithms (in TCP
and elsewhere) that adapt to {\em whatever} the lower layers are
doing.  Remy provides a way to achieve this goal.
