\section{Related Work}
\label{remy:related}

Starting with Ramakrishnan and Jain's DECBit scheme~\cite{decbit} and
Jacobson's TCP Tahoe (and Reno) algorithms~\cite{Jacobson88},
congestion control over heterogeneous packet-switched networks has
been an active area of research. End-to-end algorithms typically
compute a congestion window (or, in some cases, a transmission rate)
as well as the round-trip time (RTT) using the stream of
acknowledgments (ACKs) arriving from the receiver. In response to
congestion, inferred from packet loss or, in some cases, rising
delays, the sender reduces its window; conversely, when no congestion
is perceived, the sender increases its window.

There are many different ways to vary the window. Chiu and
Jain~\cite{chiujain} showed that among linear methods, additive
increase / multiplicative decrease (AIMD) converges to high
utilization and a fair allocation of throughputs, under some
simplifying assumptions (long-running connections with synchronized
and instantaneous feedback). Our work relaxes these assumptions to
handle flows that enter and leave the network, and users who care
about latency as well as throughput. Remy's algorithms are not
necessarily linear, and can use both a window and a rate pacer to
regulate transmissions.

In this paper, we compare Remy's generated algorithms with several
end-to-end schemes, including NewReno~\cite{newreno},
Vegas~\cite{vegas}, Compound TCP~\cite{compound}, Cubic~\cite{cubic},
and DCTCP for datacenters~\cite{dctcp}. NewReno has the same
congestion-control strategy as Reno---slow start at the beginning, on
a timeout, or after an idle period of about one retransmission timeout
(RTO), additive increase every RTT when there is no congestion, and a
one-half reduction in the window on receiving three duplicate ACKs
(signaling packet loss). We compare against NewReno rather than Reno
because NewReno's loss recovery is better.

Brakmo and Peterson's Vegas is a delay-based algorithm, motivated by
the insight from Jain's CARD scheme~\cite{card} and Wang and Crowcroft's
DUAL scheme~\cite{dual} that increasing RTTs may be a congestion
signal.  Vegas computes a BaseRTT, defined as the RTT in the absence
of congestion, and usually estimated as the first RTT on the
connection before the windows grow. The expected throughput of the
connection is the ratio of the current window size and BaseRTT, if
there is no congestion; Vegas compares the {\em actual} sending rate,
and considers the difference, {\em diff}, between the expected and
actual rates.  Depending on this difference, Vegas either increases
the congestion window linearly ({\em diff} $< \alpha$), reduces it
linearly ({\em diff} $> \beta$), or leaves it unchanged.

Compound TCP~\cite{compound} combines ideas from Reno and Vegas: when
packet losses occur, it uses Reno's adaptation, while reacting to
delay variations using ideas from Vegas. Compound TCP is more
complicated than a straightforward hybrid of Reno and Vegas; for
example, the delay-based window adjustment uses a binomial
algorithm~\cite{binomial}. Compound TCP uses the delay-based window to
identify the absence of congestion rather than its onset, which is a
key difference from Vegas.

Rhee and Xu's Cubic algorithm is an improvement over their previous
work on BIC~\cite{bic}. Cubic's growth is independent of the RTT (like
H-TCP~\cite{htcp}), and depends only on the packet loss rate,
incrementing as a cubic function of ``real'' time. Cubic is known to
achieve high throughput and fairness independent of RTT, but it also
aggressively increases its window size, inflating queues and bloating
RTTs (see \S\ref{remy:eval}).

Other schemes developed in the literature include equation-based
congestion control~\cite{ebcc}, binomial control~\cite{binomial},
FastTCP~\cite{fasttcp}, HSTCP, and TCP Westwood~\cite{westwood}.

End-to-end control may be improved with explicit router participation,
as in Explicit Congestion Notification (ECN)~\cite{ecn},
VCP~\cite{vcp}, active queue management schemes like
RED~\cite{Floyd93}, BLUE~\cite{BLUE}, CHOKe~\cite{choke},
AVQ~\cite{AVQ}, and CoDel~\cite{CoDel} fair queueing, and explicit
methods such as XCP~\cite{xcp} and RCP~\cite{rcp}.  AQM schemes aim to
prevent persistent queues, and have largely focused on reacting to
growing queues by marking packets with ECN or dropping them even
before the queue is full. CoDel changes the model from reacting to
specific average queue lengths to reacting when the delays measured
over some duration are too long, suggesting a persistent
queue. Scheduling algorithms isolate flows or groups of flows from
each other, and provide weighted fairness between them.  In XCP and
RCP, routers place information in packet headers to help the senders
determine their window (or rate). One limitation of XCP is that it
needs to know the bandwidth of the outgoing link, which is difficult
to obtain accurately for a time-varying wireless channel.

In \S\ref{remy:eval}, we compare Remy's generated algorithm with XCP and
with end-to-end schemes running through a gateway with the CoDel AQM and
stochastic fair queueing (sfqCoDel).

TCP congestion control was not designed with an explicit optimization
goal in mind, but instead allows overall network behavior to emerge
from its rules. Kelly et al.~present an interpretation of various TCP
congestion-control variants in terms of the implicit goals they
attempt to optimize~\cite{Kelly98}.  This line of work has become
known as Network Utility Maximization (NUM); more recent work has
modeled stochastic NUM problems~\cite{stochasticnum}, in which flows
enter and leave the network. Remy may be viewed as combining the
desire for practical distributed endpoint algorithms with the explicit
utility-maximization ethos of stochastic NUM.

We note that TCP stacks have adapted in some respects to the changing
Internet; for example, increasing bandwidth-delay products have
produced efforts to increase the initial congestion
window~\cite{dukkipati2010argument,chu2012increasing}, including
recent proposals \cite{allman2010init,touch2012automating} for this
quantity to automatically increase on the timescale of months or
years. What we propose in this paper is an automated means by which
TCP's entire congestion-control algorithm, not just its initial
window, could adapt in response to empirical variations in underlying
networks.



%Remy is a distributed POMDP-style method... Say something more here.
