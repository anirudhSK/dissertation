\begin{abstract}

When designing a distributed network protocol, typically it is
infeasible to fully define the target network where the protocol is
intended to be used. It is therefore natural to ask: How faithfully do
protocol designers really need to understand the networks they design
for? What are the important signals that endpoints should
listen to?  How can researchers gain confidence that systems that work
in testing scenarios will also perform adequately on real networks
that are inevitably more complex, or future networks yet to be
developed? Is there a tradeoff between the performance of a protocol
and the breadth of its intended operating range of networks? What is
the cost of playing fairly with cross-traffic that is governed by
another protocol?

We examine these questions quantitatively in the context of congestion
control, by using an automated protocol-design tool to approximate the
best possible congestion-control scheme given imperfect prior
knowledge about the network.  We found only weak evidence of a tradeoff
between operating range and performance, even when operating range was
extended to cover a thousand-fold range of link rates. We found that
it may be acceptable to simplify some characteristics of the
network---such as its topology---when modeling for design
purposes. Some other features, such as the degree of multiplexing and
the aggressiveness of contending endpoints, are important to capture
in a model.
\end{abstract}
