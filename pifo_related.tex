\section{Related Work}
\label{s:related}

\medskip
\noindent
\textbf{The Push-in First-out Queue.}
\an{PIFOs were first introduced as a proof construct to prove that a
combined input-output queued router could exactly emulate an output-queued
router~\cite{pifo}. We show here that PIFOs can be used as an abstraction for
programmable scheduling at line rate.}

\medskip
\noindent
\textbf{Packet scheduling algorithms.}
The literature is replete with scheduling algorithms~\cite{pFabric, hpfq,
stopngo, stfq, lstf, srpt, drr, rcsd} . Yet, line-rate routers support only a
few: DRR, traffic shaping, and strict priorities. As \S\ref{s:expressive}
shows, PIFOs allow a line-rate router to run many of these scheduling
algorithms, which, so far, have only been run on software routers.

\medskip
\noindent
\textbf{Programmable routers.} Recent work has proposed hardware architectures~\cite{tofino, flexpipe,
xpliant, rmt} and software abstractions~\cite{p4, domino_sigcomm} for
programmable routers.  While many packet-processing tasks can be programmed on
these routers, scheduling isn't one of them. Programmable routers can {\em
assist} a PIFO-based scheduler by providing a programmable ingress pipeline for
scheduling and shaping transactions, without requiring a dedicated atom
pipeline inside each PIFO block.  However, they still need PIFOs for
programmable scheduling.

\medskip
\noindent
\textbf{Universal Packet Scheduling (UPS).} UPS~\cite{ups} shares our goal of
flexible packet scheduling by seeking a single scheduling algorithm that is
{\em universal} and can emulate any scheduling algorithm. Theoretically, UPS
finds that the well-known LSTF scheduling discipline~\cite{lstf} is universal
if packet departure times for the scheduling algorithm to be emulated are known
up front. Practically, UPS shows that by appropriately initializing slacks, many different scheduling objectives can be
emulated using LSTF. LSTF is programmable using PIFOs, but the set of schemes
practically expressible with LSTF is limited. For example, LSTF cannot
express:
\begin{CompactEnumerate}
\item Hierarchical scheduling algorithms such as HPFQ, because it
  uses only one priority queue.
\item Non-work-conserving algorithms. For such algorithms LSTF must know the
  departure time of each packet up-front, which is not practical.
\item Short-term bandwidth fairness in fair queueing, because LSTF maintains no
  router state except one priority queue. As shown in
  Figure~\ref{fig:sched_trans}, programming a fair queueing algorithm requires us
  to maintain a virtual time state variable. Without this, a new flow could have
  arbitrary virtual start times, and be deprived of its fair share indefinitely.
  UPS provides a fix to this that requires
  estimating fair shares periodically, which is hard to do in
  practice.
\item Scheduling policies that aggregate flows from distinct endpoints into a
  single flow at the router. An example is fair queueing across video and web
  traffic classes, regardless of endpoint.  Such policies require the router to
  maintain the state required for fair queueing because no end point sees all the
  traffic within a class.  However, LSTF cannot maintain and update router state
  progammatically.
\end{CompactEnumerate}
\an{The restrictions in UPS/LSTF are a result of a limited programming
model. UPS assumes that routers are fixed and cannot be programmed to modify
packet fields. Further, it only has a single priority queue.  By using atom
pipelines to execute scheduling and shaping transactions, and by composing
multiple PIFOs together, PIFOs express a wider class of scheduling algorithms.}

%\begin{figure}
%  \centering
%  \includegraphics[width=0.7\columnwidth]{state_reqd.pdf}
%  \caption{A router's scheduling algorithm, such as WFQ, might aggregate flows
%  from different end hosts into a single flow at the router for the purpose of
%  scheduling.}
%  \label{fig:state}
%\end{figure}

\medskip
\noindent
\textbf{Hardware designs for priority queues.}
\an{P-heap is a pipelined binary heap scaling to 4-billion entries~\cite{bhagwan,
pheap}.  However, each P-heap supports traffic belonging to a {\em single} 10
Gbit/s input port in an input-queued router and there is a separate P-heap
instance for each port~\cite{bhagwan}.  This per-port design incurs prohibitive
area overhead on a shared-memory router, and prevents sharing of the data
buffer and binary heap across output ports. Conversely, it isn't easy to
overlay multiple logical PIFOs over a single P-heap, which would allow the
P-heap to be shared across ports.}
%%
%%\an{
%%In contrast to a hardware implementation of a generic priority queue as a heap,
%%our design for the PIFO exploits two domain-specific insights. First, there is
%%considerable structure in the ranks: ranks within a flow strictly increase with
%%time.  Second, the packet buffers on shared-memory routers used in datacenters
%%today are much smaller than those on deep-buffered core routers in the past.
%%This permits a simpler, albeit less scalable, design relative to heaps.
%%}
