This chapter focuses on the hardware and software required to program {\em
stateful data-plane algorithms} on high-speed routers. These algorithms process
and transform packets, reading and writing state in the router. Examples
include active queue management~\cite{red,avq,codel}, congestion control with
router feedback~\cite{xcp, rcp}, network measurement~\cite{opensketch,
bitmap_george}, and data-plane traffic engineering~\cite{conga, flowlets}.
Hence, the central question of this chapter is: {\em how do we enable
programmable stateful packet processing on a high-speed router?} Concretely,
what is the right instruction set for programmable state modification, what is
the the right programming model for stateful algorithms, and how do we compile
from the programming model to the instruction set?

In answering these questions, this chapter makes three new contributions.  Our
first contribution is {\em \absmachine}, a machine model for programmable
high-speed routers~(\S\ref{s:absmachine}).  \absmachine is inspired by existing
programmable router chips, which provide programmable, but stateless packet
transformations. \absmachine extends such chips with functionality required for
high-speed and programmable state modifications.  Specifically, \absmachine
models two important constraints (\S\ref{s:atomConstraints}) for stateful
operations at the router's line rate: the inability to share state between
different packet-processing units and the requirement that any router state
modifications be visible to the next packet entering the router. Based on these
constraints, we introduce {\em atoms} to represent a programmable router's
packet-processing units.

Our second contribution is a new abstraction to program data-plane algorithms
called {\em packet transactions} (\S\ref{s:transactions}). A packet transaction
is a sequential code block that is atomic and isolated from other such code
blocks.  Packet transactions provide programmers with the illusion that the
transaction's body executes serially from start to finish on each packet in the
order in which packets arrive at the router's ingress or egress pipeline.
Conceptually, when programming with packet transactions, there is no overlap in
packet processing across packets---akin to an infinitely fast single-threaded
processor carrying out packet processing on each packet. The code within a
packet transaction is written in {\em \pktlanguage{}}, a new imperative
domain-specific language (DSL). To our knowledge, \pktlanguage is the first to
offer such a high-level programming abstraction for high-speed routers.

Packet transactions let the programmer focus on the operations needed for each
packet without worrying about other concurrent packets. Packet transactions
have an \textit{all-or-nothing} guarantee: all packet transactions accepted by
the packet transactions compiler will run at the router's line rate, or be
rejected.  There is no ``slippery slope'' of running network algorithms at
lower speeds as with network processors or software routers: when compiled, a
packet transaction runs at the router's line rate, or not at all.  Performance
is not just predictable, but guaranteed.

Our third contribution is a compiler from \pktlanguage packet transactions to a
\absmachine target~(\S\ref{s:compiler}). The \pktlanguage compiler extracts
{\em codelets} from  transactions: code fragments, which if executed
atomically, guarantee a packet transaction's semantics. It then uses program
synthesis~\cite{sketch_asplos} to map codelets to atoms, rejecting the
transaction if the atom cannot execute the codelet.

We evaluate \pktlanguage's expressiveness by programming a variety of
data-plane algorithms (Table~\ref{tab:algorithms}) in \pktlanguage. We find
that \pktlanguage provides a concise and natural programming model for stateful
data-plane algorithms.  Next, we design a set of small set of atoms that can
express these algorithms (\S\ref{ss:targets}).  We show that compiler targets
with these atoms are feasible at a 1 GHz clock rate in a 32-nm standard-cell
library with $< 2\%$ cost in area relative to a 200 \si{\milli\metre\squared}
baseline router chip~\cite{gibb_parsing}.

We compile data-plane algorithms written in \pktlanguage to these targets
(\S\ref{domino_ss:compiler}) to show how a target's atoms determine the
algorithms it can support (Table~\ref{tab:algo_atoms}). To study the generality
of our atoms, we froze the design of our atoms and compiled a new set of
\pktlanguage algorithms to our atoms. As Table~\ref{tab:atoms_generalize}
shows, our atoms were able to generalize to new and unanticipated use cases,
providing empirical evidence of their programmability. Based on our experiences
with \pktlanguage, we distill several lessons for programmable router design
(\S\ref{ss:lessons}).

Code for the \pktlanguage compiler, the \absmachine machine model, and the code
examples listed in Tables~\ref{tab:algorithms} and \ref{tab:atoms_generalize} is available at
\url{http://web.mit.edu/domino}.
