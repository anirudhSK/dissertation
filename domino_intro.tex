This chapter focuses on the hardware and software required to program {\em
stateful data-plane algorithms} on high-speed routers. These algorithms process
and transform packets, reading and writing state in the router. Examples
include active queue management~\cite{red,avq,codel}, congestion control with
router feedback~\cite{xcp, rcp}, network measurement~\cite{opensketch,
bitmap_george}, and data-plane traffic engineering~\cite{conga, flowlets}.

As explained earlier (\S\ref{s:intro_background} and
Figure~\ref{fig:router_algos}), such algorithms are either not available on
high-end routers or are hardwired into fixed-function routers where they cannot
be modified. In other words, there is no way to program a new algorithm or
modify the algorithmic logic of an old one on a high-speed router. While
high-speed programmable router chips~\cite{flexpipe, xpliant} are available,
their programmability is largely restricted to {\em stateless} tasks such as
packet forwarding (\S\ref{ss:prog_router_chips}). By contrast to packet
forwarding, which doesn't modify state in the data plane, many data-plane
algorithms create and modify {\em algorithmic state} in the router as part of
packet processing.

The central question of this chapter is: {\em how do we enable programmable
stateful processing on a high-speed router?} In more detail, what is the right
programming model for stateful algorithms, what is the underlying instruction
set, and how do we compile from the programming model to the instruction set?

In answering these questions, this chapter makes three new contributions.
First, {\em \absmachine}, a machine model for programmable line-rate
routers~(\S\ref{s:absmachine}).  \absmachine is inspired by existing
programmable router chips, but extends them with functionality required for
state modifications in the router's ASIC.  Specifically, \absmachine models two
important constraints (\S\ref{s:atomConstraints}) for stateful line-rate
operations: the inability to share state between different packet-processing
units, and the requirement that any router state modifications be visible to
the next packet entering the router. Based on these constraints, we introduce
{\em atoms} to represent a programmable router's packet-processing units.

Second, a new abstraction to program and implement data-plane algorithms: a
{\em packet transaction} (\S\ref{s:transactions}). A packet transaction is a
sequential code block that is atomic and isolated from other such code blocks.
Packet transactions provide programmers with the illusion that the
transaction's body executes serially from start to finish on each packet, with
no overlap in packet processing across packets---akin to an infinitely fast
single-threaded processor carrying out packet processing on each packet. The
code within a packet transaction is written in {\em \pktlanguage{}}, a new
imperative domain-specific language (DSL) for (\S\ref{s:transactions}).
\pktlanguage is  to our knowledge the first to offer such a high-level
programming abstraction for line-rate routers.

Packet transactions let the programmer focus on the operations needed for each
packet without worrying about other concurrent packets. Packet transactions
have an \textit{all-or-nothing} guarantee: all packet transactions accepted by
the packet transactions compiler will run at line rate, or be rejected.  There
is no ``slippery slope'' of running network algorithms at lower speeds as with
network processors or software routers: when compiled, a packet transaction
runs at the router's line rate, or not at all.  Performance is not just
predictable, but guaranteed.

Third, {\em a compiler from \pktlanguage packet transactions to a \absmachine
target}~(\S\ref{s:compiler}). The \pktlanguage compiler extracts {\em codelets}
from  transactions: code fragments, which if executed atomically, guarantee a
packet transaction's semantics. It then uses program
synthesis~\cite{sketch_asplos} to map codelets to atoms, rejecting the
transaction if the atom cannot execute the codelet.

We evaluate \pktlanguage's expressiveness by programming a variety of
data-plane algorithms (Table~\ref{tab:algorithms}) in \pktlanguage and compare with
P4. We find that \pktlanguage provides a more concise and natural programming
model for stateful data-plane algorithms.  Next, we design a set of small set
of atoms that can express these algorithms (\S\ref{ss:targets}).  We show that
compiler targets with these atoms are feasible in a 32-nm standard-cell library
with $< 2\%$ cost in area relative to a 200 \si{\milli\metre\squared} baseline
router chip~\cite{gibb_parsing}.  Finally, we compile data-plane algorithms
written in \pktlanguage to these targets (\S\ref{domino_ss:compiler}) to show
how a target's atoms determine the algorithms it can support (Table~\ref{tab:algo_atoms}). We also distill
with several lessons for programmable router design (\S\ref{ss:lessons}) based
on our experience with \pktlanguage.
%TODO: Tone down comparison with P4 here. That's not the point here.

Code for the \pktlanguage compiler, the \absmachine machine model, and the code
examples listed in Table~\ref{tab:algorithms} is available at
\url{http://web.mit.edu/domino}.
