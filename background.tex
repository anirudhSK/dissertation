% An abridged history of programmability in routers and how it relates to this dissertation

% The early days of routers: 1969 through the mid 90s
% Everything was in software

% Active Networks: mid to late 90s
% Execute programs carried by packets

% Software routers and NPUs: 99 to present: Click (99), IXPs (2001), RouteBricks, PacketShader, NetFPGA, etc., and associated languages for software routers (packetC, Intel's C dialects, etc.).
% Turn a CPU/GPU/FPGA/multi-core into a router

% Software-defined networking (2008 to now): SANE, Ethane, OpenFlow, and associated languages (Pyretic, Frenetic, etc.). Could incorporate OpenSketch, FAST, DevoFlow, etc. here
% Unified interface to high-speed switching chips
% Promise: this unified interface will solve all our programmability problems

% One possibility for SDNv2.0: Fabric-based SDN (HotSDN 2013), edge-core split
% Too much feature creep in OpenFlow; propose that routers just do source routing, edge does rest

% Another possibility for SDNv2.0: Programmable switching chips (2013): XPliant, FlexPipe, Tofino, RMT, Gibbs parsing paper
% Actually make routers programmable, but restricted to match-action processing and parsing

% Concurrent work (since 2013): FastPass/Flexplane, TPP, INT, SNAP, UPS, UnivMon, SONATA etc.
