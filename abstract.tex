% $Log: abstract.tex,v $
% Revision 1.1  93/05/14  14:56:25  starflt
% Initial revision
% 
% Revision 1.1  90/05/04  10:41:01  lwvanels
% Initial revision
% 
%
%% The text of your abstract and nothing else (other than comments) goes here.
%% It will be single-spaced and the rest of the text that is supposed to go on
%% the abstract page will be generated by the abstractpage environment.  This
%% file should be \input (not \include 'd) from cover.tex.

Historically, the evolution of network routers was driven primarily by
performance. Recently, owing to the need for better control over network
operations and the constant demand for new features, programmability of routers
has become as important as performance.  However, today's fastest routers,
which run at line rate, use fixed-function hardware, which cannot be modified
after deployment. This disseration will describe three router primitives we
have developed to build programmable routers at line rate.

The first is a programmable packet scheduler. The second is a way to program
stateful packet-processing algorithms to manage network resources. The third is
a design to measure programmer-defined statistics, such as counters and moving
average filters, on a per-flow basis, while supporting a large number of flows.
Together, these primitives allow us to program several packet-processing
functions at line rate for the first time, such as in-network congestion
control, active queue management, data-plane load balancing, network
measurement, and packet scheduling.
