% $Log: abstract.tex,v $
% Revision 1.1  93/05/14  14:56:25  starflt
% Initial revision
% 
% Revision 1.1  90/05/04  10:41:01  lwvanels
% Initial revision
% 
%
%% The text of your abstract and nothing else (other than comments) goes here.
%% It will be single-spaced and the rest of the text that is supposed to go on
%% the abstract page will be generated by the abstractpage environment.  This
%% file should be \input (not \include 'd) from cover.tex.

Historically, the evolution of network routers was driven primarily by
performance. Recently, owing to the need for better control over
network operations and the constant demand for new features,
programmability of routers has become as important as performance.
However, today's fastest routers, which run at line rate, use
fixed-function hardware, which cannot be modified after deployment. I
will describe two router primitives we have developed to build
programmable routers at line rate. The first is a programmable packet
scheduler. The second is a way to execute stateful packet-processing
algorithms to manage network resources. Together, these primitives
allow us to program several packet-processing functions at line rate,
such as in-network congestion control, active queue management,
data-plane load balancing, network measurement, and packet scheduling.

%TODO: Add performance queries.

%A third system, Mosh (mobile shell), is a remote-terminal application
%and protocol that supports roaming, intermittent connectivity, and
%rolling latency compensation. Mosh explicitly models the application
%state at the client and server, allowing it to decide at runtime how
%to most efficiently keep the client up to date, based on application
%objectives and semantics.
