% $Log: abstract.tex,v $
% Revision 1.1  93/05/14  14:56:25  starflt
% Initial revision
% 
% Revision 1.1  90/05/04  10:41:01  lwvanels
% Initial revision
% 
%
%% The text of your abstract and nothing else (other than comments) goes here.
%% It will be single-spaced and the rest of the text that is supposed to go on
%% the abstract page will be generated by the abstractpage environment.  This
%% file should be \input (not \include 'd) from cover.tex.
Historically, the evolution of network routers was driven primarily by
performance. Recently, owing to the need for better control over network
operations and the constant demand for new features, programmability of routers
has become as important as performance.  However, today's fastest routers,
which have 10--100 ports each running at a line rate of 10--100 Gbit/s, use
fixed-function hardware, which cannot be modified after deployment. This
disseration describes three router hardware primitives and their
corresponding software programming models that allow network operators to
program specific classes of router functionality on such fast routers.

First, we develop a system for programming stateful packet-processing
algorithms such as algorithms for in-network congestion control, buffer
management, and data-plane traffic engineering. The challenge here is the fact
that these algorithms maintain and update state on the router.  We develop a
small but expressive instruction set for stateful manipulation on fast routers,
and then expose this to the programmer through a high-level programming model
and compiler.

Second, we develop a system to program packet scheduling: the task of picking
which packet to transmit next on a link. Our main contribution here is the
finding that many packet scheduling algorithms can be programmed using one
simple idea: a priority queue of packets in hardware coupled with a software
program to assign each packet's priority in the priority queue.

Third, we develop a system for programmable and scalable measurement of network
statistics. Our goal is to allow programmers to flexibly define what they want
to measure for each flow while scaling to a large number of flows. We formalize
a class of statistics that permit a scalable
implementation and show that it includes many useful statistics (\eg
moving averages and counters).

These systems show that it is possible to program several packet-processing
functions on fast routers for the first time. Based on these systems, we
distill two lessons for designing fast and programmable routers in the future.
First, specializing designs to programming specific classes of router
functionality improves performance by 10x or more relative to a general-purpose
solution that can program any router functionality. Second, joint design of
hardware and software provides us with more leverage relative to designing only
one of them while keeping the other fixed.
