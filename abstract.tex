% $Log: abstract.tex,v $
% Revision 1.1  93/05/14  14:56:25  starflt
% Initial revision
% 
% Revision 1.1  90/05/04  10:41:01  lwvanels
% Initial revision
% 
%
%% The text of your abstract and nothing else (other than comments) goes here.
%% It will be single-spaced and the rest of the text that is supposed to go on
%% the abstract page will be generated by the abstractpage environment.  This
%% file should be \input (not \include 'd) from cover.tex.
In the Internet architecture, transport protocols are the glue between
an application's needs and the network's abilities. But as the
Internet has evolved over the last 30 years, the implicit assumptions
of these protocols have held less and less well. This can cause poor
performance on newer networks---cellular networks, datacenters---and
makes it challenging to roll out networking technologies that break
markedly with the past.

Working with collaborators at MIT, I have built two systems that
explore an objective-driven, computer-generated approach to protocol
design. My thesis is that making protocols a \emph{function} of stated
assumptions and objectives can improve application performance and
free network technologies to evolve.

Sprout, a transport protocol designed for videoconferencing over
cellular networks, uses probabilistic inference to forecast network
congestion in advance. On commercial cellular networks, Sprout gives
2-to-4 times the throughput and 7-to-9 times less delay than Skype,
Apple Facetime, and Google Hangouts.

This work led to Remy, a tool that programmatically generates
protocols for an uncertain multi-agent network. Remy's
computer-generated algorithms can achieve higher performance and
greater fairness than some sophisticated human-designed schemes,
including ones that put intelligence inside the network.

The Remy tool can then be used to probe open questions of Internet
congestion control---e.g., what is the cost of maintaining backwards
compatibility with existing algorithms? Is there a tradeoff between a
protocol's performance now and its ability to adapt to networks of the
future? How easy is it to ``learn'' a network protocol to achieve
desired goals, given a necessarily imperfect model of the networks
where it ultimately will be deployed?

%A third system, Mosh (mobile shell), is a remote-terminal application
%and protocol that supports roaming, intermittent connectivity, and
%rolling latency compensation. Mosh explicitly models the application
%state at the client and server, allowing it to decide at runtime how
%to most efficiently keep the client up to date, based on application
%objectives and semantics.
