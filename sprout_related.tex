\section{Related Work}
\label{sprout:related}

\paragraph{End-to-end algorithms.}
%Although a large number of congestion control algorithms have been
%developed over the past three decades, 

Traditional congestion-control algorithms generally do not
simultaneously achieve high utilization and low delay over paths with
high rate variations.
%They also often introduce long and variable
%delays.
Early TCP variants such as Tahoe and Reno~\cite{Jacobson88} do not
explicitly adapt to delay (other than from ACK clocking), and require
an appropriate buffer size for good performance. TCP
Vegas~\cite{Brakmo94}, FAST~\cite{FAST}, and Compound TCP~\cite{CTCP}
incorporate round-trip delay explicitly, but the adaptation is
reactive and does not directly involve the receiver's observed rate.

LEDBAT~\cite{ledbat} (and TCP Nice~\cite{tcpnice}) share our goals of
high throughput without introducing long delays, but LEDBAT does not
perform as well as Sprout. We believe this is because of its choice of
congestion signal (one-way delay) and the absence of forecasting.
Some recent work proposes TCP receiver modifications to combat
bufferbloat in 3G/4G wireless networks~\cite{tcpbufferbloat}.
%We note
%that it is not enough to reactively throttle the receiver window when
%delays rise, because one needs mechanisms to predict rate increases
%and outages as well.
Schemes such as ``TCP-friendly'' equation-based
rate control~\cite{ebcc} and binomial congestion
control~\cite{binomial} exhibit slower transmission rate variations
than TCP, and in principle could introduce lower delay, but perform
poorly in the face of sudden rate changes~\cite{slowcc-dynamic}.

Google has proposed a congestion-control scheme~\cite{WebRTCdraft} for
the WebRTC system that uses an arrival-time filter at the receiver,
along with other congestion signals, to decide whether a real-time
flow should increase, decrease, or hold its current bit rate. We plan
to investigate this system and assess it on the same metrics as the
other schemes in our evaluation.

%In contrast to all these schemes, Sprout uses the observed
%distribution of packet interarrivals at the receiver as a signal,
%and incorporates stochastic predictions to forecast future rates that
%balance bit rate and delay. Rate forecasts allow Sprout to react on
%time scales faster than the round-trip time of the connection, which
%is unusual behavior for an end-to-end scheme. 
%Sprout's forecasts bear
%some similarity with the approach presented in a Hotnets position
%paper~\cite{WB11} (both deal with network uncertainty), but the
%problems tackled and the methods proposed in that paper are completely
%different.
\vspace{\baselineskip}
\noindent \textbf{Active queue management.} Active queue management schemes such as RED~\cite{Floyd93} and its
variants, BLUE~\cite{BLUE}, AVQ~\cite{AVQ}, etc., drop or mark packets
using local indications of upcoming congestion at a bottleneck queue,
with the idea that endpoints react to these signals before queues
grow significantly. Over the past several years, it has proven
difficult to automatically configure the parameters used in these
algorithms. To alleviate this shortcoming, CoDel~\cite{CoDel} changes
the signal of congestion from queue length to the delay experienced by
packets in a queue, with a view toward controlling that delay,
especially in networks with deep queues
(``bufferbloat''~\cite{bufferbloat}).

Our results show that Sprout largely holds its own with CoDel over
challenging wireless conditions without requiring any gateway
modifications. It is important to note that network paths in practice
have several places where queues may build up (in LTE infrastructure, in baseband modems, in IP-layer
queues, near the USB interface in tethering mode, etc.), so one may
need to deploy CoDel at all these locations, which could be
difficult. However, in networks where there is a lower degree of
isolation between queues than the cellular networks we study, CoDel
may be the right approach to controlling delay while providing good
throughput, but it is a ``one-size-fits-all'' method that assumes
that a single throughput-delay tradeoff is right for all traffic. 
%

% Sprout is end-to-end; it does not require any AQM at the bottleneck
% gateways, and might work well if all the sources in the network used
% it. That, however, is a difficult test to conduct in practice, so the
% immediately deployable ``use case'' is in cellular wireless networks
% where per-user queueing and (weighted and channel-dependent)
% round-robin scheduling is the norm. Per-user queues have traditionally
% been viewed as ``non-scalable'' in the folklore of Internet
% architecture, but cellular networks (for all their other shortcomings)
% demonstrate that such isolation is both practical and useful (e.g.,
% for accounting, billing, providing loose rate guarantees tied to
% subscriptions, etc.).  For wired networks with fair queueing gateways,
% a good congestion control solution is the packet-pair
% scheme~\cite{packetpair}, which relies on transmitting pairs (or
% bunches) of back-to-back packets to determine transmission rates. This
% method works well when the link rate is constant, but when link rates
% vary quickly, packet-pair methods are unlikely to work well.

%\paragraph{Adaptive applications.}
%In terms of application-layer adaptation, Alfalfa's new interface to
%video codec libraries enables finer-grained adaptation compared with
%other systems. Our use is an example of Application-Level Framing
%(ALF)~\cite{alf}, but unlike previous ALF-inspired applications, the
%adaptation is at a finer grain. Alfalfa does not just pick a suitable
%application data unit (ADU) from pre-determined choices, but instead
%constructs data frames dynamically using knowledge of previous
%receptions as well as the forecasted network conditions.  The ALF
%principle has been used in a variety of media delivery protocols and
%architectural proposals, including RTP~\cite{RTP}, MBone
%applications~\cite{Mbone}, the Congestion Manager API~\cite{cm},
%etc. Much of the previous work on ALF-inspired protocols has focused
%on handling packet losses~\cite{audio-perkins,cmvideo}, whereas in
%cellular networks the main problem is coping with rapid rate variations.
