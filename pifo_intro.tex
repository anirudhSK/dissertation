Packet scheduling is the task of picking the next packet to transmit on a
router's outgoing links. The research literature is replete with scheduling
algorithms~\cite{wfq, srr, lstf, drr, pFabric} tailored to different network
performance objectives (\eg fairness, low flow completion times, low tail
latencies, etc.).  However, today's fastest routers provide only a small menu
of scheduling algorithms: typically, a combination of Deficit Round Robin
(DRR)~\cite{drr}, strict priority scheduling, and traffic shaping. A network
operator can change parameters (\eg weights in DRR, or rate limits in traffic
shaping) in these algorithms. But, an operator cannot change the core logic in
an existing algorithm, nor program a new one, without building new router
hardware.

 This chapter develops hardware primitives and software programming models for
{\em programmable packet scheduling} in high-speed routers. With a {\em
programmable} packet scheduler, network operators could customize scheduling
algorithms to application requirements, \eg minimizing flow completion
times~\cite{pFabric} using Shortest Remaining Processing Time (SRPT)~\cite{srpt},
allocating bandwidth flexibly across flows or tenants~\cite{eyeq, faircloud}
using Weighted Fair Queueing~\cite{wfq} or minimizing tail packet delays using
Least Slack Time First~\cite{lstf}.  With a programmable packet scheduler,
router vendors could implement scheduling algorithms as programs running on a
programmable router chip, making it easier to verify and modify these
algorithms compared to baking in the same algorithms into a chip as rigid
hardware.

We begin this chapter by deconstructing packet scheduling
(\S\ref{s:deconstruct}). First, we observe that all scheduling algorithms make
two basic decisions: in {\em what order} packets should be scheduled and {\em
when} they should be scheduled, corresponding to work-conserving and
non-work-conserving algorithms respectively.  Second, we observe that for many
scheduling algorithms, these two decisions can be made when a packet is
enqueued. These observations suggest a natural hardware primitive for packet
scheduling: a {\em push-in first-out queue (PIFO)}~\cite{pifo}. A PIFO is a
priority queue that allows elements to be pushed into an arbitrary position
based on an element's {\em rank} (the scheduling order or time),\footnote{When
the rank denotes the scheduling time, the PIFO implements a calendar queue; we
distinguish between PIFOs and priority queues for this reason.} but always
dequeues elements from the head.

We then develop a PIFO-based programming model for scheduling (\S\ref{s:pifo})
with two key ideas. First, we allow an operator programming the scheduler to
set a packet's rank in a PIFO by supplying a small program for computing packet
ranks (\S\ref{ss:wfq}).  Coupling this program with a single PIFO allows the
operator to program any scheduling algorithm where the relative scheduling
order of buffered packets does not change with future packet arrivals. Second,
operators can compose PIFOs together in a tree to program several hierarchical
scheduling algorithms that violate this relative ordering property
(\S\ref{ss:hpfq} and \S\ref{ss:hshaping}).

We find that a PIFO-based scheduler lets us program many scheduling algorithms
(\S\ref{s:expressive}), \eg Weighted Fair Queueing~\cite{wfq}, Token Bucket
Filtering~\cite{tbf}, Hierarchical Packet Fair Queueing~\cite{hpfq},
Least-Slack Time-First~\cite{lstf}, the Rate-Controlled Service
Disciplines~\cite{rcsd}, and fine-grained priority scheduling (\eg Shortest Job
First). Until now, any high speed implementations of these algorithms---if they
exist at all---have been baked into the router's hardware. We also describe the
limits of the PIFO abstraction (\S\ref{pifo_ss:limitations}) by presenting
examples of scheduling algorithms that cannot be programmed using PIFOs.

To evaluate the hardware feasibility of PIFOs, we implemented a hardware design
for PIFOs in Verilog (\S\ref{s:design}). We synthesized this design to an
industry standard 16-nm standard-cell library (\S\ref{s:hardware}). The main
operation in our design is sorting an array of PIFO elements at the router's
line rate to determine the next packet to be dequeued. To implement this sort,
traditionally considered hard~\cite{sfq, drr}, we exploit two observations.
One, most scheduling algorithms schedule across flows, with packet ranks
increasingly monotonically within each flow.  Hence, we only need to sort the
head (earliest) packets of all flows to dequeue from a PIFO.  Two, transistor
scaling now makes it feasible to sort a surprisingly large number of packets (in
this case, the head packets in a PIFO) at high speed.

We evaluate our hardware design (\S\ref{s:hardware}) and find that it is
feasible to build a programmable scheduler, which
\begin{CompactItemize}
  \item supports 5-level hierarchical scheduling, where the scheduling
    algorithms at each level are programmable;
  \item runs at a clock frequency of 1 GHz---sufficient for a single-pipeline
    64-port shared-memory router with a line rate of 10 Gbit/s on each port;
  \item uses only 4\% additional chip area compared to a
    shared-memory router that supports only a small menu of scheduling
    algorithms; and
  \item has the same buffer size as a typical shared-memory router
    in a datacenter (\textasciitilde 60K packets, \textasciitilde 1K flows)~\cite{trident2}.
\end{CompactItemize}

 C++ code for a hardware reference model of our programmable scheduler and
Verilog code for our hardware design are available at
\url{http://web.mit.edu/pifo/}.
