\section{Packet transactions}
\label{s:transactions}

\begin{figure}[!t]
\begin{subfigure}{0.6\textwidth}
\begin{small}
\begin{lstlisting}[style=customc]
#define NUM_FLOWLETS    8000
#define THRESH          5
#define NUM_HOPS        10

struct Packet {
  int sport;
  int dport;
  int new_hop;
  int arrival;
  int next_hop;
  int id; // array index
};

int last_time [NUM_FLOWLETS] = {0};
int saved_hop [NUM_FLOWLETS] = {0};

void flowlet(struct Packet pkt) {
  pkt.new_hop = hash3(pkt.sport,
                      pkt.dport,
                      pkt.arrival)
                % NUM_HOPS;

  pkt.id  = hash2(pkt.sport,
                  pkt.dport)
            % NUM_FLOWLETS;

  if (pkt.arrival - last_time[pkt.id] @\label{line:ifStart}@
      > THRESH)
  { saved_hop[pkt.id] = pkt.new_hop; } @\label{line:ifEnd}@

  last_time[pkt.id] = pkt.arrival;
  pkt.next_hop = saved_hop[pkt.id];
}
\end{lstlisting}
\end{small}
\caption{Flowlet switching written in \pktlanguage}
\label{fig:flowlet_code}
\end{subfigure}
\begin{subfigure}{0.4\textwidth}
\includegraphics[width=\columnwidth]{domino_pipe.pdf}
\caption{6-stage \absmachine pipeline for flowlet
switching.  Control flows from top to bottom. Stateful atoms are in grey.}
\label{fig:flowlet_pipeline}
\end{subfigure}
\caption{Programming flowlet switching in \pktlanguage}
\vspace{-0.35in}
\end{figure}

A programmer programs a data-plane algorithm by writing it as a packet
transaction in \pktlanguage (Figure~\ref{fig:flowlet_code}).  The \pktlanguage
compiler then compiles this transaction to an atom pipeline for a \absmachine
machine (Figure~\ref{fig:flowlet_pipeline}). We first describe packet
transactions in greater detail by walking through the example shown in
Figure~\ref{fig:flowlet_code} (\S\ref{ss:flowlet}). Next, we discuss language
constraints in \pktlanguage (\S\ref{ss:constraints}) informed by the domain of
high-speed routers.  We then discuss how the execution of packet transactions
is triggered (\S\ref{ss:guards}) and how we handle multiple transactions
(\S\ref{ss:multiple}).

\subsection{\pktlanguage by example}
\label{ss:flowlet}

We use flowlet switching~\cite{flowlets} as our running example. Flowlet
switching is a load-balancing algorithm that sends bursts of packets, called
flowlets, from a TCP flow on a randomly chosen next hop. These bursts are
chosen such that two consecutive bursts are separated by a large time interval
of quiescence between them.  This ensures that packets do not arrive out of
order at a TCP receiver---even if packets in different bursts are reordered
because they experience different delays along different load-balanced paths.
For ease of exposition, we use only the source and destination ports in the
hash function that randomly computes the next hop for flowlet switching.

Figure~\ref{fig:flowlet_code} shows flowlet switching in \pktlanguage and
demonstrates its core language constructs. All packet processing happens in the
context of a packet transaction (the function \texttt{flowlet} starting at line
17). The function's argument type {\tt Packet} declares the fields in a packet
(lines 5--12)\footnote{A field is either a packet header, \eg source port ({\tt
sport}) and destination port ({\tt dport}), or packet metadata ({\tt id}).}
that can be referenced by the function body (lines 18--32).  The function body
can also modify persistent router state using global variables (\eg
\texttt{last\_time} and \texttt{saved\_hop} on lines 14 and 15, respectively).
The function body may use \textit{intrinsics} such as \texttt{hash2} on line 23
to directly access hardware accelerators on the router such as hash generators.
The \pktlanguage compiler uses an intrinsic's signature to analyze read/write
dependencies (\S\ref{ss:pipelining}), but otherwise considers an intrinsic a
blackbox. The compiler directly passes intrinsics to the underlying router chip
for execution, similar to intrinsics in GCC~\cite{intrinsics}.
% TODO: Define metadata somewhere?

\Para{Packet transaction semantics.}
Semantically, the programmer views the router as invoking the packet
transaction serially from start to finish on each packet in the order in which
packets arrive, with no concurrent packet processing.  Put differently, the
packet transaction modifies the passed-in packet argument and runs to
completion, before starting on the next packet.  These semantics allow the
programmer to program under the illusion that a single, extremely fast,
processor is serially executing the packet processing code for all packets. The
programmer does not have to worry about parallelizing the code within and
across pipeline stages to run on a high-speed router pipeline.

\subsection{The \pktlanguage language}
\label{ss:constraints}
\pktlanguage's syntax (Figure~\ref{fig:grammar}) is similar to an imperative
language, but with several constraints (Table~\ref{tab:restrict}).  These
constraints are required so that the compiler can provide deterministic
performance guarantees for \pktlanguage programs.  Memory allocation, unbounded
iteration counts, and unstructured control flow cause variable performance,
which may prevent an algorithm from achieving the router's line rate. Hence,
\pktlanguage forbids all three.

Additionally, within a \pktlanguage transaction, each array can only be
accessed using a single packet field. Repeated accesses to the same array are
allowed only if that packet field is unmodified between accesses. For example,
all read and write accesses to \texttt{last\_time} use the index
\texttt{pkt.id}. \texttt{pkt.id} is not modified during the course of a single
transaction execution (single packet); it only changes between executions
(packets).  This restriction on arrays mirrors restrictions on the stateful
memories attached to atoms (\S\ref{s:atomConstraints}), which are single ported
and hence only allow a single memory address to be read/written every clock
cycle.

\begin{table}
  \begin{tabular}{p{0.95\columnwidth}}
   No iteration (while, for, do-while).\\
   No unstructured control flow (goto, break, continue).\\
   No heap, dynamic memory allocation, or pointers.\\
   At most one location in each array is accessed by a single execution of a transaction. \\
   No access to unparsed portions of the packet (payload).\\
  \end{tabular}
  \caption{Restrictions in \pktlanguage}
  \label{tab:restrict}
\end{table}

%TODO: Read this grammar figure in the pdf and check correctness
\begin{figure}[!t]
\newcommand{\sep}{~|~}
\begin{small}
\begin{eqnarray*}
l \in \text{literals} \quad v \in \text{variables} \quad &bop& \in \text{binary ops} \quad
uop \in \text{unary ops} \\
%
e \in \text{expressions} &::=& e.f \sep l \sep v \sep e~bop~e \sep uop~e \sep e[d.f] \sep \\
                   & &   f(e_1, e_2, \ldots) \\
%
s \in \text{statements} &::=& e = e \sep {\tt if}~(e)~\{ s \}~{\tt else}~ \{ s \} \sep s~;~s \\
%
t \in \text{packet txns} &::=& name(v) \{ s \} \\
%
%vlist \in varList &::=& var;vlist \sep vlist \\
%
d \in \text{packet decls} &::=& \{ v_1, v_2, \ldots \} \\
%
sv \in \text{state var inits} &::=& v = e \sep sv~;~sv \\
%
p \in \text{\pktlanguage programs} &::=& \{ d ; sv; t \}
\end{eqnarray*}
\end{small}
\caption{\pktlanguage grammar. Types (void, struct, int, and Packet) are elided for simplicity.}
\label{fig:grammar}
\end{figure}

\subsection{Triggering the execution of packet transactions}
\label{ss:guards}
Packet transactions specify \textit{how} to process packet headers and state.
To specify {\em when} to execute packet transactions, programmers use {\em
guards}: predicates on packet fields that trigger a transaction if a packet
matches the guard. For example, the guard {\tt (pkt.tcp\_dst\_port == 80)}
could trigger heavy-hitter detection~\cite{opensketch} on packets with TCP
destination port 80.

Guards can be realized using the match unit in a match-action table, with the
actions being the atoms compiled from a packet transaction. Guards can take
various forms, \eg exact, ternary, longest-prefix, and range-based matches,
depending on the matches supported by the match units in the match-action
pipeline. In this chapter, we focus only on compiling packet transactions, \ie
generating the actions for the match-action tables. We assume that the
programmer specifies a reasonable guard that can be supported using the match
unit.
%TODO: Maybe talk above about compiling to guards etc.

\subsection{Handling multiple transactions}
\label{ss:multiple}
So far, we have discussed a single packet transaction corresponding to a single
data-plane algorithm. In practice, a router would run multiple data-plane
algorithms, each processing its own subset of packets. To address this, we
envision a policy language that specifies pairs of guards and transactions.
Realizing a policy is straightforward when all guards are disjoint: each
guard-transaction pair can be compiled independently. When guards overlap,
multiple transactions need to execute on the same subset of packets, requiring
a mechanism to compose transactions.

One composition semantics is to run the two transactions one after another
sequentially in a user-specified order. This can be achieved by concatenating
the two transaction bodies to create a larger transaction.  We leave a detailed
exploration of alternative composition semantics to future work.  We focus only
on compiling a single packet transaction here.
