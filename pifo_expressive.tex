\section{The expressiveness of PIFOs}
\label{s:expressive}

In addition to the three examples from \S\ref{s:pifo}, we now provide several
more examples of scheduling algorithms that can be programmed using our
programming model (\S\ref{ss:lstf} through \S\ref{ss:other}) and also describe
the limitations of our programming model (\S\ref{pifo_ss:limitations}).

\subsection{Least Slack-Time First}
\label{ss:lstf}

Least Slack-Time First (LSTF)~\cite{lstf,ups} schedules packets at each router
in increasing order of packet slacks, \ie the time remaining until each
packet's deadline.  Packet slacks are initialized at an end host or edge router
and are decremented by the wait time at each router's queue. We can program
LSTF using a simple scheduling transaction:
\begin{lstlisting}[style=customc]
  p.rank  = p.slack + p.arrival_time
\end{lstlisting}

The addition of the packet's arrival time to the slack already carried in the
packet ensures that packets are dequeued in order of what their slack {\em
would be} at the time of dequeue, not what their slack time {\em is} at the
time of enqueue. Then, after packets are dequeued, we subtract the time at
which the packet is dequeued from the packet's slack, which has the effect of
decrementing the slack by the wait time at the router's queue. This subtraction
can be achieved by programming the egress pipeline to decrement one header
field by another using the techniques from Chapter~\ref{chap:domino}.

\subsection{Stop-and-Go Queueing}
\label{ss:stopngo}

\begin{figure}[h]
  \begin{lstlisting}[style=customc]
  if (now >= frame_end_time):
    frame_begin_time = frame_end_time
    frame_end_time   = frame_begin_time + T
  p.rank = frame_end_time
  \end{lstlisting}
\caption{Shaping transaction for Stop-and-Go Queueing.}
\label{fig:stopngo}
\end{figure}

Stop-and-Go Queueing~\cite{stopngo} is a non-work-conserving algorithm that
provides bounded delays to packets using a framing strategy. Time is divided
into non-overlapping frames of equal length \texttt{T}, where every packet
arriving within a frame is transmitted at the end of the frame, smoothing out
any burstiness in traffic patterns induced by previous hops.

The shaping transaction in Figure~\ref{fig:stopngo} can be used to program
Stop-and-Go Queueing. {\tt frame\_begin\_time} and {\tt frame\_end\_time} are
two state variables that track the beginning and end of the current frame in
wall-clock time.  When a packet is enqueued, its departure time is set to the
end of the current frame.  Multiple packets with the same departure time are
sent out in first-in first-out order, as guaranteed by a PIFO's semantics for
breaking ties with equal ranks (\S\ref{s:pifo}).

\subsection{Minimum rate guarantees}
\label{ss:min_rate}

A common scheduling policy on many routers today is providing a minimum rate
guarantee to a flow, provided the sum of such guarantees does not exceed the
link capacity. A minimum rate guarantee can be programmed using PIFOs with a
two-level PIFO tree, where the root of the tree implements strict priority
scheduling across flows. Flows below their minimum rate are scheduled
preferentially to flows above their minimum rate. Then, at the next level of
the tree, the PIFOs implement the FIFO discipline for each flow.

When a packet is enqueued, we execute a scheduling transaction corresponding to
the FIFO discipline at its leaf node, setting its rank to the wall-clock time
on arrival. At the root, a PIFO reference (the packet's flow identifier) is
pushed into the root PIFO using a rank that reflects whether the flow is above
or below its rate limit after the arrival of the current packet. To determine
this, we run the scheduling transaction in Figure~\ref{fig:min_rate} that uses
a token bucket (the state variable {\tt tb}) that can be filled up until {\tt
BURST\_SIZE} to decide if the arriving packet puts the flow above or below {\tt
min\_rate}.

\begin{figure}
  \begin{lstlisting}[style=customc]
  # Replenish tokens
  tb = tb + min_rate * (now - last_time)
  if (tb > BURST_SIZE):
    tb = BURST_SIZE

  # Check if we have enough tokens
  if (tb > p.size):
    p.over_min = 0 # under min. rate
    tb = tb - p.size
  else:
    p.over_min = 1 # over min. rate

  last_time = now
  p.rank = p.over_min
  \end{lstlisting}
\caption{Scheduling transaction for minimum rate guarantees.}
\label{fig:min_rate}
\end{figure}

Note that a single PIFO node with the scheduling transaction in
Figure~\ref{fig:min_rate} is not sufficient. It causes packet reordering within
a flow: an arriving packet can cause a flow to move from a lower to a higher
priority and, in the process, leave before low priority packets from the same
flow that arrived earlier. The two-level tree solves this problem by attaching
priorities to transmission opportunities for a specific flow, instead of
attaching priorities to specific packets. Now, if an arriving packet causes a
flow to move from low to high priority, the next packet scheduled from this
flow is the earliest packet of that flow chosen in FIFO order, not the packet
that just arrived.

\subsection{Other examples}
\label{ss:other}

We now briefly describe several more scheduling algorithms that can be
programmed using PIFOs.

\begin{CompactEnumerate}
\item \textbf{Fine-grained priority scheduling.} Many algorithms schedule the
packet with the lowest value of a field initialized by the end host. These
algorithms can be programmed by setting the packet's rank to the appropriate
field. Examples of such algorithms and the fields they use are: strict priority
scheduling (IP TOS field), Shortest Flow First (flow size), Shortest Remaining
Processing Time (remaining flow size), Least Attained Service (bytes received
for a flow), and Earliest Deadline First (time until a deadline).
\item \textbf{Service-Curve Earliest Deadline First
    (SC-EDF)~\cite{sced}} schedules packets in increasing order of a
  deadline computed from a flow's service curve, which specifies the
  service a flow should receive over any given time interval. We can
  program SC-EDF using a scheduling transaction that sets a packet's
  rank to the deadline computed by the SC-EDF algorithm.
\item \textbf{Rate-Controlled Service Disciplines (RCSD)~\cite{rcsd}}
  such as Jitter-EDD~\cite{jitteredd} and Hierarchical Round
  Robin~\cite{hrr} are a class of non-work-conserving schedulers. This
  class of schedulers can be implemented using a combination of a rate
  regulator to shape traffic and a packet scheduler to schedule the
  shaped traffic. An RCSD algorithm can be programmed using PIFOs by
  programming the rate regulator using a shaping transaction and the
  packet scheduler using a scheduling transaction.
  
\item \textbf{Incremental deployment of programmable scheduling.}
  Operators may wish to use programmable scheduling only for a
  subset of their traffic. This can be programmed as a hierarchical
  scheduling algorithm, with one FIFO class dedicated to legacy
  traffic and another to experimental traffic. Within the experimental
  class, an operator could program any scheduling tree, \eg WFQ, LSTF, HPFQ.
\end{CompactEnumerate}

\subsection{Limitations of our programming model}
\label{pifo_ss:limitations}

\Para{Changing the scheduling order of all packets of a flow.} 
A tree of PIFOs can enable some algorithms (\eg HPFQ in \S\ref{ss:hpfq}) where
the relative scheduling order of buffered packets changes in response to new
packet arrivals. However, it does not permit arbitrary changes to the
scheduling order of buffered packets. In particular, it does not support
changing the scheduling order for {\em all} buffered packets of a flow when a
new packet from that flow arrives.

An example of an algorithm that needs this capability is
pFabric~\cite{pFabric}. pFabric introduces ``starvation prevention'' to
schedule the packets of the flow with the shortest remaining size in FIFO
order, to prevent packet reordering within a flow. To see why this is beyond
the capabilities of PIFOs, consider the sequence of arrivals below, where pi(j)
represents a packet from flow i with remaining size j. The remaining size
is the number of unacknowledged bytes in a flow.
\begin{CompactEnumerate}
\item Enqueue p0(7).
\item Enqueue p1(9), p1(8).
\item The scheduling order is: p0(7), p1(9), p1(8).
\item Enqueue p1(6).
\item The new order is: p1(9), p1(8), p1(6), p0(7).
\end{CompactEnumerate}

Specifying these semantics are beyond the capabilities of PIFOs.\footnote{This
is ironic because we started this project to program pFabric on a high-speed
router, but have ended up being able to do almost everything but that!} For
instance, adding a level of hierarchy with a PIFO tree does not help. Suppose
we programmed a PIFO tree implementing FIFO at the leaves and picking among
flows at the root based on the remaining flow size. This would result in the
scheduling order p1(9), p0(7), p1(8), p1(6), after enqueueing p1(6). The problem
is that there is no way to change the scheduling order for {\em multiple}
references to flow 1 in the root PIFO by enqueueing only one reference to flow
1.
%% Mohammad: The above might still be too confusing; I know we don't
%% have much room, but a figure would really help with this. Let's run
%% this specifically by Jeff and Hari and see if it is comprehensible.
%% TODO: Yes, it's still a bit confusing. We could draw a figure.
%% Maybe we could remove CBQ to make room for this figure?

A single PIFO {\em can}, however, implement pFabric without starvation
prevention, which is identical to the Shortest Remaining Processing Time (SRPT)
discipline (\S\ref{ss:other}).  It can also implement the Shortest Flow First
(SFF) discipline (\S\ref{ss:other}), which performs almost as well as
pFabric~\cite{pFabric}.

\Para{Traffic shaping across multiple nodes in a scheduling tree.}  Our
programming model attaches a single shaping and scheduling transaction to a
tree node. This lets us enforce rate limits on a single node, but not across
multiple nodes.

As an example, PIFOs cannot express the following policy: WFQ on a set of flows
A, B, and C, with the additional constraint that the aggregate throughput of A
and B combined does not exceed 10 Mbit/s. One work around is to implement this
as HPFQ across two classes C1 and C2, with C1 containing A and B, and C2
containing C alone. Then, we can enforce the rate limit of 10 Mbit/s on C1 as
in Figure~\ref{fig:hshaping}. However, this is not equivalent to our desired
policy. More generally, our programming model for programmable scheduling
establishes a one-to-one relationship between the scheduling and shaping
transactions, which excludes some scheduling algorithms.

\Para{Output rate limiting.} The PIFO abstraction enforces rate limits using a
shaping transaction, which determines a packet or PIFO reference's scheduling
time {\em before} it is enqueued into a PIFO.  The shaping transaction permits
rate limiting on the {\em input} side, \ie before elements are enqueued. An
alternate form of rate limiting is on the {\em output}, \ie by limiting the
rate at which elements are scheduled.

To illustrate the difference between input and output rate limiting, consider a
scheduling algorithm with two priority queues, \texttt{LO} and \texttt{HI},
where \texttt{LO} is to be rate limited to 10 Mbit/s. To program this using
input rate limiting, we would use a shaping transaction to impose a 10 Mbit/s
rate limit on \texttt{LO} and a scheduling transaction to implement strict
priority scheduling between \texttt{LO} and \texttt{HI}. Now, assume packets
from \texttt{HI} starve \texttt{LO} for a long period of time. During this
time, packets from \texttt{LO}, after leaving the shaping PIFO, accumulate in
the PIFO shared with \texttt{HI}. Now, if there are suddenly no more
\texttt{HI} packets, all packets from \texttt{LO} are transmitted at the
router's line rate, and are no longer rate limited to 10 Mbit/s until all
instances of \texttt{LO} are drained out of the PIFO shared with \texttt{HI}.
Input rate limiting still provides long-term rate guarantees, while output rate
limiting provides short-term guarantees as well.
%TODO: Consider adding sentence on how output rate limiting does rate limiting
%in a more fine-grained manner.
