\chapter*{Acknowledgments}
\addcontentsline{toc}{chapter}{Acknowledgements}%

In one sense, this thesis work has occupied a third of my
life---counting from when I became a graduate student in 2003, and
keeping the clock running for my sojourns away as I figured out what
to work on when I came back. Arriving at this point would have been
impossible without the help of a large number of people.

My advisor, Hari Balakrishnan, has been a tireless counselor, mentor,
booster, and co-author. I'm lucky to have had the benefit of his
wisdom, his nose for interesting problems, and his tendency to get
hooked and excited and to champion any idea that comes from his students.

I'm especially grateful to my frequent co-author and labmate, Anirudh
Sivaraman, who taught me how to be a productive collaborator, and
without whom most of this work could not have been realized.

Thank you to my thesis committee, of Hari Balakrishnan, Dina Katabi,
Scott Shenker, and Leslie Kaelbling, who have reached out across
disciplines to support and guide this work.

Just before I headed off to MIT as an undergraduate in 1999, I was
advised in strong terms to look up Gerald Jay Sussman when I got
there. I didn't know what I was in for! I have worked with Gerry since
I was a freshman and cannot easily express how much I have gained from
his mentorship, his knowledge, and his sense of taste.

My informal ``backup advisors'' on the ninth floor---Nickolai
Zeldovich and M.~Frans Kaashoek---have been patient through
innumerable practice talks and late-night idea-bouncing
sessions. Outside the ninth floor, I have been fortunate to have the
support and counsel of Victor Bahl, Hal Abelson, Mike Walfish,
Jonathan Zittrain, and Anne Hunter.

I owe much to my time in the Wall Street Journal's late Boston
bureau. My editors Gary Putka and Dan Golden, and my colleague Charles
Forelle, taught me how to develop a nose for investigations and how to
tell a story. Gary's ``dun-colored warren of cubicles'' represented a
journalistic Camelot and was as intellectually challenging as any lab.

Many others helped to develop the ideas in this work, especially
Pratiksha Thaker, Chris Amato, Andrew McGregor, Tim Shepard, John
Wroclawski, Bill McCloskey, Josh Mandel, Carrie Niziolek, Damon
Wischik, Hariharan Rahul, Garrett Wollman, Juliusz Chroboczek, Chris
Lesniewski, Vikash Mansinghka, Eric Jonas, Marissa Cheng, and Allie
Brownell.

To my labmates and ninth floor coinhabitants---Katrina LaCurts, Raluca
Ada Popa, Shuo Deng, Lenin Ravindranath, Ravi Netravali, Peter
Iannucci, Tiffany Chen, Amy Ousterhout, Jonathan Perry, Neha Narula,
Austin Clements, Emily Stark, Cody Cutler, Eugene Wu, Dan Ports, Silas
Boyd-Wickizer, and Sheila Marian---thank you for your companionship
and our time together.

Immeasurable gratitude is due to my mom, Joan Winstein, and my sister Allison,
for supporting and putting up with me and giving me a sense of how
things ought to be. To all our sadness, my father passed away just
after making it to Allison's master's recital and just after my return
to graduate school in 2011. When I was young, my dad would sometimes
bring home his graduate students in physics for dinner, some of whom
became surrogate older siblings for a time. Much later, for his
retirement, they all wrote about his incredible attention to detail
and his sense of right and wrong in constructing tests of Mother
Nature---things I have always admired and hope have partly
rubbed off on me.

When he died, the university was kind enough to let me ghostwrite his
obituary, and so my dad will always be remembered the way I saw
him. I'll add one more thing, by dedicating this dissertation to his
memory.
