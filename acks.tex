\chapter*{Acknowledgments}
\addcontentsline{toc}{chapter}{Acknowledgments}%

I was fortunate to work with two outstanding advisors, Hari Balakrishnan and
Mohammad Alizadeh. Each contributed a unique perspective to my research.  Hari
was the master of the highest-order bit. He taught me to focus on the most
important thing on hand, whether it was an elusive sentence required to finish
a paragraph, the title of a key slide during a presentation, or a single
sentence summarizing an entire paper. Hari took me on as his student when my
options were limited; this dissertation is my way of repaying that debt.
Mohammad taught me to persist until a concept was clear to the point of being
obvious. He made me appreciate the peace of mind that comes with mathematical
rigor and showed me how to work with hardware engineers.  Most importantly, he
taught me how to listen and incorporate opposing viewpoints while moving my own
research forward.

Keith Winstein has been a friend and mentor through graduate school. He taught
me to love clean code, corrected my writing, showed me how to give an engaging
talk, and convinced me that it was always possible to get a computer to do your
bidding. In April 2013, he wrote a blog post wondering why software-defined
networking did not include the data plane. This dissertation is a belated
answer to his question.

My thesis committee members, George Varghese and Nick McKeown, helped shape my
research taste. George taught me to go beyond the surface in interdisciplinary
work; he pushed me to discover crisp connections to and differences from adjacent
disciplines. Nick recommended that I spend some time interning at Barefoot
Networks, which significantly improved my dissertation, and showed me how to
combine intellectual rigour with impact.

As part of my dissertation work, I spent a year interning at Barefoot Networks.
It is not often that a fledgling startup gives an intern free rein to pursue
open-ended research. My manager, Changhoon Kim, gave me almost endless rope to
go after ideas that I thought were interesting. Mihai Budiu taught me how to
engineer a compiler. Anurag Agrawal patiently explained a router's scheduler to
me. In addition, I benefited from conversations with many other engineers and
interns at Barefoot Networks, including Antonin, KRam, Ravindra, CK, and Naga.

This dissertation represents joint work with many collaborators: Alvin Cheung,
Mihai Budiu, Changhoon Kim, Mohammad Alizadeh, Hari Balakrishnan, George
Varghese, Nick McKeown, Steve Licking, Suvinay Subramanian, Sharad Chole,
Shang-Tse Chuang, Anurag Agrawal, Tom Edsall, Sachin Katti, Srinivas Narayana,
Vikram Nathan, Prateesh Goyal, Venkat Arun, and Vimalkumar Jeyakumar. The
broader P4 community provided an ideal setting for my work, both to get
feedback and as a vehicle for translating some of these ideas into practice. 

My grad school colleagues past and present, Ravi, Amy, Vikram, NG, Tiffany,
Shuo, Katrina, Lenin, Jonathan, Peter, Hongzi, Pratiksha, Somak, Ameesh, Alvin,
and Eugene provided a great reason to come to work everyday, whether it was to
get ice cream at Tosci's or play an afternoon game of ping-pong. Ravi has been
a great sounding board, tempering my naive optimism with a healthy dose of
reality. Suvinay and NG have been close friends through graduate school, and I
am glad to have had them around. My friends outside grad school, Raghav,
Aakash, Akhil, Raghunandan, Siddharth, and Praneeth, have provided me with
much-needed breaks from work with the occasional catch ups.

Sheila Marian has been a phenomenal admin assistant, assisting with paper work
when I was away from MIT, booking travel for me, and ordering cakes for thesis
defenses. Janet Fischer at the EECS graduate office has patiently extended
every Ph.D.  deadline. My academic advisor, Piotr Indyk, has looked out for me
since my first year at MIT, especially when I was switching advisors.  Sylvia
Hiestand at the MIT International Students Office got all of my paper work in
order during my year away from MIT.

My parents, Rama and Sivaraman, have put up with my churlish ways during my
Ph.D.; I am grateful to them for their patience and for not asking me when I
would graduate. My sister, Vibhaa, continues to cheer me up when I am down.  My
grandmother's utter lack of interest in my research is a much needed breath of
fresh air. My in-laws have been a steady source of support from afar during my
job search. My wife, Tejaswi, has gone through both the highs and lows of my
Ph.D. along with me. I am grateful for her ability to listen carefully and for
her honest advice during stressful times. 

My late grandfather, Dr. V. Ramamurti, was the reason I embarked on a Ph.D.
Throughout high school, he spent an inordinate amount of time patiently
teaching me mathematics and physics and indulging my pesky questions. As a
faculty member, who also consulted for industry, he taught me not to stray too
far from reality. I dedicate this thesis to him.
