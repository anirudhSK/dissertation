\chapter*{Acknowledgments}
\addcontentsline{toc}{chapter}{Acknowledgments}%

As a Ph.D. student, I was fortunate to work with two outstanding advisors, Hari
Balakrishnan and Mohammad Alizadeh. To me, Hari was the master of the
highest-order bit. He reminded me at all times to focus on that which was most
important, whether it was the elusive sentence required to complete a
paragraph, the title of a key slide during a presentation, nudging me in the
right direction when I needed life advice, or the simplest way to explain a
particular concept. Hari took a chance on me in the spring of 2012, and agreed
to take me on as a student solely based on my stint as his TA, and this
dissertation is my way of repaying that debt. I was fortunate to start working
with Mohammad in 2014 right after he had accepted an offer from MIT. Mohammad
taught me the value of digging deep until a concept was crystal clear and
mathematically precise. He taught me to appreciate the beauty and headiness
that comes with mathematical rigor and showed me, by example, how to work with
router hardware engineers.

Keith Winstein has been a friend, philosopher, and guide through graduate
school.  In the two years we spent working together almost on a daily basis, he
taught me to love clean designs, corrected my writing, improved my ability to
give talks, and convinced me that, with sufficient ingenuity, it was always
possible to get the computer to do your bidding. In a blog post following NSDI
in April 2013, he asked why software-defined networking did not address the
data plane. This dissertation, in some ways, is a four-year delayed and
somewhat lengthy answer to those questions.

My thesis committee members, George Varghese, and Nick McKeown were
instrumental in shaping my research taste. George taught me what doing great
interdisciplinary work was about; he pushed me to go deeper till I discovered
connections and differences from adjacent disciplines. George also entertained
by harebrained ideas until I fleshed them out into something reasonable. Nick
recommended that I spend some time interning at Barefoot Networks pursuing the
questions in this dissertation. Through his own example, he showed me it was
possible to combine intellectual rigour with industrial relevance.

I spent close to a year interning at Barefoot Networks carrying out some of the
work in this dissertation. It is not often that a fledgling startup allows an
intern free reign for a year to pursue open-ended research. My manager,
Changhoon Kim, gave me almost endless rope to go after the ideas I thought were
interesting. Mihai Budiu taught me how to engineer a compiler, and Anurag
Agrawal patiently explained every bit of the router's scheduler. In addition, I
benefited from conversations with many other engineers and interns at Barefoot
Networks, including Antonin, KRam, Ravindra, CK, and Naga. 

This dissertation represents joint work with a number of collaborators: Alvin
Cheung, Mihai Budiu, Changhoon Kim, Mohammad Alizadeh, Hari Balakrishnan,
George Varghese, Nick McKeown, Steve Licking, Suvinay Subramanian, Mohammad
Alizadeh, Sharad Chole, Shang-Tse Chuang, Anurag Agrawal, Tom Edsall, Sachin
Katti, Srinivas Narayana, Vikram Nathan, Prateesh Goyal, Venkat Arun, and
Vimalkumar Jeyakumar. The broader P4 community provided an ideal setting to
carry out this work, both to get feedback from its community members and as a
vehicle for translating some of these ideas into practice. 

My grad school colleagues past and present, Ravi, Amy, Vikram, Tiffany, Shuo,
Katrina, Lenin, Jonathan, and Peter, provided a great reason to come to work
everyday, whether it was to get ice cream at Tosci's, commiserate about grad
school, or play an afternoon game of ping-pong.  Suvinay Subramanian has been a
steady friend through graduate school, and I am glad to have had him around. My
friends outside grad school, Raghav, Aakash, Akhil, and Praneeth have provided
me with much-needed breaks from work from time to time.

Sheila Marian has been a phenomenal admin assistant, assisting with paper work
when I was away from MIT for a year, booking travel for me, and ordering cakes
for thesis defenses. Janet Fischer has probably seen me at the EECS graduate
office more times than she can count and has patiently extended every one of my
Ph.D. deadlines. Piotr Indyk was my academic advisor for the last seven years,
and has looked out for me since my first year in graduate schools, especially
when I needed help changing advisors. Sylvia Hiestand at the MIT International
Students Office got all of my paper work in order when I needed to spend a year
away from MIT.

My parents, Rama and Sivaraman, have put up with my churlish ways during the
course of my Ph.D.; I am grateful to them for their patience and for not asking
me when I would graduate. My sister, Vibhaa, continues to cheer me up when I am
down.  My grandmother's complete lack of interest in my research is a much
needed breath of fresh air. My in-laws have been a steady source of support
from afar during my job search. My wife, Tejaswi, has gone through both the
highs and lows of my Ph.D. in a long-distance relationship for seven years; I
am grateful for her calming presence and honest advice during stressful times. 

My late grandfather, Dr. V. Ramamurti, has been an inspiration to me since I
was a kind, and was the reason I embarked on a Ph.D. in the first place.
Unfortunately, his untimely demise last year meant that he could not see me
complete my Ph.D. I dedicate this thesis to him.
