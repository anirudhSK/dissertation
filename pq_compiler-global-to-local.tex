\subsection{Network-wide to switch-level queries}
\label{sec:network-to-switch-local}

%%\input{network-to-local-algorithm}

The compiler should partition a network-wide query (\Sec{language}) into a
switch-level query to generate a switch-specific configuration. We achieve this
in two steps. First, we determine the switches that contribute tuples to the
query output stream. Then we determine how to partition aggregation functions
written over the entire network into switch-local aggregations.

\Para{Determining stream output locations.} It is straightforward to determine
the locations, \ie set of switches, that contribute tuples to the query output
stream. The {\ct filter}s are the only operators which restrict the switches
that contribute tuples. We evaluate the set of switches implied by the {\ct
  filter} predicate through basic syntactic checks of the form {\ct switch ==
  X}, and combining the implied sets for more complex predicates using set union
and intersection. The {\ct zip} operator has outputs that appear at the
intersection of the locations of its two inputs. The {\ct map} and {\ct groupby}
operators appear at the same locations as their inputs. Starting from the {\ct
  \pktlog}, we recursively evaluate the stream locations.

The stream output locations for the running example are shown in
\Fig{compiler-ast-manipulations}(b). The incoming {\ct \pktlog} appears at {\em
  all} network switches, but is restricted to just {\ct S1} and {\ct S2} by the
filter in the query (left branch). These locations are propagated to the root of
the AST through the {\ct zip} operator.

%% We transform a network-wide AST to a switch-local
%% AST by annotating the nodes in the AST with an attribute involving the {\ct
%%   locations} at which the query results may appear.
%% %%\Alg{global-analysis} computes two values for each node in the query AST.
%% The location is expressed as a set of switches.

%% It is straightforward to write down the locations where each node in the query
%% AST must be active, given a list of locations where the operands appear. We
%% start by assuming that the {\ct \pktlog} appears at all switches in the
%% network. For the {\ct map} and {\ct groupby} nodes, the resulting location is
%% the same as the operand; for the {\ct zip} operator, it is the intersection of
%% the locations of the operands.

%% For a {\ct filter}, the output location may be restricted by the predicate. Let
%% us suppose the network has switches {\ct S1, ..., S10}. From the {\ct filter}
%% predicate, we extract a set of switches that is implied by it. For example, a
%% predicate {\ct switch == S4} implies the location set {\ct \{S4\}}, whereas a
%% predicate {\ct switch == S1 || switch != S3} implies the locations {\ct \{S1,
%%   S2, S4, ..., S10\}}, \ie all the switches except {\ct S3}. The final location
%% of a {\ct filter} query is the intersection of the locations implied by its
%% predicate and the locations of its operand.

%% A {\ct filter} removes all switches not implied by the defined predicate (line
%% \ref{line:get-switches}): for example, {\ct filter(switch == 10)} removes all
%% switches but the switch with identifier 10. Unless the retained set only has one
%% switch, the {\ct filter} output retains the type of its operand (lines
%% \ref{line:filter-retains-start}-\ref{line:filter-retains-end}). A {\ct map}
%% retains its operands' information, as a {\ct map} neither affects the switches
%% nor type of packet stream processed (line \ref{line:map-return}).

%% A {\ct groupby} stream appears at the same locations as its operand. However, it
%% converts a multi-switch stream into a single-switch stream (SSS) if the {\ct
%%   switch} attribute appears in the list of fields by which the stream is
%% aggregated (line \ref{line:fold-switch}). Otherwise, it checks whether the
%% stream is aggregated by the unique packet identifier {\ct packetUid}, or
%% involves an associative function at the top of the query AST (lines
%% \ref{line:fold-packet} and \ref{line:fold-associative}). If none of these
%% conditions is true, the algorithm throws an error (line \ref{line:fold-error})
%% since the conditions for correct aggregation processing are unmet.

%% A {\ct zip} operator simply returns the type of its operands if the operand
%% types match (SSS/MSS) and SSS otherwise (lines
%% \ref{line:zip-type-match-start}-\ref{line:zip-type-match-end}). The resulting
%% stream appears at the intersection of the switches where the operand streams
%% appear.

\Para{Partition network-wide aggregations.} \TheSystem provides the abstraction
of a single network-wide input stream of tuples. However, this makes some
aggregations challenging to implement correctly over a {\em distributed} network
of switches. For example, consider computing an EWMA over packet latencies,
processing tuples in the order of packet arrivals {\em across} switches. This
aggregation cannot be partitioned cleanly over switches, because the
computation requires accounting for packet interleavings across multiple
switches.

We simplify our implementation by only allowing aggregations which can be
cleanly decomposed into independent switch operations, without explicit
coordination among switches.
%% and then re-combined. This prevents
%% users from writing aggregations that depend on processing {\em multiple packets
%%   across multiple switches} in order. We believe this is not a major restriction
%% since none of our example aggregations require this.
Concretely, we only allow aggregation functions which either (i) operate on one
switch, (ii) operate on one packet (identified by {\ct uid}), or (iii)
are associative.
%%
%% We check three conditions: The aggregation is either (i) by switch (and may
%% include other aggregation fields), (ii) per-packet, or (iii) an associative
%% function. 

We label each AST node with an additional attribute, the stream {\em type},
corresponding to whether its output stream consists of tuples from a single
switch or multiple switches. Stream type is distinct from stream location: for
example, a {\ct groupby} aggregating by {\ct switch} produces a stream which has
the type of a single-switch-stream, though its tuples may appear at multiple
switches.

Determining and propagating the stream type up an AST is straightforward. The
base {\ct \pktlog} is a multi-switch-stream. The {\ct filter} or {\ct zip}
operators produce a single-switch-stream if their output only appears at a
single switch. The {\ct groupby} produces a single-switch-stream if it
aggregates by {\ct switch.} In all other cases, the operators retain the stream
type of their operands. If all {\ct groupby}s either produce
single-switch-streams or aggregate information per-packet, they can be trivially
partitioned. If they violate both conditions, they must be declared associative
by the programmer, using the {\ct assoc} keyword in the definition of the
aggregation.

The stream types for our running example are shown in
\Fig{compiler-ast-manipulations}(c). The {\ct filter} produces output streams at
two switches, hence is still a multi-switch-stream. The {\ct groupby} aggregates
by {\ct switch} and hence is a single-switch-stream. The output of this process
is a set of independent switch-level ASTs corresponding to each switch location
that the AST root operator appears in, \ie {\ct S1, S2}.


%% \Para{Recognize aggregations that cannot be distributed over the network.} In
%% the same pass, we also recognize whether aggregation functions can be faithfully
%% distributed over multiple switches in the network. Specifically, while the
%% abstract model of query processing permits writing aggregation functions over
%% packets appearing at multiple switches, in general this may require out-of-band
%% communication between the switches.

%% We restrict such out-of-band communication between switches, and simplify our
%% query implementation by disallowing aggregations that depend on the order of
%% arrivals of {\em multiple packets} across {\em multiple switches}. We mandate
%% that all aggregation functions fall into one of three classes:
%% \begin{enumerate}
%% \item \label{req:per-switch} aggregation is over multiple packets but a result
%%   is desired for each switch;
%% \item \label{req:per-packet} aggregation is over multiple switches but a result
%%   is desired for each packet; or
%% \item \label{req:associative} the aggregation operation is associative and at
%%   the root of the query AST.
%% \end{enumerate}

%% We recognize whether aggregation functions adhere to these restrictions by
%% annotating each query AST node with a second attribute, namely the {\em type} of
%% the output stream: whether the result of the operator should be treated as a
%% stream seen at just one switch, or at multiple switches. Note that the stream
%% type is distinct from the stream location. A query performing a {\ct groupby}
%% using the {\ct switch} field may produce a stream appearing at all network
%% switches, but the output stream only appears on one switch as far as operators
%% consuming the stream are concerned.

%% We can recursively determine the type of query AST nodes to verify conditions
%% (\ref{req:per-switch}) and (\ref{req:per-packet}) above. We start with the fact
%% that the {\ct \pktlog} is a multi-switch-stream. It is only transformed to a
%% single-switch-stream on recursing through the query AST under two
%% circumstances: \begin{enumerate}
%% \item A {\ct groupby} operator aggregates the input stream by {\ct switch}; or
%% \item A {\ct filter} or {\ct zip} operator produces a result whose location only
%%   contains one switch.
%% \end{enumerate}
%% Given a type annotation for each node in the query AST, it is straightforward to
%% check conditions on aggregations involving only a single switch or a single
%% packet.

%% However, checking associativity (condition \ref{req:associative}) is difficult
%% in practice. We take the programmer's aid by requiring that such functions
%% prefix the keyword {\ct assoc} to the definition of the aggregation function.

%% The result of the transformations in this section is a network-wide query AST
%% with switch locations annotated on the root of the AST, from which we can easily
%% produce query ASTs for each switch. In \ngs{figure forthcoming}, we show the
%% annotated query AST for the running example.
