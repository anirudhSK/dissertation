\subsection{Network-wide to router-local queries}
\label{sec:network-to-router-local}

The compiler partitions a network-wide query written over all packets at all
queues in the network (\S\ref{sec:language}) into router-local queries, which
are then transformed into router-specific configurations. We achieve this in
two steps. First, we determine the {\em stream location}, \ie the set of
routers that contribute tuples to a stream, for the final output stream of
query. For instance, the output stream of a query that filters by router id $s$
has a stream location equal to the singleton set $s$. Second, we determine how
to partition queries with aggregation functions written over the entire network
into router-local queries.

\Para{Determining stream location for the final output stream.} To start with,
the stream location of {\ct \pktlog} is the set of all routers in the network.
The stream location of the output of a {\ct filter} is the set of routers
implied by the filter's predicate. We evaluate the set of routers contributing
tuples to the output of a {\ct filter} operation through basic syntactic checks
of the form {\ct router == X} on the {\ct filter} predicate.  We combine router
sets for boolean combinators ({\ct or} and {\ct and}) inside filter predicates
using set union and intersection respectively.  The stream location of the
output of a {\ct zip} operator is the intersection of the stream locations of
the two inputs.  Stream locations are unchanged by the {\ct map} and {\ct
groupby} operators.

The stream locations for the running example are shown in
\Fig{compiler-ast-manipulations}b. The stream location of {\ct \pktlog} is the
set of all network routers, but is restricted to just {\ct S1} and {\ct S2} by
the {\ct filter} in the query (left branch). This location is then propagated
to the root of the AST through the {\ct zip} operator in the query.

\Para{Partitioning network-wide aggregations.}
As described in \S\ref{sec:language}, we only permit aggregations that satisfy
one of three conditions: (1) they operate independently on each router, (2)
operate independently on each packet, or (3) are associative and commutative.
We describe below how we check the first condition, failing which we check the
last two conditions syntactically: either the {\ct groupby} aggregates by {\ct
uid} (condition 2) or contains programmer annotations {\ct assoc} and {\ct
comm} (condition 3).

%TODO: What does operate indep. mean?
%TODO: Need to explain what partitioned means ...
To check if an aggregation operates independently on each router, we label each
AST node with a boolean attribute, {\em router-partitioned}, corresponding to
whether the output stream has been partitioned by the router at which it
appears. Intuitively, if a stream is router-partitioned, we allow
packet-order-dependent aggregations over multiple packets of that stream;
otherwise, we do not.

Determining and propagating {\em router-partitioned} through an AST is
straightforward. The base {\ct \pktlog} is not router-partitioned. The {\ct
filter} and {\ct zip} operators produce a router-partitioned stream if their
output only appears at a single router. The {\ct groupby} produces a
router-partitioned stream if it aggregates by {\ct router.} In all other cases,
the operators retain the operands' router-partitioned attribute.

The router-partitioned attributes for our running example are shown in
\Fig{compiler-ast-manipulations}c. The {\ct filter} produces output streams at
two routers, hence is not router-partitioned. The {\ct groupby} aggregates by
{\ct router} and hence is router-partitioned.

%TODO: How does the paragraph below happen?
After the partitioning checks have succeeded, we are left with a set of
independent router-local ASTs corresponding to each router location that the
AST root operator appears in, \ie {\ct S1, S2}.
