\section{Related Work}
\label{sec:related}
\Para{Endpoint-based monitoring.} Owing to limited switch support for
measurement, many systems monitor network performance from
endpoints alone~\cite{netpoirot, minlan-snap, dapper-sosr, trumpet,
azure-smartnic}. While endpoint solutions are necessary for application
context (\eg socket calls), they are inadequate to debug all network problems. A
real network needs both endpoint and switch-based
systems because each sees something the other cannot.

\Para{Switch-based monitoring.} Traditionally, switch-based monitoring has
focused on per-flow counts, not performance measurement. For example,
NetFlow~\cite{netflow} and sFlow~\cite{sflow} provide traffic summaries
through flow and packet sampling. Packet-capture systems~\cite{cisco-span,
niksun, netsight, everflow, pathdump, path_query} collect entire packets or
digests. These approaches sample extensively to lower collection
overheads, and are useful for posthoc traffic analysis. However, neither
approach captures details of {\em performance} phenomena (\eg TCP incast) as
specified by a flexible language like \TheSystem.

Sketches~\cite{univmon, flowradar, counterbraids, dream} and earlier work
on programmable switch measurements~\cite{progme, opensketch} provide traffic
volume statistics using summary data structures on switches.  Unlike sketches,
\TheSystem does
not have an accuracy-memory tradeoff, since counting is
linear-in-state and counters can be measured accurately. Instead, \TheSystem
trades off memory size with cache eviction rate (\Sec{eval}). \TheSystem also
allows users to perform a broader set of aggregations with full
accuracy.

%% \TheSystem also enables users
%% to perform other more general aggregations without losing accuracy.

%
%With INT alone, performance information may be lost, since packets carrying the
%INT data may be dropped on the way to endpoints.

In-band Network Telemetry (INT)~\cite{int, tpp} exposes queue lengths to
endpoints by stamping it on the packet itself. \TheSystem builds on INT and
provides flexible filters and aggregations {\em directly in switches}.
\TheSystem's data aggregation in switches saves
the bandwidth needed to collect INT data distributed over many endpoints.
In addition, the
Tetration chip provides flow-level telemetry, exposing a fixed set of metrics
including latency, window and packet size variation, and a ``burst
measurement''~\cite{tetration-telemetry}. In contrast, \TheSystem provides
programmable aggregation functions and aggregation levels.%% , \eg port
%% versus flow-level.

\Para{Programmable switches.} \TheSystem builds on programmable switch
architectures, leveraging their support for flexible
parsing~\cite{gibb_parsing}, forwarding~\cite{rmt, openflow} and stateful
processing~\cite{domino_sigcomm}. In addition, we design hardware for scalable
high-speed aggregation for a broad class of aggregation functions.

\TheSystem's vision and hardware primitive are similar to an earlier
position paper~\cite{marple-hotnets}. In addition to that work, \TheSystem
provides a functional query language (as opposed to SQL), a query compiler, a
formal characterization of which aggregation functions admit a scalable
implementation, a measurement of switch resources taken up by common queries,
and an end-to-end demonstration of its use.

%\begin{table}[t]
%\centering
%\small
%%\vspace{-0.15in}
%\begin{tabular}{|c|c|c|c|c|} \hline
%\bf{src $\rightarrow$ dst} & \bf{protocol} & \bf{\# Bursts} & \bf{Time ($\mu$s)} & \bf{\# Packets} \\ \hline
%h3:34573 $\rightarrow$ h4:4888 & UDP & 19 & 8969085 & 6090 \\
%h4:4888 $\rightarrow$ h3:34573 & UDP & 18 & 10558176 & 5820 \\
%h1:1777 $\rightarrow$ h2:58439 & TCP & 1 & 72196926 & 61584 \\
%h2:58439 $\rightarrow$ h1:1777 & TCP & 1 & 72248749 & 33074 \\ \hline
%\end{tabular}
%\caption{Per-flow burst statistics from \TheSystem.}
%\label{t:mininet-flowstats}
%\end{table}

%SK->NG: I don't think the Brown work is related to us.
%%\TheSystem shares the vision of developing new monitoring switch primitives with
%%the prior work of Nelson~\etal~\cite{stateful-switch-monitoring}. The semantic
%%switch features they propose pertain to stateful properties of network
%%protocols; our focus is on performance metrics. Notwithstanding, our key-value
%%store primitive can support a subset of their semantic features.
%% The key-value store we propose can be used to implement
%% some of the stateful property monitoring use cases in this paper, like the
%% stateful firewall and network address translation.

\Para{Network query languages.} Prior network query languages~\cite{gigascope,
frenetic, path_query, streaming-monitoring} allow users to ask questions
primarily about traffic volumes and count statistics, since their input data is
collected using NetFlow and match-action rule counters~\cite{openflow}. In
contrast, \TheSystem enables asking expressive {\em performance} questions on
data collected with purpose-built switch hardware. \TheSystem shares some
functional and relational constructs with Gigascope~\cite{gigascope} and
Sonata~\cite{streaming-monitoring}, but supports aggregations directly in the
switch.

\Para{Language-directed computer design.} %% The hardware design process
%% used to create \TheSystem is (i) creating a language that expresses the
%% kinds of switch monitoring we want and (ii) developing an efficient
%% hardware design that supports the language.
Our hardware design process is inspired by early
efforts on language-directed computer
design~\cite{language-directed-computer-design, ditzel_patterson, soar},
aimed at designing efficient hardware to support expressive high-level languages.
