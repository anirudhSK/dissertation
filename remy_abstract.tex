\begin{abstract}

  This paper describes a new approach to end-to-end congestion control
  on a multi-user network. Rather than manually formulate each
  endpoint's reaction to congestion signals, as in traditional
  protocols, we developed a program called Remy that generates
  congestion-control algorithms to run at the endpoints.

  In this approach, the protocol designer specifies their prior
  knowledge or assumptions about the network and an objective that the
  algorithm will try to achieve, e.g., high throughput and low
  queueing delay.  Remy then produces a distributed algorithm---the
  control rules for the independent endpoints---that tries to achieve
  this objective.

% Traditionally, congestion-control protocols
%  were developed by specifying the algorithm that endpoints run, so as
%  to achieve good performance and prevent congestion collapse.
%  We reverse this tradition. Instead of hand-crafting rules for
%  endpoint behavior, 
%In this approach, we specify an uncertain model of network scenarios
%and a utility function of throughput and delay. Remy then produces a
%distributed algorithm---that is, the control rules for the
%endpoints---that attempts to maximize the total utility.  Remy's
%output depends on the model of the network and workload, so it can
%change with time as network scenarios and workloads evolve.
%
% A computer
 % program called Remy then designs a congestion-control algorithm from
 % first principles in response to these inputs.
%The algorithm can be used by a TCP
%  sender, without changes to the receiver.

  In simulations with ns-2, Remy-generated algorithms
  outperformed human-designed end-to-end techniques, including TCP
  Cubic, Compound, and Vegas. In many cases, Remy's algorithms also
  outperformed methods that require intrusive in-network changes,
  including XCP and Cubic-over-sfqCoDel (stochastic fair queueing with
  CoDel for active queue management).

  Remy can generate algorithms both for networks where some parameters
  are known tightly \textit{a priori}, e.g.~datacenters, and for
  networks where prior knowledge is less precise, such as cellular
  networks. We characterize the sensitivity of the resulting
  performance to the specificity of the prior knowledge, and the
  consequences when real-world conditions contradict the assumptions
  supplied at design-time.

%\category{C.2.1}{Computer-Communication Networks}{Network Architecture
%  and Design --- Network communications}

%\keywords{congestion control, computer-generated algorithms}

%\terms{Design, Performance}

\end{abstract}
