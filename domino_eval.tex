\section{Evaluation}
\label{s:eval}

\begin{table}[!t]
\begin{small}
  \begin{tabular}{|p{0.3\columnwidth}|p{0.6\columnwidth}|p{0.06\columnwidth}|p{0.06\columnwidth}|}
\hline
Algorithm & Stateful operations & LOC & P4 LOC\\
\hline
Bloom filter (3 hash functions) & Test/Set membership bit on every packet. & 29 & 104 \\
\hline
Heavy Hitters~\cite{opensketch} (3 hash functions) & Increment count-min sketch~\cite{cormode} on every packet. & 35 & 192 \\
\hline
Flowlets~\cite{flowlets} & Update saved next hop if flowlet threshold is exceeded.& 37 & 107 \\
\hline
RCP~\cite{rcp} & Accumulate RTT sum if RTT is under maximum allowable RTT. & 23 & 75 \\
\hline
Sampled NetFlow~\cite{sampled_nflow} & Sample a packet if packet count reaches N. Reset count to 0 when it reaches N. & 18 & 70 \\
\hline
HULL~\cite{hull} & Update counter for virtual queue. & 26 & 95 \\
\hline
Adaptive Virtual Queue~\cite{avq} & Update virtual queue size and virtual capacity. & 36 & 147 \\
\hline
Priority computation for weighted fair queueing (Chapter~\ref{chap:pifo}) & Compute packet's virtual start time using finish time of last packet in that flow. & 29 & 87 \\
\hline
DNS TTL change tracking~\cite{dns_change} & Track number of changes in announced TTL for each domain. & 27 & 119 \\
\hline
CONGA~\cite{conga} & Update best path's utilization/id if we see a better path. Update best path utilization alone if it changes.  & 32 & 89\\
\hline
%trTCM~\cite{trTCM} & Update token counts for each token bucket & Doesn't map & 7, 3 & Either \\
%\hline
CoDel~\cite{codel} & Update whether we are marking or not, time for next mark, number of marks so far, and time at which minimum queueing delay will exceed target. & 57 & 271\\
\hline
\end{tabular}
\end{small}
\caption{Data-plane algorithms}
\label{tab:algorithms}
\end{table}


We evaluate \pktlanguage's expressiveness by using it to program several
data-plane algorithms (Table~\ref{tab:algorithms}), and comparing it to writing
them in P4~(\S\ref{ss:expressiveness}). To validate that these
algorithms can run at line rate, we design a concrete set of \absmachine
machines (Table~\ref{tab:templates}) as compiler targets for
\pktlanguage~(\S\ref{ss:targets}).  We estimate that these machines are
feasible in hardware because their atoms incur modest chip area overhead.  We
use the \pktlanguage compiler to compile the algorithms in
Table~\ref{tab:algorithms} to the targets in
Table~\ref{tab:templates}~(\S\ref{domino_ss:compiler}).  We conclude with some
lessons for programmable router design~(\S\ref{ss:lessons}).

\subsection{Expressiveness}
\label{ss:expressiveness}

We program several data-plane algorithms (Table~\ref{tab:algorithms}) using
\pktlanguage. These algorithms encompass data-plane traffic engineering,
in-network congestion control, active queue management, network security, and
measurement. We also used \pktlanguage to express the priority computation for
programming scheduling using push-in first-out queues, as described in greater
detail in Chapter~\ref{chap:pifo}.

In all these cases, the algorithms are already available as blocks of
imperative code from online sources; translating them to \pktlanguage syntax
was straightforward. In contrast, expressing any of them in P4\footnote{We are
referring to P4 at the time the Domino paper was published. As we describe in
\S\ref{s:impact}, many of Domino's ideas (including packet transactions) are
part of the latest version of P4.}
requires manually teasing out portions of the algorithm that can reside in
independent match-action tables and then chaining these tables together.

Of the algorithms in Table~\ref{tab:algorithms}, only flowlet switching has a
publicly available P4 implementation~\cite{p4_flowlet} that we can compare
against. This implementation requires 231 lines of uncommented P4, compared to
only 37 lines of \pktlanguage code in Figure~\ref{fig:flowlet_code}. Not only
that, using P4 also requires the programmer to manually specify tables, the
actions within tables, and how tables are chained---all to implement a single
data-plane algorithm. The \pktlanguage compiler automates this process; to
demonstrate this, we developed a backend for \pktlanguage that generates the
equivalent P4 code. We list the number of lines of code for these
auto-generated P4 programs in Table~\ref{tab:algorithms}.
%%
%%Data-plane algorithms on software platforms today (NPUs, Click, the Linux qdisc
%%subsystem~\cite{qdisc})  are programmed in languages resembling
%%\pktlanguage---hence we are confident that the \pktlanguage syntax is already
%%familiar to network engineers.
%%
\subsection{Compiler targets}
\label{ss:targets}

We design a set of compiler targets for \pktlanguage based on the \absmachine
machine model (\S\ref{s:absmachine}). First, we describe how to assess the
feasibility of atoms: whether they can run at a 1 GHz clock frequency, and what
area overhead they incur in silicon. Next, we discuss the design of stateless
and stateful atoms separately. Finally, we discuss how these stateless and
stateful atoms are combined together in our compiler targets.

\Para{Atom feasibility.}
We synthesize a digital circuit corresponding to an atom template by writing
the atom template in Verilog, and using the Synopsys Design
Compiler~\cite{synopsys_dc} to compile the Verilog code. The Design Compiler
checks if the resulting circuit meets timing at 1 GHz in a 32-nm standard-cell
library, and outputs its gate area. We use this gate area, along with the area
of a 200 \si{\milli\metre\squared} baseline router chip~\cite{gibb_parsing},
to estimate the area overhead for provisioning a \absmachine machine with
multiple instances of this atom.

% (a library of primitive gates designed using a
%transistor with a feature size of 32 nm)

\Para{Designing stateless atoms.}
Stateless atoms are easier to design because complicated stateless operations can
be broken up into multiple pipeline stages without violating
atomicity~(\S\ref{ss:atoms}). We design a stateless atom that can support
simple arithmetic (add, subtract, left shift, right shift), logical (and, or,
xor), relational ({\tt >=}, {\tt <=}, {\tt ==}, {\tt !=}), and conditional
instructions (C's ``{\tt ?}'' operator) on a pair of packet fields. Any packet
field can also be substituted with a constant operand. This stateless atom
meets timing at 1 GHz and occupies an area of 1384 \si{\micro\meter\squared}
(Table~\ref{tab:templates}).

\Para{Designing stateful atoms.}
The choice of stateful atoms determines the algorithms a line-rate router can
support. Unlike a stateless computation that can be easily pipelined over
multiple stateless atoms, there is no easy way to pipeline a stateful
computation over multiple stateful atoms, while still guaranteeing atomicity.
For instance, \S\ref{ss:atoms} illustrates how atomicity is violated even for a
counter, a very simple stateful operation.

However, designing the right set of stateful atoms for a router is a
chicken-and-egg problem: the choice of stateful atoms determines which
algorithms can execute on that target, while the choice of algorithms dictates
what stateful atoms are required in the router. Indeed, other programmable
substrates (\eg graphics processors and digital signal processors) go through
an iterative process to design a good instruction set.

 We use the Domino compiler to help us design a good set of stateful atoms.
Concretely, we pick a data-plane algorithm and partially execute the
\pktlanguage compiler to generate a codelet pipeline.  We then inspect the
stateful codelets, and create an atom that expresses all the computations
required by the stateful codelets. We verify that this atom can indeed express
all these computations by fully executing the compiler on the data-plane
algorithm with that atom as the target. We then move on to the next algorithm,
and check if the same atom suffices. If not, we repeatedly extend our atom
through a process of trial-and-error to capture more computations, all the
while using the compiler to verify our intuitions on extending atoms.

In the process, we generate a containment hierarchy of atoms
(Table~\ref{tab:templates}), each of which works for a subset of algorithms.
The complexity of the atoms increases as we proceed down this hierarchy. A more
complex stateful atom can support more data-plane algorithms, but occupies more
area and may not meet timing at 1 GHz. These atoms start out with the simplest
stateful capability: the ability to read or write state alone.  They then move
on to the ability to read, add, and write back state atomically (RAW), a
predicated version of the same (PRAW), and so on. When synthesized to a 32-nm
standard-cell library, all our stateful atoms meet timing at 1 GHz.  However,
the atom's area and minimum end-to-end propagation delay increases with the
atom's complexity~(Table~\ref{tab:templates}).

\Para{The compiler targets.}
We design seven \absmachine machines as compiler targets. A single \absmachine
machine has 600 atoms.
\begin{CompactEnumerate}
\item 300 are stateless atoms of the single stateless atom type from
Table~\ref{tab:templates}.
\item 300 are stateful atoms of one of the seven stateful atom types from
Table~\ref{tab:templates} (Read/Write through Pairs).
\end{CompactEnumerate}
These 300 stateless and stateful atoms are laid out physically as 10 stateless
and stateful atoms per pipeline stage and 30 pipeline stages. While the number
300 and the pipeline layout are arbitrary, they are sufficient for all examples
in Table~\ref{tab:algorithms}, and incur modest area overhead as we show next. In
the future, we anticipate these numbers being decided empirically based on what
algorithms are most frequently run on such platforms.

We estimate the area overhead of these seven targets relative to a 200
\si{\milli\metre\squared} chip~\cite{gibb_parsing}, which is at the lower end
of chip sizes today. For this, we multiply the individual atom areas from
Table~\ref{tab:templates} by 300 for both the stateless and stateful atoms. For
300 atoms, the area overhead is 0.2 \% for the stateless atom and 0.9 \% for
the Pairs atom, the largest among our stateful atoms.  The area overhead
combining both stateless and stateful atoms for all our targets is at most
1.1\%---a modest price for the programmability it provides.
%TODO: Add mux area? I think it's too detailed to include here.
\begin{table}[!t]
  \centering
  \begin{small}
  \begin{tabular}{|p{0.26\textwidth}|p{0.36\textwidth}|p{0.08\textwidth}|p{0.04\textwidth}|}
    \hline
    Atom & Description & Area (\si{\micro\metre\squared}) at 1 GHz & Min. delay (ps) \\
    \hline
    Stateless & Arithmetic, logic, relational, and conditional operations on packet/constant operands & 1384 & 387 \\
    \hline
    Read/Write & Read/Write packet field/constant into single state variable. & 250 & 176 \\
    \hline
    ReadAddWrite (RAW) & Add packet field/constant to state variable (OR) Write packet field/constant into state variable. & 431 & 316 \\
    \hline
    Predicated ReadAddWrite (PRAW) & Execute RAW on state variable only if a predicate is true, else leave unchanged. & 791 & 393 \\
    \hline
    IfElse ReadAddWrite (IfElseRAW) & Two separate RAWs: one each for when a predicate is true or false. & 985 & 392 \\
    \hline
    Subtract (Sub) & Same as IfElseRAW, but also allow subtracting a packet field/constant. & 1522 & 409 \\
    \hline
    Nested Ifs (Nested) & Same as Sub, but with an additional level of nesting that provides 4-way predication. & 3597 & 580 \\
    \hline
    Paired updates (Pairs) & Same as Nested, but allow updates to a pair of state variables, where predicates can use the earlier values of both state variables. & 5997 & 606 \\
    \hline
  \end{tabular}
  \end{small}
  \caption{Atom areas and minimum critical-path delays in a 32-nm standard-cell
library.  All atoms meet timing at 1 GHz. Each of the seven compiler targets
contains 300 instances of one of the seven stateful atoms (Read/Write to Pairs)
and 300 instances of the single stateless atom.}
  \label{tab:templates}
\end{table}

\subsection{Compiling \pktlanguage algorithms to \absmachine machines}
\label{domino_ss:compiler}
We now consider every target from Table~\ref{tab:templates} and every
data-plane algorithm from Table~\ref{tab:algorithms} to determine if the algorithm
can run at the line rate of a particular \absmachine machine. Because every
target is uniquely identified by its stateful atom type, we use the name of the
stateful atom to refer to the target itself.

We say an algorithm can run at line rate on a \absmachine machine if every
codelet within the data-plane algorithm can be mapped (\S\ref{ss:code_gen}) to
either the stateful or stateless atoms provided by the \absmachine machine.
Because our stateful atoms are arranged in a containment hierarchy, we list the
\textit{most expressive} stateful atom/target required for each data-plane
algorithm in Table~\ref{tab:algo_atoms}.
\begin{table}[!t]
\begin{small}
\begin{tabular}{|p{0.12\columnwidth}|p{0.1\columnwidth}|p{0.05\columnwidth}|p{0.05\columnwidth}|p{0.08\columnwidth}|p{0.25\columnwidth}|p{0.17\columnwidth}|}
\hline
Algorithm & Most expressive atom required & Stages & Max. atoms/stage & Ingress or Egress Pipeline? & Amount of stateful memory proportional to? & Guard \\
\hline
Bloom filter & Read/Write & 4 & 3 & Either & (Size of bloom filter array) * (\# of hash functions) & Match all packets \\
\hline
Heavy hitters & RAW & 10 & 9 & Either & (Size of sketch array) * (\# of hash functions) & Match all packets \\
\hline
Flowlet switching & PRAW & 6 & 2 & Ingress & Size of flowlet table & Match all packets \\
\hline
RCP & PRAW & 3 & 3 & Egress & Constant: 3 scalar integers (total number of bytes, sum of RTTs, number of packets) & Match all packets \\
\hline
Sampled NetFlow & IfElseRAW & 4 & 2 & Either & Constant: 1 scalar integer for the counter & Match all packets \\
\hline
HULL & Sub & 7 & 1 & Egress & Number of output ports & Match on output port \\
\hline
AVQ & Nested & 7 & 3 & Ingress & Number of output ports & Match on output port \\
\hline
Priorities for weighted fair queueing & Nested & 4 & 2 & Ingress & Number of flows & Match on output port \\
\hline
DNS TTL change tracking~\cite{dns_change} & Nested & 6 & 3 & Ingress & Number of DNS records tracked & Match all DNS packets  \\
\hline
CONGA & Pairs & 4 & 2 & Ingress & Number of destination ToRs & Match all packets \\
\hline
%trTCM~\cite{trTCM} & Update token counts for each token bucket & Doesn't map & 7, 3 & Either \\
%\hline
CoDel & Doesn't map & 15 & 3 & Egress & Number of output ports or number of output queues & Match on output port or output queue \\
\hline
\end{tabular}
\end{small}
\caption{Compiling Domino algorithms to Banzai machines with different atoms}
\label{tab:algo_atoms}
\end{table}


As Table~\ref{tab:algo_atoms} shows, the choice of stateful atom determines what
algorithms can run on a router. For instance, with only the ability to read or
write state, only Bloom filters can run at line rate, because it
only requires the ability to test and set membership bits.  Adding the ability
to increment state (the RAW atom) permits heavy hitter detection to run at line rate,
because it employs a count-min sketch that is incremented on each packet.

Table~\ref{tab:algo_atoms} also lists where each algorithm runs (ingress or
egress), the algorithm's memory requirements for storing state, and guards
(\S\ref{ss:guards}) for each algorithm. We explain each of these three columns
below.

An algorithm may run either on the ingress pipeline (\eg load balancing
algorithms like flowlet switching), egress pipeline (\eg active queue
management algorithms like CoDel), or both (\eg packet sampling algorithms like
sampled NetFlow). This choice depends on what information the algorithm needs
to execute and what other router functionlity depends on this algorithm. For
instance, CoDel needs access to a packet's queueing delay, which is only
available after the packet is dequeued from the scheduler and sent to the
egress pipeline. On the other hand, flowlet switching is a load balancing
algorithm that determines the next hop for a packet (and hence the output
port). Because the output port determines which output queue within the
scheduler the packet must be enqueued into, it needs to be determined in the
ingress pipeline, before the packet enters the scheduler.

Different algorithms require different amounts of memory to store the state
associated with the algorithm. For instance, the amount of stateful memory is
proportional to the size of each array and the number of arrays (hash
functions) in the case of heavy hitter detection and Bloom filters. On the
other hand, the amount of stateful memory is proportional to the number of
output ports for algorithms that run independently for each output port (\eg
HULL, AVG) or proportional to the number of output queues for algorithms that
run independently at each output queue (\eg CoDel).

The guard column specifies how packets are selected for execution by an
algorithm's packet transaction. For instance, the heavy hitter detection
algorithm runs a single packet transaction on all packets, while the DNS TTL
change tracking algorithm runs a single packet transaction on all {\em DNS}
packets. On the other hand, to implement CoDel, the router needs to run a
separate instance of CoDel's packet transaction for either each output port or
each output queue (depending on whether the CoDel algorithm is running
independently on a single queue for each output port or independently on
multiple queues for each output port). In the case of CoDel, the guard selects
packets by their output port or output queue and runs the appropriate instance
of the same CoDel packet transaction on the selected packet.

\subsection{Lessons for programmable routers}
\label{ss:lessons}

\Para{Atoms that allow modifications to a single state variable support many
algorithms.} The algorithms from Bloom Filter through DNS TTL Change Tracking
in Table~\ref{tab:algo_atoms} can run at line rate using the Nested Ifs atom that
modifies a single state variable.

\Para{But, some algorithms modify a pair of state variables atomically.}
An example is CONGA; we reproduce CONGA's relevant code snippet below:
\begin{verbatim}
  if (p.util < best_path_util[p.src]) {
    best_path_util[p.src] = p.util;
    best_path[p.src] = p.path_id;
  } else if (p.path_id == best_path[p.src]) {
    best_path_util[p.src] = p.util;
  }
\end{verbatim}
Here, \texttt{best\_path} (the path id of the best path for a particular
destination) is updated conditioned on \texttt{best\_path\_util} (the
utilization of the best path to that destination)\footnote{{\tt p.src} is the
address of the host originating this message, and hence the destination for the
host receiving it and executing CONGA.} and vice versa. These two state
variables cannot be separated into different stages and still guarantee a
packet transaction's semantics because they are mutually dependent on each
other.  The Pairs atom, where the update to a state variable is conditioned on
a predicate of a pair of state variables, allows CONGA to run at line rate.

\Para{There will always be algorithms that cannot run at a target's line rate.}
While the targets and their atoms in Table~\ref{tab:templates} are sufficient
for several data-plane algorithms, there are algorithms that they cannot run at
line rate.  An example is CoDel, which cannot be implemented because it
requires a square root operation that is not provided by any of our targets.
One possibility is a look-up table abstraction that allows us to approximate
such mathematical functions. However, regardless of what set of atoms we design
for a particular target, there will always be algorithms that cannot run at
line rate.

%%This is because the set of computations that can be
%%expressed by an atom (or a finite pipeline of atoms) are finite because these
%%computations all need to finish within a ns, while the set of algorithms is
%%infinite.

\begin{table}[!t]
  \centering
  \begin{small}
    \begin{tabular}{|p{0.18\textwidth}|p{0.61\textwidth}|p{0.06\textwidth}|}
  \hline
  Atom & Circuit & Min. delay (ps) \\
  \hline
  Read/Write & \centering\includegraphics[width=0.4\textwidth]{domino_rw.pdf} & 176 \\
  \hline
  ReadAddWrite (RAW) & \centering\includegraphics[width=0.5\textwidth]{domino_raw.pdf} & 316\\
  \hline
  \pbox{0.1\textwidth}
  {Predicated\\
  ReadAddWrite (PRAW)} & \centering\includegraphics[width=0.6\textwidth]{domino_pred_raw.pdf}  & 393 \\
  \hline
  \end{tabular}
\end{small}
\caption{An atom's minimum critical-path delay (\ie, the lowest possible
critical-path delay on a particular standard cell library) increases with
circuit depth.  Mux is a multiplexer. RELOP is a relational operation (>, <,
==, !=). {\tt x} is a state variable. {\tt pkt.f1} and {\tt pkt.f2} are packet
fields. {\tt Const} is a constant operand.}
\label{tab:circuits}
\end{table}



\Para{Atom design is constrained by timing, not area.} Atoms are affected by
two factors: their area and their timing, \ie the delay on the critical path of
the atom's combinational circuit. For the few hundred atoms that we require,
atom area is insignificant (< 2\%) relative to chip area.  Further, even for
future atoms that are larger, area may be controlled by provisioning fewer atom
instances.

However, atom timing is critical. Table~\ref{tab:templates} shows a 3.4
$\times$ range in minimum critical-path delay (\ie the lowest achievable
critical path on a particular standard-cell library) between the simplest and
the most complex atoms.  This increase can be explained by looking at the
simplified circuit diagrams for the first three atoms
(Table~\ref{tab:circuits}), which show an increase in circuit depth with atom
complexity.

%TODO: critical-path delay or min. critical-path delay. Decide.
% This is really really muddled.
Because the clock frequency of a circuit is at least as small as the reciprocal
of the critical-path delay, a more complex atom results in a lower clock
frequency and a lower line rate. Although all our atoms have a minimum
critical-path delay under 1 ns (1 GHz), it is conceivable that they can be
extended with functionality that causes the atom to violate timing at 1 GHz.

In summary, for a router designer, the critical path of atoms is the most
important metric to optimize. The most programmable line-rate routers will have
the highest density of useful stateful functionality squeezed into a critical
path budget of 1 clock cycle.

%%\textbf{Compilation time:}
%%Compilation time is dominated by SKETCH's search procedure.  To speed up the
%%search, we limit SKETCH to search for constants (\eg for addition) of size up
%%to 5 bits, given that the constants seen within stateful codelets in our
%%algorithms are small. Our longest compilation time is 10 seconds when CoDel
%%doesn't map to a \absmachine machine with the Pairs atom because SKETCH has to
%%rule out every configuration in its search space.  This time will increase if
%%we increase the bit width of constants that SKETCH has to search; however,
%%because the data-plane algorithms themselves are small, we don't expect
%%compilation times to be a concern.
