\section{Query compiler}
\label{sec:compiler}

We compile \TheSystem queries to two targets: the P4 behavioral
model~\cite{p4-bmv2}, configured by emitting P4 code~\cite{p4_16}, and the
Banzai machine model, configured by emitting Domino code~\cite{domino_sigcomm}.
In both cases, the emitted code configures a switch pipeline, where each stage
is a match-action table~\cite{rmt} or our key-value
store.\footnote{In this paper, we do not consider the problem of reconfiguring
the switch pipeline on the fly as queries change.}

A preliminary pass of the compiler over the input query converts the query to an
abstract syntax tree (AST) of functional operators (\Fig{compiler-ast-manipulations}a). The compiler then:
\begin{enumerate}
\item produces switch-local ASTs from a global AST
  (\Sec{network-to-switch-local});
\item produces P4 and Domino pipeline configurations from switch-local ASTs
  (\Sec{pipeline-layout}); and
\item specifically recognizes linear-in-state aggregation functions, and sets up
  auxiliary state required to merge such functions for a scalable implementation
  (\Sec{linear-in-state-compilation}). To scalably implement associative
  aggregation functions (\S\ref{sec:associative}), we use the programmer
  annotation {\ct assoc} to determine if an aggregation is associative. If it is
  associative, the merge function is the aggregation function itself.
\end{enumerate}

\begin{figure}[!t]{
\figcodesize
\begin{lstlisting}
def oos_count([count, lastseq], [tcpseq, payload_len]):
  if lastseq != tcpseq:
    count = count + 1
    emit()
  lastseq = tcpseq + payload_len

tcps    = filter((*\pktlog{}*), proto == TCP
                 and (switch == S1 or switch == S2));
tslots  = map((*\pktlog{}*), [tin/epoch_size], [epoch]);
joined  = zip(tcps, tslots);
oos     = groupby(joined,
                  [(*\codeftuple{}*), switch, epoch],
                  oos_count);
\end{lstlisting}
}
\caption{Running example for \TheSystem compiler (\Sec{compiler}).}
\label{fig:running-example-code}
\end{figure}

\begin{figure}
  \centering
  \[
  \begin{array}{ccc}
    \includegraphics[width=0.31\columnwidth]{running-example-AST.pdf} &
    \includegraphics[width=0.31\columnwidth]{running-example-annotated-AST-2.pdf} &
    \includegraphics[width=0.31\columnwidth]{running-example-annotated-AST-3.pdf}
    \\
    \mbox{(a)} & \mbox{(b)} & \mbox{(c)} \\
  \end{array}
  \]
\caption{Abstract Syntax Tree (AST) manipulations for the running example
  (\Sec{compiler}). (a) Operator AST. (b) Stream location (set of switches). (c)
  Stream switch-partitioned (boolean).}
\label{fig:compiler-ast-manipulations}
\end{figure}

We use the query shown in \Fig{running-example-code} as a running example to
illustrate the details in the compiler. The query counts the number of
out-of-sequence TCP packets over each time epoch, measured at two switches {\ct
  S1} and {\ct S2} in the network.


%% four main pieces of compilation
\input{compiler-global-to-local}
\input{compiler-ast-to-pipeline}
\input{compiler-bph}
