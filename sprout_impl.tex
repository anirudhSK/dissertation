\section{Experimental testbed}
\label{s:impl}
\label{ss:platform}
We use trace-driven emulation to evaluate Sprout and compare it with
other applications and protocols under reproducible network
conditions. Our goal is to capture the variability of cellular
networks in our experiments and to evaluate each scheme under the same
set of time-varying conditions.

\subsection{Saturator} Our strategy is to characterize the behavior
of a cellular network by saturating its uplink and downlink at the
same time with MTU-sized packets, so that neither queue goes empty. We
record the times that packets actually cross the link, and we treat
these as the ground truth representing all the times that packets
\emph{could} cross the link as long as a sender maintains a backlogged
queue.

\label{ss:saturator}

Because even TCP does not reliably keep highly variable links
saturated, we developed our own tool. The Saturator runs on a laptop
tethered to a cell phone (which can be used while in a car) and on a
server that has a good, low-delay ($<$ 20~ms) Internet path to the
cellular carrier.

The sender keeps a window of $N$ packets in flight to the receiver,
and adjusts $N$ in order to keep the observed RTT greater than 750~ms
(but less than 3000~ms). The theory of operation is that if packets
are seeing more than 750~ms of queueing delay, the link is not
starving for offered load. (We do not exceed 3000~ms of delay because
the cellular network may start throttling or dropping packets.)

\begin{figure}
  \caption{Block diagram of Cellsim}
\hspace{\baselineskip}

\noindent \def\svgwidth{\columnwidth}\input{emu.pdf_tex}

\label{f:cellsim}

\end{figure}

There is a challenge in running this system in two directions at once
(uplink and downlink), because if both links are backlogged by
multiple seconds, feedback arrives too slowly to reliably keep both
links saturated. Thus, the Saturator laptop is actually connected to
\emph{two} cell phones. One acts as the ``device under test,'' and its
uplink and downlink are saturated.  The second cell phone is used only
for short feedback packets and is otherwise kept unloaded. In our
experiments, the ``feedback phone'' was on Verizon's LTE network,
which provided satisfactory performance: generally about 20~ms delay
back to the server at MIT.

We collected data from four commercial cellular networks: Verizon
Wireless's LTE and 3G (1xEV-DO / eHRPD) services, AT\&T's LTE service,
and T-Mobile's 3G (UMTS) service.\footnote{We also attempted a
  measurement on Sprint's 3G (1xEV-DO) service, but the results
  contained several lengthy outages and were not further analyzed.} We
drove around the greater Boston area at rush hour and in the evening
while recording the timings of packet arrivals from each network,
gathering about 17 minutes of data from each. Because the traces were
collected at different times and places, the measurements cannot be
used to compare different commercial services head-to-head.

The Verizon LTE and 1xEV-DO (3G) traces
were collected with a Samsung Galaxy Nexus smartphone running Android 4.0. The AT\&T
trace used a Samsung Galaxy S3 smartphone running Android 4.0, and the
T-Mobile trace used a Samsung Nexus S smartphone running Android 4.1.

\subsection{Cellsim}

We then replayed the traces in a network emulator, which we call
Cellsim (Figure~\ref{f:cellsim}). It runs on a PC and takes in packets
on two Ethernet interfaces, delays them for a configurable amount of
time (the propagation delay), and adds them to the tail of a
queue. Cellsim releases packets from the head of the queue to the
other network interface according to the same trace that was
previously recorded by Saturator. If a scheduled packet delivery
occurs while the queue is empty, nothing happens and the opportunity
to deliver a packet is wasted.\footnote{This accounting is done on a
  per-byte basis. If the queue contains 15 100-byte packets, they will
  all be released when the trace records delivery of a single
  1500-byte (MTU-sized) packet.}

Empirically, we measure a one-way delay of about 20~ms in each
direction on our cellular links (by sending a single packet in one
direction on the uplink or downlink back to a desktop with good
Internet service). All our experiments are done with this
propagation delay, or in other words a 40~ms minimum RTT.

Cellsim serves as a transparent Ethernet bridge for a Mac or PC under
test. A second computer (which runs the other end of the connection)
is connected directly to the Internet. Cellsim and the second computer
receive their Internet service from the same gigabit Ethernet switch.

We tested the latest (September 2012) real-time implementations of all
the applications and protocols (Skype, Facetime, etc.) running on
separate late-model Macs or PCs (Figure~\ref{f:software}).

\begin{figure}
  \caption{Software versions tested}
\hspace{\baselineskip}

\begin{centering}

\label{f:software}

%\scriptsize
\noindent \begin{tabular}{|l|l|l|l|}
\hline
Program & Version & OS & Endpoints \\
\hline
\hline
Skype & 5.10.0.116 & Windows 7 & Core i7 PC \\
Hangouts & as of 9/2012 & Windows 7 & Core i7 PC \\
Facetime & 2.0 (1070) & OS X 10.8.1 & MB Pro (2.3 GHz i7),\\
& & & MB Air (1.8 GHz i5) \\
TCP Cubic & in Linux 3.2.0 & & Core i7 PC \\
TCP Vegas & in Linux 3.2.0 & & Core i7 PC \\
LEDBAT & in $\mu$TP & Linux 3.2.0 & Core i7 PC \\
Compound TCP & in Windows 7 & & Core i7 PC \\
\hline
\end{tabular}

\end{centering}
\end{figure}

We also added stochastic packet losses to Cellsim to study Sprout's
loss resilience. Here, Cellsim drops packets from the tail of the
queue according to a specified random drop rate.  This approach
emulates, in a coarse manner, cellular networks that do not have deep
packet buffers (e.g., Clearwire, as reported
in~\cite{Mahajan12}). Cellsim also includes an optional implementation
of CoDel, based on the pseudocode in ~\cite{CoDel}.

\subsection{SproutTunnel}

We implemented a UDP tunnel that uses Sprout to carry arbitrary
traffic (e.g.~TCP, videoconferencing protocols) across a cellular link
between a mobile user and a well-connected host, which acts as a relay
for the user's Internet traffic. SproutTunnel provides each flow with
the abstraction of a low-delay connection, without modifying carrier
equipment. It does this by separating each flow into its own queue,
and filling up the Sprout window in round-robin fashion among the
flows that have pending data.

The total queue length of all flows is limited to the receiver's most
recent estimate of the number of packets that can be delivered over
the life of the forecast. When the queue lengths exceed this value,
the tunnel endpoints drop packets from the head of the longest
queue. This algorithm serves as a dynamic traffic-shaping or
active-queue-management scheme that adjusts the amount of buffering to
the predicted channel conditions.
