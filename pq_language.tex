\section{The \TheSystem Query language}
\label{sec:language}
This section describes the \TheSystem query language.  \S\ref{sec:aggregation}
then covers the router implementation of the language constructs, while
\S\ref{sec:compiler} describes the compiler.  \TheSystem provides the
abstraction of a network-wide stream of performance information. The tuples in
the stream contain performance metadata, such as queue lengths and timestamps
when a packet entered and departed queues, for each packet at each queue in the
network. Network operators write queries on this stream as if the entire stream
is processed by a single hypothetical server running the query. In reality, the
compiler partitions the query across the network and executes each part on
individual routers.

\TheSystem programs process the performance stream using familiar functional
constructs ({\ct filter}, {\ct map}, {\ct groupby}, and {\ct zip}), all of
which take streams as inputs and produce a stream as output. This functional
language model is expressive enough to support diverse performance monitoring
use cases, but still simple enough to implement in high-speed hardware.
\TheSystem's language constructs are summarized in
Table~\ref{tab:language-constructs-table}.

\begin{figure}[!t]
\centering
\begin{tabular}{ll}
\hline

\textbf{Construct} &
\textbf{Description} \\

\hline
\hline

{\ct \pktlog} &
Stream of packet performance metadata. \\

\hline

{\ct filter(R, pred)} &
Output tuples in {\ct R} satisfying predicate {\ct pred}. \\

\hline

{\ct map(R, [exprs],} &
Evaluate expressions over fields of {\ct R}, emitting \\

{\ct \ \ [fields])} &
tuples with new fields. \\

\hline

{\ct groupby(R, } &
Evaluate function {\ct fun} over the input stream {\ct R} \\

{\ct \ \ [fields], fun)} &
partitioned by {\ct fields}, producing tuples on {\ct emit().} \\

\hline

{\ct zip(R, S)} &
Merge fields in incoming {\ct R} and {\ct S} tuples.\\

%% &
%% are both valid. \\

\hline

\end{tabular}
\caption{Summary of \TheSystem language constructs.}
\label{fig:language-constructs-table}
\end{figure}


\subsection{Packet performance stream} As part of the base input stream, which we
call {\ct \pktlog}, \TheSystem provides one tuple for each packet at each queue
with the following fields.

\begin{lstlisting}
(router, qid, hdrs, uid, tin, tout, qsize)
\end{lstlisting}

{\ct router} and {\ct qid} denote the router and queue at which the
packet was observed. A packet may traverse multiple queues even within a single
router, so we provide distinct fields. The regular packet headers (Ethernet,
IP, TCP, \etc) are available in the {\ct hdrs} set of fields, with a {\ct uid}
that uniquely determines a packet.\footnote{It is usually possible to
use a combination of the \txtftuple and IP ID field as the {\ctfoot uid}.}

The packet performance stream provides access to a variety of performance
metadata: {\ct tin} and {\ct tout} denote the enqueue and dequeue timestamps of
a packet, while {\ct qsize} denotes the queue depth when a packet is enqueued.
It is beneficial to have two timestamps to detect co-habitation of the queue by
packets belonging to different flows.  Additionally, it is beneficial to have a
queue size, since we cannot always determine the queue size from the two
timestamps: a link may service multiple queues, and the speed at which any
specific queue drains may not be known. 

Tuples in {\ct \pktlog} are processed in order of packet dequeue time ({\ct
tout}), because this is the earliest time at which all tuple fields in {\ct
\pktlog} are known.\footnote{We assume clock synchronization to let us compare
{\ctfoot tin} and {\ctfoot tout} values from different routers. Without
synchronization, the programmer can still write queries that do not compare
time-valued fields {\ctfoot tin} and {\ctfoot tout} across routers.} If a
packet is dropped, {\ct tout} and {\ct qsize} are {\ct infinity.} Tuples
corresponding to dropped packets may be processed in an arbitrary order.

\subsection{Restricting packet performance metadata of interest} Consider the
example of tracking packets that experience high queueing latencies at a
specific queue ({\ct Q}) and router ({\ct S}). This is expressed by the query:

\begin{lstlisting}
result = filter((*\pktlog,*) qid == Q and router == S
                and tout - tin > 1ms)
\end{lstlisting}

As shown above, the {\ct filter} operator restricts the user's attention to
those tuples with the relevant performance metadata. A filter has the form {\ct
filter(R, pred)} where {\ct R} is some stream containing performance metadata
(\eg {\ct \pktlog}), and the filter predicate {\ct pred} may involve packet
headers, performance metadata, or both. The result of a {\ct filter} is another
stream that contains only tuples satisfying the predicate.

\subsection{Computing stateless functions over packets} \TheSystem lets users
compute functions of the fields available in the incoming stream, to express
new quantities of interest. A simple example is rounding packet timestamps to
an `epoch':

\begin{lstlisting}
result = map((*\pktlog,*) [tin/epoch_size], [epoch]);
\end{lstlisting}

The {\ct map} operator evaluates the expression {\ct tin/epoch\_size}, written
over the fields available in the tuple stream, and produces a new field {\ct
epoch}. The general form of this construct is {\ct map(R, [expression],
[field])} where a list of {\ct expression}s over fields in the input stream
{\ct R} creates a list of new {\ct field}s in the {\ct map} output stream.

\subsection{Aggregating statefully over multiple packets} \TheSystem allows
aggregating statistics over multiple tuples at user-specified granularities.
For example, the following query counts packets belonging to each
transport-level flow (\ie 5-tuple):

\begin{lstlisting}
result = groupby((*\pktlog{}*), [(*\codeftuple{}*)], count)
def count([c], []):
  c = c + 1;
\end{lstlisting}

Here, the {\ct groupby} partitions the incoming {\ct \pktlog} into substreams
based on the transport 5-tuple, and then applies the aggregation function {\ct
count} to count the number of tuples in each substream.

\TheSystem allows users to write flexible order-dependent aggregation functions
over the tuples of each substream. For example, a user can track latency spikes
for each connection by maintaining an exponentially weighted moving average
(EWMA) of queueing latencies:

\begin{lstlisting}
result = groupby((*\pktlog{}*), [(*\codeftuple{}*), router], ewma);
def ewma([avg], [tin, tout]):
  avg = ((1-alpha)*avg) + (alpha*(tout-tin));
\end{lstlisting}

Here the aggregation function {\ct ewma} evolves an EWMA {\ct avg} using the
current value of {\ct avg} and incoming packet timestamps. Unlike the previous
{\ct count} example, the EWMA aggregation function depends on the order of
packets being processed, \ie the order of their $tout$ values.

{\ct groupby}s take the general form {\ct groupby(R, [aggFields], fun)}, where
the aggregation function {\ct fun} operates over tuples sharing attributes in a
list {\ct aggFields} of headers and performance metadata. This construct is
inspired by folds in functional
programming~\cite{comprehensive-comprehensions}.  Such order-dependent folds
are challenging to express in existing query languages. For instance, SQL only
allows order-independent commutative aggregations (\eg count, average, sum).

The aggregation function {\ct fun} is written in an imperative form, with two
arguments: (1) a list of state variables that represent the value(s) being
aggregated across packets and (2) a list of relevant incoming tuple fields that
are used in the aggregation function. Each statement in {\ct fun} can be an
assignment to an expression ({\ct x = ...}), a branching statement ({\ct if
pred \{...\} else \{...\}}), or a special {\ct emit()} statement that controls
the output stream of the {\ct groupby}. Below, we show an example of an
aggregation that detects a new connection:

\begin{lstlisting}
result = groupby((*\pktlog{}*), [(*\codeftuple{}*)], new_flow);
def new_flow([fcount], []):
  if fcount == 0:
    fcount = 1
    emit()
\end{lstlisting}

The output of a {\ct groupby} is a stream containing the aggregation fields
(\eg \txtftuple) and the aggregated values (\eg {\ct fcount}). The output
stream contains only tuples for which the {\ct emit()} statement is encountered
when executing the aggregation function. For example, the output stream of {\ct
new\_flow} consists of the first packet of every new transport-level
connection. If the function has no {\ct emit()}s, the user can still read the
aggregated fields and their current aggregated state values as a table.

\subsection{Chaining together multiple queries}
Because all \TheSystem constructs produce and consume streams, \TheSystem
allows users to write queries that take in the results of previous queries as
inputs. A stream of tuples flows from one query to the next, and each query may
modify information in the incoming tuple or even drop the tuple entirely. For
example, the program below tracks the size distribution of flowlets, \ie bursts
of packets from the same \txtftuple separated by more than a fixed time amount
{\ct delta}.

\begin{lstlisting}
fl_track = groupby((*\pktlog{}*), [(*\codeftuple{}*)], fl_detect);
def fl_detect([last_time, size], [tin]):
  if (tin - last_time > delta):
    emit()
    size = 1
  else:
    size = size + 1
  last_time = tin
\end{lstlisting}

The function {\ct fl\_detect} detects new flowlets using the last time a packet
from the same flow was seen. Because of the {\ct emit()} statement's location,
the flowlet size from {\ct fl\_track} is only streamed out to other operators
when the first packet of a new flowlet is encountered.

\begin{lstlisting}
fl_bkts  = map(fl_track, [size/16], [bucket]);
fl_hist  = groupby(fl_bkts, [bucket], count);
\end{lstlisting}

The map {\ct fl\_bkts} bins the flowlet size emitted by {\ct fl\_track} into a
bucket index, which is used to count the number of flowlets in the
corresponding bucket in {\ct fl\_hist}.

\subsection{Joining results across queries} \TheSystem provides a {\ct zip} operator
that ``joins'' (in the database sense of the word) the results of two queries
to check whether two conditions hold simultaneously. Consider the example of
detecting the fan-in of packets from many connections into a single queue,
characteristic of TCP incast~\cite{tcpincast}. This can be checked by combining
two distinct conditions: (1) the number of active flows in a queue over a short
interval of time is high and (2) the queue occupancy is large.

A user can first compute the number of active flows over the current epoch using
two aggregations:

\begin{lstlisting}
R1 = map((*\pktlog{}*), [tin/epoch_size], [epoch]);
R2 = groupby(R1, [(*\codeftuple{}*), epoch], new_flow);
R3 = groupby(R2, [epoch], count);
\end{lstlisting}

The number of active flows in this epoch can be combined with the queue
occupancy information in the original packet stream through the {\ct zip}
operator:

\begin{lstlisting}
R4 = zip(R3, (*\pktlog{}*));
result = filter(R4, qsize > 100 and count > 25);
\end{lstlisting}

%TODO: Too wordy below. Could be clearer.
The result of a {\ct zip} operation over two input streams is a single stream
containing tuples that are a concatenation of all the fields in the two
streams, whenever both input streams contain valid tuples processed from the
same original packet tuple. A {\ct zip} is a special kind of stream join where
the result can be computed without having to synchronize the two streams,
because tuples of both streams originate from {\ct \pktlog}. The result of the
{\ct zip} can be processed like any other stream: the {\ct filter} in the {\ct
result} query checks the two incast conditions above.

We did not find a need for more general joins akin to joins in streaming query
languages like CQL~\cite{cql-vldb}. Streaming joins have semantics that can be
quite complex and may produce large results, \ie $O(\#pkts^2)$.  Hence,
\TheSystem restricts users to simple {\ct zip} joins.

We show several examples of \TheSystem queries in
Table~\ref{tab:example-perf-queries}.  For instance, \TheSystem can express
measurements of simple counters, TCP reordering of various forms, high-loss
connections, flows with high end-to-end network latencies, and TCP fan-in. 

%Further, the \TheSystem compiler
%(\Sec{compiler}) rejects queries that require processing multiple packets in
%order across multiple routers.

\subsection{Restrictions on \TheSystem queries}
Some aggregations are challenging to implement over a network-wide stream. For
example, consider an EWMA over some packet field across all packets seen
anywhere in the entire network, while processing packets in the order of their
{\ct tout} values. Even with clock synchronization, this aggregation is hard to
implement because it requires us to either (1) coordinate between routers or
(2) stream all packets to a central location. Both of these alternatives impose
significant overheads; hence, we disallow such queries.

\TheSystem's compiler rejects queries with aggregations that need to process
{\em multiple packets} at {\em multiple routers} in {\em order of their {\ct
tout} values}. Concretely, we only allow aggregations that relax one of these
three conditions, and thus either
\begin{enumerate}
\item operate independently on each router, in which case we naturally
  partition queries by router (\eg a per-flow EWMA of queueing latencies on a
  particular queue belonging to a particular router), or
\item operate independently on each packet, in which case we have the packet
  perform the coordination by carrying the aggregated state to the next router on
  its path (\eg a rolling average link utilization seen by the packet
  along its path), or
\item are associative and commutative, in which case independent router-local results can be
  combined in any order to produce a correct overall result for the network,
  \eg a count of how many times packets from a flow
  appeared throughout the network. In this case, we rely on the
  programmer to annotate the aggregation function with the
  {\ct assoc} and {\ct comm} keywords.
\end{enumerate}
