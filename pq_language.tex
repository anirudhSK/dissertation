\section{\THESYSTEM Query language}
\label{sec:language}
%TODO: We seem to keep mixing up folds and groupbys
% SK->NG: We need to say somewhere that the result of any query is a stream
% and a table both of which reside on the measurement servers
\TheSystem provides the abstraction of a network-wide stream of performance
information. The tuples in the stream contain performance metadata, such as
queue lengths and timestamps when a packet entered and departed queues, for {\em
  each packet} and {\em each queue} in the network. Network administrators write
{\em queries} on this stream as if the entire stream is processed at a single
compute node executing the query.

\TheSystem programs process the performance stream using familiar functional
constructs like {\ct filter}, {\ct map}, {\ct fold} and {\ct zip}, all of which
take streams as inputs and produce a stream as output. This {\em functional}
language model is both expressive to support diverse performance monitoring use
cases and simple to implement corresponding hardware primitives.

\Para{Packet performance stream.} \TheSystem provides tuples for each packet at
each queue with the following fields:
\begin{lstlisting}
(switch, qid, hdrs, uid, tin, tout, qsize)
\end{lstlisting}
Here {\ct switch} and {\ct qid} denote the switch and switch queue at which the
packet was observed. A packet may traverse multiple queues even within a single
switch, so we provide distinct fields. The regular packet headers (Ethernet, IP,
TCP, \etc) are available in the {\ct hdrs} set of fields, with a {\ct uid} that
uniquely determines a packet.\footnote{In many networks, it may be possible to
  use a combination of the \txtftuple and IP ID field.}

We provide access to a variety of performance metadata: {\ct tin} and {\ct tout}
denote the timestamps of ingress and egress of the packet in the queue, while
{\ct qsize} denotes the queue depth at the enqueue time of the packet. It is
beneficial to have two timestamps to detect co-habitation of the queue by
packets belonging to different flows. Additionally, it is beneficial to have a
queue size since we cannot always determine the queue size from the two
timestamps, as links may service multiple queues. If a packet
is dropped, the {\ct tout} and {\ct qsize} values are {\ct infinity.}

%% The {\ct path} field provides information about the path of the packet in the
%% network, which may be used to detect route changes or aggregate information by
%% the path of a packet~\cite{path_query}. It may be a VXLAN or MPLS tunnel
%% header.
%% We leave the precise meaning of the field up to the implementation.
% Anirudh: We leave ... maybe better way to word it?

\Para{Restricting packet performance metadata of interest.} Consider the
example of tracking packets that experience spikes in queueing latencies at a
specific queue (identified by {\ct Q}). This is expressed by the query
\begin{lstlisting}
result = filter((*\pktlog,*) qid == Q && tout - tin > 1ms)
\end{lstlisting}
The {\ct filter} operator restricts the user's attention just to the packets
with the relevant performance metadata. A filter has the form {\ct filter(R,
pred)} where {\ct R} is some stream containing performance metadata (\eg {\ct
\pktlog}), and the filter predicate {\ct pred} may involve packet headers or
performance metadata or both. The result of a filter is another stream that
only contains tuples that satisfy the predicate.

\Para{Computing stateless functions over packets.} \TheSystem lets users compute
functions of the fields available in the incoming stream, to express new metrics
of interest. A simple example is rounding packet timestamps to an `epoch':
\begin{lstlisting}
result = map((*\pktlog,*) [tin/epoch_size], [epoch]);
\end{lstlisting}
A {\ct map} evaluates an expression, \ie {\ct tin/epoch\_size}, written over the
fields available in the tuple stream and produces a new field, \ie {\ct
  epoch}. The general form of this construct is {\ct map(R, [expression],
  [field])} where a list of {\ct expression}s over fields in the input stream
{\ct R} creates a list of new {\ct field}s in the {\ct map} output stream.

\Para{Aggregating statefully over multiple packets.} \TheSystem allows
aggregating statistics over multiple packets with the same values of headers
and/or performance metadata, to track information at user-specified
granularities. For example, the following query counts packets belonging to each
transport-level connection:
\begin{lstlisting}
  result = groupby((*\pktlog{}*), [(*\codeftuple{}*)], count);
\end{lstlisting}
Here, the {\ct groupby} construct partitions the incoming {\ct \pktlog} into
substreams based on the transport five-tuple, and then applies the aggregation
function {\ct count} over the tuples of each substream.
% Anirudh: I use the term aggregate slightly different in Section 1.
% We should make sure they mean the same thing everywhere. I think
% your sense of using it to mean flow is better.
\TheSystem allows users to write flexible aggregation functions over the
tuples of each substream. For example, a user can track latency spikes {\em
for each connection} by maintaining an exponentially weighted moving average
(EWMA) of the queueing latencies:
\begin{lstlisting}
result = groupby((*\pktlog{}*), [(*\codeftuple{}*)], ewma);
def ewma(avg, tin, tout):
  avg = ((1-alpha)*avg) + (alpha*(tout-tin));
\end{lstlisting}
Here the aggregation function, {\ct ewma}, evolves an EWMA {\ct avg} using the
current value of {\ct avg} and incoming packet timestamps. Unlike the previous
example using the {\ct count} aggregation function, the EWMA aggregation
function depends on the order of packets being processed.

A {\ct groupby} takes the general form {\ct groupby(R, [aggFields], fun)} where
the function {\ct fun} operates over tuples sharing attributes in a list {\ct
  aggFields} of headers and performance metadata. This construct is inspired by
{\ct fold}s in functional programming applied to
aggregates~\cite{comprehensive-comprehensions}. Such {\ct fold}s are challenging
to express concisely with regular or streaming SQL.
% Anirudh: I like the term partitions over aggregates. Not sure what is better?

The aggregation function {\ct fun} itself is written in an imperative form, with
arguments corresponding to the stored state and relevant incoming tuple
fields. Each statement in {\ct fun} can be an assignment to an expression, \ie
{\ct x = ...} or a branching statement, \ie {\ct if pred \{...\} else \{...\}},
or a special {\ct emit()} statement that controls the {\ct groupby} output
stream. Below, we show an example of an aggregation function that detects a new
connection:
\begin{lstlisting}
result = groupby((*\pktlog{}*), [(*\codeftuple{}*)], new_flow);
def new_flow(fcount):
  if fcount == 0:
    fcount = 1
    emit()
\end{lstlisting}
The result of a {\ct groupby} is a stream containing the aggregation fields (\ie
\txtftuple) and the aggregated values (\ie {\ct fcount}). The output stream only
contains tuples for which the {\ct emit()} statement is encountered during
execution. For example, the output stream of {\ct new\_flow} consists of the
first packet of every new transport-level connection. The user can also read off
all the aggregated fields and corresponding values as a table.
%% If there are no {\ct
%%   emit()} statements in the aggregation function, \TheSystem allows the user to
%% read off all the key-value entries as a table.

% Anirudh: Maybe talk about emit before talking about composing queries
\Para{Composing results of multiple queries.}
%\TheSystem provides an intuitive model for query composition.
It is natural to compose queries over the results of previous queries, since the
functional operators both produce and consume streams. A stream of tuples flows
from one query to the next, and each query operator may add or filter out
information from the flowing tuple, or even drop the tuple entirely. For
example, the program below tracks the distribution of the sizes of {\em
  flowlets}, \ie bursts of packets from the same \txtftuple separated by more
than a fixed time amount {\ct delta}.%% Flowlets
%% provide a useful traffic granularity to load-balance network
%% traffic~\cite{flowlets}.
\begin{lstlisting}
fl_track = groupby((*\pktlog{}*), [(*\codeftuple{}*)], fl_detect);
def fl_detect(last_time, size, tin):
  if (tin - last_time > delta):
    emit()
    size = 1
  else:
    size = size + 1
  last_time = tin
\end{lstlisting}
Here, the function {\ct fl\_detect} detects new flowlets using the last time a
packet from the same flow was seen. Because of the location of the {\ct emit()}
statement, the flowlet size from {\ct fl\_track} is only streamed to other
operators on encountering the first packet of a new flowlet.
%% A special statement within aggregation functions, {\ct emit()}, explicitly
%% decides that a tuple containing the current value of the state (i.e., flowlet
%% size) should be streamed to queries consuming the result, shown below:
\begin{lstlisting}
fl_bkts  = map(fl_track, [size/16], [bucket]);
fl_hist  = groupby(fl_bkts, [bucket], count);
\end{lstlisting}
The map {\ct fl\_bkts} bins the flowlet size emitted by {\ct fl\_track} into a
bucket index, which is used to count the number of flowlets in the corresponding
bucket in {\ct fl\_hist}.

%Anirudh: In general a table of each operator would probably help more than grammer.
% Anirudh: Explicitly say what the result of each query is. A map's result is ...
% Anirudh: We seem to mixup tuples and packets.
\Para{Joining results across queries.} \TheSystem provides a {\ct zip} operator
to ``join'' the results of two queries to check whether two conditions hold
simultaneously. Consider the example of detecting the fan-in of packets from
many connections into a single queue, characteristic of TCP
incast~\cite{tcpincast}. This can be checked by combining two distinct
conditions: (i) the number of active flows in a queue over a short interval of
time is high, and (ii) the queue occupancy is large.

\input{constructs}

A user can first compute the number of active flows over the current epoch using
two aggregations:
% Anirudh: new_flow is not declared here.
\begin{lstlisting}
R1 = map((*\pktlog{}*), [tin/epoch_size], [epoch]);
R2 = groupby(R1, [(*\codeftuple{}*), epoch], new_flow);
R3 = groupby(R2, [epoch], count);
\end{lstlisting}
The number of active flows in this epoch can be combined with the queue
occupancy information in the original packet stream through the {\ct zip}
operator:
\begin{lstlisting}
R4 = zip(R3, (*\pktlog{}*));
result = filter(R4, qsize > 100 and count > 25);
\end{lstlisting}
The result of a {\ct zip} operation over two input streams is a concatenation of
all the fields in the operands, whenever both input streams contain valid tuples
processed from the same original packet tuple. A {\ct zip} is a special kind of
stream join where the result can be computed without any blocking, since tuples
of both streams originate from the same packet tuple. The result of the {\ct
  zip}, \ie {\ct R4}, can be processed like any other stream as before: the {\ct
  filter} in the {\ct result} query checks the two incast conditions above.

For our use cases, we did not find a need for more general joins akin to
streaming query languages. The general semantics of streaming joins can be quite
complex~\cite{cql-vldb}, and such joins may produce large results, \ie
$O(\#pkts^2)$.
%% More general joins may lead to large query result sizes
%% (\ie $O(\#pkts^2)$). Further, the semantics of streaming joins in general can be
%% quite complex~\cite{cql-vldb}. Hence, \TheSystem restricts users to simple {\ct
%%   zip} joins.

\Fig{language-constructs-table} summarizes the language constructs in
\TheSystem. We find the full generality of well-established query languages like
SQL or streaming SQL~\cite{aurora, sparksql} neither necessary nor sufficient
for asking performance questions. We show several examples of \TheSystem queries
in \Fig{example-perf-queries}. The language is useful to express measurements of
simple counters, TCP reordering of various forms, high-loss connections, flows
with high end-to-end network latencies, and TCP fan-in.

