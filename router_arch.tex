\section{The Hardware Architecture of a High-Speed Router}
\label{s:router_arch}

% Say that a router has two primary functions: the control plane and the data plane.
% --> The data plane is in hardware, the control plane runs on an embedded CPU.
% --> Something like PCI between the control and data plane (cite Open Compute Project.)

% Now how is the data plane architected?
% --> Why pipeline instead of shared memory? There's an interesting point here.
% --> (Use animations from the older ppts)

% So the arch is a shared pipeline
% --> a shared pipeline shared across multiple ports,
% --> the scheduler that resides in between the two pipelines.

% What happens when you want to scale further.
% --> maybe multiple such pipelines and
% --> differences between single vs. multi-pipeline routers (multi-ported memory, split-brain)

% What happens in each pipeline?
% --> What is match-action processing?
% --> spend a decent amount of time explaining the machine model: packets come in,
% they are looked up, we carry out some action, and so on so forth.

% How do we evaluate hardware:
% --> simulation, synthesis, emulation 

% conclusion
% --> where each of our three pieces fits within the
% --> context of a router ASIC or router chip.

% Useful text to use when writing this section

%%%%%\subsection{Background: programmable routers}
%%%%%Packets arriving at a router~(top half of Figure~\ref{domino_fig:router}) are
%%%%%parsed by a programmable parser that turns packets into header fields. These
%%%%%header fields are first processed by an ingress pipeline consisting of
%%%%%match-action tables arranged in stages. Processing a packet at a stage may
%%%%%modify its header fields, through match-action rules, as well as some
%%%%%persistent state at that stage, \eg packet counters. After the ingress
%%%%%pipeline, the packet is queued. Once the scheduler dequeues the packet, it is
%%%%%processed by a similar egress pipeline before it is transmitted.
%%%%%
%%%%%To reduce chip area, each ingress and egress pipeline is shared across a number
%%%%%of router ports and handles aggregate traffic belonging to all these ports,
%%%%%regardless of packet sizes. The number of ingress and egress pipelines depends
%%%%%on the aggregate capacity of the router. For instance, a 64-port router with a
%%%%%line rate of 10 Gbit/s per port and a minimum packet size of 64 bytes needs to
%%%%%process around a billion packets per second, after accounting for minimum
%%%%%inter-packet gaps~\cite{rmt}.  This requirement can be supported by a single
%%%%%pipeline that runs at 1 GHz.  For higher aggregate capacities, multiple 1-GHz
%%%%%pipelines are required.\footnote{It is technically challenging to achieve a
%%%%%clock rate higher than 1 GHz in router ASICs.} Throughout this chapter, we
%%%%%assume a single pipeline router, with the pipeline running at 1 GHz.
%%%%%Equivalently, every pipeline stage handles a new packet every clock cycle (1
%%%%%ns). We discuss multi-pipeline routers later (\S\ref{ss:multiple}). 
%%%%%
