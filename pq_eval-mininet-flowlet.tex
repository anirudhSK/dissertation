\subsection{Case study \#2: Flowlet size distributions}
\label{s:eval:mininet-flowlet}
\label{sec:eval:mininet-flowlet}

We demonstrate another use case for Marple: computing the flowlet size
distribution as a function of the flowlet \emph{threshold}, the time gap above
which consecutive packets are considered to be in different flowlets.  This
analysis has many practical uses, \eg for configuring flowlet-based load
balancing strategies~\cite{conga, letflow}.  In particular, the performance of
the LetFlow~\cite{letflow} load balancing algorithm depends heavily on the
distribution of flowlet sizes. 

Our setup uses Mininet with a single router connecting five hosts: a single
client and four servers. Flow sizes are drawn from an empirical distribution
computed from a trace of a real datacenter~\cite{empirical-flow-data}.  The
router runs the ``flowlet size histogram'' query from
Table~\ref{tab:example-perf-queries} for six values of {\ct delta}, the
flowlet threshold.

Figure~\ref{fig:flowletcdf} shows the CDF of flowlet sizes for various values
of {\ct delta}. Note that the actual values of {\ct delta} are a consequence of
the bandwidth allowed by the Mininet setup; a datacenter deployment would
likely use much lower {\ct delta} values.

\begin{figure}[!t]
\centering
\vspace{-0.2in}
\includegraphics[width=0.8\columnwidth]{pq_flowlet-cdf.pdf}
\vspace{-0.3in}
\caption{CDF of flowlet sizes for different flowlet thresholds.}
\vspace{-0.15in}
\label{fig:flowletcdf}
\end{figure}
