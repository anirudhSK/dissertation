\section{Summary}
\label{s:pifo_summary}
%TODO: First paragraph could be made to flow better with the remaining two
Existing research into programmable and fast routers has looked at programming
the router's parser~\cite{glen_parsing} and its match-action
pipeline~\cite{rmt}. But, so far, it has been considered off-limits to program
the packet scheduler---in part because the desired algorithms are so varied,
and because the scheduler sits at the heart of the shared packet buffer where
timing requirements are tightest.  It has been widely assumed too hard to find
a useful abstraction that can also be implemented in fast hardware.

The results in this chapter suggest a promising abstraction for programmable
packet scheduling.  Our abstraction exploits the fact that in many practical
schedulers, the relative order of packets that are already buffered does not
change in response to new packet arrivals.  Put differently, when a packet
arrives, it can be pushed into the right location based on a packet priority
(push in), but packets are always dequeued from the head (first out).  This
observation suggests a natural primitive for programmable scheduling called a
Push In First Out Queue (PIFO). A PIFO is a priority queue where packets are
pushed in based on a rank field, and the next packet to be dequeued is the
packet with the earliest rank.

A single PIFO expresses many schedulers, \eg token bucket shaping, weighted
fair queueing, and strict priority scheduling.  Further,  PIFOs  can  be
combined  to  express  hierarchical  schedulers.   Finally, PIFOs  are
feasible  in hardware:   a hardware design for a programmable 5-level
hierarchical scheduler costs less than 4\% additional chip area.
