\section{Summary}
\label{s:pifo_summary}

Existing research into programmable and fast routers has looked at programming
the router's parser~\cite{glen_parsing} and its match-action
pipeline~\cite{rmt, domino}. But, so far, it has been considered off-limits to
program the packet scheduler---in part because the desired algorithms are so
varied, and because the scheduler sits at the heart of the shared packet buffer
where timing requirements are tightest.  It has been widely assumed too hard to
find a useful abstraction that can also be implemented in fast hardware. PIFOs
appear to be a very promising abstraction: they include a variety of existing
algorithms, and allow us to express new ones. Further, they can be implemented
at line rate with modest chip area overhead.

We believe the most exciting consequence will be the creation of many new
schedulers, invented by network operators, iterated and refined, then deployed
for their own needs. No longer will research experiments be limited to
simulation and progress constrained by a vendor's choice of scheduling
algorithms. Those needing a new algorithm could create their own, or even
download one from an open-source repository or a reproducible SIGCOMM paper.
