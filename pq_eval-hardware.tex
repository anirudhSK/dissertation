\subsection{Hardware compute resources}
\label{s:eval:hardware}
\label{sec:eval:hardware}

\Fig{example-perf-queries} shows several \TheSystem queries. Alongside each
query, we show (1) whether all its aggregations are linear-in-state, (2)
whether it can be scaled by merging correctly with a backing store, and (3) the
switch resources required, measured through the pipeline depth (number of
stages), width (maximum number of parallel computations per stage), and
number of Banzai atoms (total number of computations) required.
%
%% Atoms are processing units or ALUs representing a programmable switch's
%% instruction set.
%
%% An atom models an atomic unit of computation provided natively by a
%% programmable switch and can implement either a stateless (\eg incrementing a
%% packet field) or stateful (\eg incrementing a switch counter) computation.

\Fig{example-perf-queries} shows that many useful queries contain only
linear-in-state aggregations, and most of them scale to a large number of keys
(\Sec{linear-in-state-description}). Notably, the flowlet size histogram and
lossy connection queries are not scalable despite being linear-in-state, since
they contain {\ct emit()} statements.  In \Sec{workaround-nonscalable}, we
showed how to rewrite some of these queries (\eg lossy connections) to scale, at
the cost of losing some accuracy.

We compute the pipeline's depth and width by compiling each query to the Banzai
switch pipeline simulator. Banzai is supplied
with stateless atoms, which perform binary operations (arithmetic, logic, and
relational) on pairs of packet fields, and one stateful atom. For the
linear-in-state operations, we use the multiply-accumulate atom as the stateful
atom, while for the other operations, we use Banzai's own NestedIf atom~\cite{domino_sigcomm}. The Domino compiler determines whether the input
program can be mapped to a pipeline with the specified atoms. As expected, all
the linear-in-state queries map to a pipeline with the multiply-accumulate
atom.

The computational resources required for \TheSystem queries are modest.  All
queries in \Fig{example-perf-queries} require a pipeline shorter than 11 stages.
This is feasible, \eg the
RMT architecture offers 32 stages~\cite{rmt}. Further, functionality other than
measurement can run in parallel
because the number of atoms required per stage is at most 6, while
programmable switches provide \textasciitilde{}100 parallel instructions per stage
(\eg RMT provides 224~\cite{rmt}).

%\Anirudh{the sentence below is redundant because once you know that you're well within
%the stage depth and stage width limits, you know it's feasible.}
%The atom count across all queries is at most 31.  Atom
%circuits occupy small areas relative to the chip overall, so hundreds of atoms
%can easily fit on a switch~\cite{domino_sigcomm}.
