\subsection{Hardware compute resources}
\label{s:eval:hardware}
\label{sec:eval:hardware}

Table~\ref{tab:example-perf-queries} shows several \TheSystem queries.  Next to
each query, we show (1) whether all its aggregations are linear-in-state, (2)
whether it can be scaled by merging correctly with a backing store, and (3) the
router resources required, measured through the pipeline depth (number of
stages), pipeline width (maximum number of parallel computations per stage),
and the number of Banzai atoms (total number of computations).

Table~\ref{tab:example-perf-queries} shows that many useful queries contain
only linear-in-state aggregations, and many of them can be implemented scalably
(\Sec{linear-in-state-description}). Notably, the flowlet size histogram and
lossy connection queries are not scalable despite being linear-in-state, since
they contain {\ct emit()} statements.  In \Sec{workaround-nonscalable}, we
showed how to rewrite some of these queries (\eg lossy connections) to scale,
at the cost of losing some accuracy.

We compute the pipeline's depth and width by compiling each query to Banzai
using the Domino compiler. When compiling each query, Banzai is supplied with a
single stateless atom type, which perform binary operations (arithmetic, logic,
and relational) on pairs of packet fields, and a stateful atom type depending
on the type of the query.  For the linear-in-state queries, we use the
multiply-accumulate atom as the stateful atom, while for the other operations,
we use the NestedIf atom (Table~\ref{tab:templates}). As expected, all the
linear-in-state queries compile to a pipeline with the multiply-accumulate
atom; for all the queries that are not linear-in-state, the NestedIf atom turns
out to be sufficiently expressive.

The computational resources required for \TheSystem queries are modest.  All
queries in Table~\ref{tab:example-perf-queries} require a pipeline shorter than
11 stages.  This is feasible, \eg the RMT architecture offers 32
stages~\cite{rmt}. Further, functionality other than measurement can run in
parallel in each stage because the number of atoms required per stage is at
most 6, while programmable routers provide a few 100 parallel instructions per
stage (\eg RMT provides 224~\cite{rmt}).
