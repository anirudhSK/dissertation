\section{Summary}
\label{s:domino_summary}

This chapter focused on the hardware and software techniques required for the
problem of programming stateful algorithms on a high-speed router. Our solution
to this problem resulted in three new contributions. Our first contribution was
\pktlanguage, an imperative domain-specific language that allows programmers to
write packet-processing code using packet transactions, which are sequential
code blocks that are atomic and isolated from other such code blocks.  Our
second contribution was \absmachine, which is a machine model based on
programmable line-rate router architectures~\cite{flexpipe, xpliant, tofino}.
\absmachine models the essential computational elements of a programmable
router and extends them with the ability to perform programmable high-speed
state manipulation. Our third contribution was the \pktlanguage compiler, which
compiles packet transactions to hardware configurations for \absmachine
targets.

As part of our evaluation (\S\ref{s:eval}), we showed that \pktlanguage offers
a concise and natural model to program stateful algorithms
(Table~\ref{tab:algorithms}). We designed a set of 7 atoms
(Table~\ref{tab:templates}) and showed that they can be used to express a
variety of stateful algorithms (Table~\ref{tab:algo_atoms}). Furthermore, these
atoms generalized to new and unanticipated use cases that were programmed after
the design of these atoms was frozen (Table~\ref{tab:atoms_generalize}). These
results suggest that it possible to program a wide variety of stateful
algorithms at high speeds using the right combination of hardware (atoms) and
programming models (packet transactions).
