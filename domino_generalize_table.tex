%TODO: Double check these numbers once
\begin{table}[!t]
\centering
\begin{small}
  \begin{tabular}{|p{0.3\columnwidth}|p{0.3\columnwidth} | p{0.1\columnwidth}|}
\hline
Algorithm & Guard & Most expressive atom required\\
\hline
HULA probe processing logic~\cite{hula}\footnote{The HULA algorithm requires
some coordination between the ingress and egress pipelines to ensure the link
utilization is available in the ingress. This coordination can be achieved by
occasionally creating a fake packet to carry information from one pipeline to
another, but comes at the cost of the link utilization being slightly out of
date at any given time. However, a slightly stale value of the link utilization
doesn't severely affect the performance of the forwarding algorithm.} & Match
all HULA probe packets & Pairs \\
\hline
HULA forwarding logic~\cite{hula} & Match all data packets & PRAW \\
\hline
BLUE increase logic~\cite{blue} & Match all packets whose enqueue queue depth exceeds a threshold & PRAW \\
\hline
BLUE decrease logic~\cite{blue} & Match all packets where the link utilization is low & Sub \\
\hline
FTP monitoring~\cite{snap} & Match all packets & PRAW \\
\hline
HashPipe (first stage)~\cite{hashpipe} & Match all packets & IfElseRAW \\
\hline
HashPipe (second and subsequent stages)~\cite{hashpipe} & Match only packets that make it past the first stage & Pairs\\
\hline
Checking for frequent domain name changes for a given IP address~\cite{snap} & Match all DNS packets & PRAW \\
\hline
Detect first packet of a flow (Figure~\ref{fig:example-perf-queries}) & Match packets belonging to a particular flow & PRAW \\
\hline
Counting packet reordering within a TCP connection (Figure~\ref{fig:example-perf-queries}) & Match packets belonging to a particular flow & PRAW \\
\hline
Heavy hitter detection~\cite{snap} & Match packets with the TCP SYN flag set & Pairs \\
\hline
Spam detection~\cite{snap} & Match all packets & Pairs \\
\hline
Stateful firewall~\cite{snap} & Match all packets & PRAW \\
\hline
Superspreader detection~\cite{snap} & Match all packets & NestedIf \\
\hline
\end{tabular}
\caption{The atoms in Table~\ref{tab:templates} generalize to new, unanticipated use cases.}
\label{tab:atoms_generalize}
\end{small}
\end{table}
