\section{Deconstructing scheduling}
\label{s:deconstruct}
%TODO: Is order and time an observation or a fact?
We observe that any scheduling algorithm makes two basic decisions: in  {\em
what order} and {\em when} should packets leave the router, broadly corresponding
to work-conserving and non-work-conserving schedulers respectively.  Scheduling
algorithms only differ in how the order and departure time are computed.
Further, in many cases, the order or departure time can be determined when the
packet is enqueued.  To see why, we look at three popular packet-scheduling
algorithms: pFabric~\cite{pFabric}, Weighted Fair Queuing (WFQ)~\cite{wfq}, and
traffic shaping~\cite{tbf}. We then extract similarities between these three
algorithms to provide intuition for a programming model for packet scheduling.

\subsection{pFabric}
pFabric~\cite{pFabric} is a recent datacenter transport design that attempts to
minimize average flow completion time by scheduling packets according to their
remaining flow size at each router, \ie it implements the SRPT scheduling
algorithm at each router~\cite{srpt}.\footnote{There are two variants of
pFabric~\cite{pFabric}, with and without starvation prevention.  We consider
the one without starvation prevention.} For each packet, end hosts insert the
remaining flow size as a packet field. At each router, packets are dequeued in
increasing order of their remaining flow size.

\subsection{Weighted Fair Queuing}
\label{ss:decon_wfq}
Weighted Fair Queuing (WFQ) provides weighted max-min bandwidth allocation
across flows sharing a link. Numerous implementations of WFQ exist, including
Start-Time Fair Queuing (STFQ)~\cite{stfq} and Deficit Round Robin~\cite{drr}.
For concreteness, we consider STFQ.\footnote{The original WFQ
implementation~\cite{wfq} is similar to STFQ, but uses a more complex virtual
time calculation.}

STFQ computes a {\em virtual start time} ({\tt p.start}) for each packet using
the algorithm below. 
\begin{lstlisting}[style=customc]
On enqueue of packet p of flow f:
--------------------------------------------
if f in T
  p.start = max(virtual_time, T[f].last_finish)
else
  p.start = virtual_time
T[f].last_finish = p.start + p.length / f.weight

On dequeue of packet p:
---------------------------------
virtual_time = p.start
\end{lstlisting}

Here, {\tt last\_finish} is a state variable maintained for each flow in table
{\tt T} that tracks the virtual finish time of its latest packet. {\tt
virtual\_time} is a queue-wide state variable updated on each dequeue.  Packets
are scheduled in increasing order of their virtual start time ({\tt p.start}).

\subsection{Traffic shaping}
Besides packet order, some non-work-conserving scheduling algorithms determine
the time at which packets depart from a queue. Traffic shaping is a canonical
example and is used to limit flows to a desired rate. A shaper has two
parameters: a shaping rate, $r$, and a burst allowance, $B$. The standard
implementation uses a {\em token bucket}~\cite{tbf}, which is incremented at a
rate $r$, subject to a cap of $B$ tokens. A packet is transmitted immediately
if the bucket has enough tokens when it is enqueued; otherwise, it has to wait
until sufficient tokens accumulate. Transmitted packets decrement the token
bucket by the packet size.

Alternatively, the transmission time of each packet can be calculated
on enqueue as follows: 
\begin{lstlisting}[style=customc]
tokens = min(tokens + r * (now - last_arrival), B)
if p.length <= tokens
  p.send_time = now
else
  p.send_time = now + (p.length - tokens) / r
tokens = tokens - p.length
last_arrival = now
\end{lstlisting}

Here, {\tt tokens} and {\tt last\_arrival} are two state variables, initialized
to $B$ and an initial time respectively.  While a standard token bucket has
only positive token counts, {\tt tokens} can fall below zero in the algorithm
above.  It is easy to show that the transmission times calculated are still
identical to those of the standard token bucket. With this alternative
algorithm shown above, packets shaped by a token bucket are transmitted in
increasing order of their transmission times ({\tt p.send\_time}).

\subsection{Summary}
The schedulers presented above (pFabric, WFQ, and traffic shaping) determine
the order or time of departure of a packet using a single number computed at
enqueue time.  For pFabric, this number is the remaining flow size and is
inserted by an end host tracking the number of remaining bytes for each flow.
For WFQ, it is the virtual start time computed by the router.  For token bucket
shaping, it is the wall-clock departure time computed by the router.  This
ability to determine scheduling order or time at enqueue unifies a variety of
diverse scheduling algorithms, and it is at the heart of our programming model
for packet scheduling.
