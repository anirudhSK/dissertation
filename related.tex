\chapter{Related Work}
\label{chap:related}

Recent academic work~\cite{rmt} and commercial router chips~\cite{tofino,
flexpipe, xpliant} have considered the problem of building routers that are
both fast and programmable. The P4 programming language~\cite{p4} has emerged
as an industry effort towards a standard programming language for these chips.
To the extent that we can glean from publicly available documents, these chips
provide flexibility on only two counts: recognizing user-specific header
formats and programmatically manipulating packet headers for functions such as
forwarding, tunneling, and access control. In particular, they do not provide
the programmability required to implement the grayed-out algorithms in
Figure~\ref{fig:router_evolution}.  These algorithms require the ability to
programmatically manipulate router state on every packet, the ability to
program which packet a router link must transmit next, and the ability to
program what statistics a router must measure.

\input{domino_related}
\section{Related Work}
\label{s:related}

\medskip
\noindent
\textbf{The Push-in First-out Queue.}
\an{PIFOs were first introduced as a proof construct to prove that a
combined input-output queued switch could exactly emulate an output-queued
switch~\cite{pifo}. We show here that PIFOs can be used as an abstraction for
programmable scheduling at line rate.}

\medskip
\noindent
\textbf{Packet scheduling algorithms.}
The literature is replete with scheduling algorithms~\cite{pFabric, hpfq,
stopngo, stfq, lstf, srpt, drr, rcsd} . Yet, line-rate switches support only a
few: DRR, traffic shaping, and strict priorities. As \S\ref{s:expressive}
shows, PIFOs allow a line-rate switch to run many of these scheduling
algorithms, which, so far, have only been run on software routers.

\medskip
\noindent
\textbf{Programmable switches.} Recent work has proposed hardware architectures~\cite{tofino, flexpipe,
xpliant, rmt} and software abstractions~\cite{p4, domino_sigcomm} for
programmable switches.  While many packet-processing tasks can be programmed on
these switches, scheduling isn't one of them. Programmable switches can {\em
assist} a PIFO-based scheduler by providing a programmable ingress pipeline for
scheduling and shaping transactions, without requiring a dedicated atom
pipeline inside each PIFO block.  However, they still need PIFOs for
programmable scheduling.

\medskip
\noindent
\textbf{Universal Packet Scheduling (UPS).} UPS~\cite{ups} shares our goal of
flexible packet scheduling by seeking a single scheduling algorithm that is
{\em universal} and can emulate any scheduling algorithm. Theoretically, UPS
finds that the well-known LSTF scheduling discipline~\cite{lstf} is universal
if packet departure times for the scheduling algorithm to be emulated are known
up front. Practically, UPS shows that by appropriately initializing slacks, many different scheduling objectives can be
emulated using LSTF. LSTF is programmable using PIFOs, but the set of schemes
practically expressible with LSTF is limited. For example, LSTF cannot
express:
\begin{CompactEnumerate}
\item Hierarchical scheduling algorithms such as HPFQ, because it
  uses only one priority queue.
\item Non-work-conserving algorithms. For such algorithms LSTF must know the
  departure time of each packet up-front, which is not practical.
\item Short-term bandwidth fairness in fair queueing, because LSTF maintains no
  switch state except one priority queue. As shown in
  Figure~\ref{fig:sched_trans}, programming a fair queueing algorithm requires us
  to maintain a virtual time state variable. Without this, a new flow could have
  arbitrary virtual start times, and be deprived of its fair share indefinitely.
  UPS provides a fix to this that requires
  estimating fair shares periodically, which is hard to do in
  practice.
\item Scheduling policies that aggregate flows from distinct endpoints into a
  single flow at the switch. An example is fair queueing across video and web
  traffic classes, regardless of endpoint.  Such policies require the switch to
  maintain the state required for fair queueing because no end point sees all the
  traffic within a class.  However, LSTF cannot maintain and update switch state
  progammatically.
\end{CompactEnumerate}
\an{The restrictions in UPS/LSTF are a result of a limited programming
model. UPS assumes that switches are fixed and cannot be programmed to modify
packet fields. Further, it only has a single priority queue.  By using atom
pipelines to execute scheduling and shaping transactions, and by composing
multiple PIFOs together, PIFOs express a wider class of scheduling algorithms.}

%\begin{figure}
%  \centering
%  \includegraphics[width=0.7\columnwidth]{state_reqd.pdf}
%  \caption{A switch's scheduling algorithm, such as WFQ, might aggregate flows
%  from different end hosts into a single flow at the switch for the purpose of
%  scheduling.}
%  \label{fig:state}
%\end{figure}

\medskip
\noindent
\textbf{Hardware designs for priority queues.}
\an{P-heap is a pipelined binary heap scaling to 4-billion entries~\cite{bhagwan,
pheap}.  However, each P-heap supports traffic belonging to a {\em single} 10
Gbit/s input port in an input-queued switch and there is a separate P-heap
instance for each port~\cite{bhagwan}.  This per-port design incurs prohibitive
area overhead on a shared-memory switch, and prevents sharing of the data
buffer and binary heap across output ports. Conversely, it isn't easy to
overlay multiple logical PIFOs over a single P-heap, which would allow the
P-heap to be shared across ports.}
%%
%%\an{
%%In contrast to a hardware implementation of a generic priority queue as a heap,
%%our design for the PIFO exploits two domain-specific insights. First, there is
%%considerable structure in the ranks: ranks within a flow strictly increase with
%%time.  Second, the packet buffers on shared-memory switches used in datacenters
%%today are much smaller than those on deep-buffered core routers in the past.
%%This permits a simpler, albeit less scalable, design relative to heaps.
%%}

\section{Related Work}
\label{sec:related}
\Para{Endpoint-based monitoring.} Owing to limited switch support for
measurement, many systems monitor network performance from
endpoints alone~\cite{netpoirot, minlan-snap, dapper-sosr, trumpet,
azure-smartnic}. While endpoint solutions are necessary for application
context (\eg socket calls), they are inadequate to debug all network problems. A
real network needs both endpoint and switch-based
systems because each sees something the other cannot.

\Para{Switch-based monitoring.} Traditionally, switch-based monitoring has
focused on per-flow counts, not performance measurement. For example,
NetFlow~\cite{netflow} and sFlow~\cite{sflow} provide traffic summaries
through flow and packet sampling. Packet-capture systems~\cite{cisco-span,
niksun, netsight, everflow, pathdump, path_query} collect entire packets or
digests. These approaches sample extensively to lower collection
overheads, and are useful for posthoc traffic analysis. However, neither
approach captures details of {\em performance} phenomena (\eg TCP incast) as
specified by a flexible language like \TheSystem.

Sketches~\cite{univmon, flowradar, counterbraids, dream} and earlier work
on programmable switch measurements~\cite{progme, opensketch} provide traffic
volume statistics using summary data structures on switches.  Unlike sketches,
\TheSystem does
not have an accuracy-memory tradeoff, since counting is
linear-in-state and counters can be measured accurately. Instead, \TheSystem
trades off memory size with cache eviction rate (\Sec{eval}). \TheSystem also
allows users to perform a broader set of aggregations with full
accuracy.

%% \TheSystem also enables users
%% to perform other more general aggregations without losing accuracy.

%
%With INT alone, performance information may be lost, since packets carrying the
%INT data may be dropped on the way to endpoints.

In-band Network Telemetry (INT)~\cite{int, tpp} exposes queue lengths to
endpoints by stamping it on the packet itself. \TheSystem builds on INT and
provides flexible filters and aggregations {\em directly in switches}.
\TheSystem's data aggregation in switches saves
the bandwidth needed to collect INT data distributed over many endpoints.
In addition, the
Tetration chip provides flow-level telemetry, exposing a fixed set of metrics
including latency, window and packet size variation, and a ``burst
measurement''~\cite{tetration-telemetry}. In contrast, \TheSystem provides
programmable aggregation functions and aggregation levels.%% , \eg port
%% versus flow-level.

\Para{Programmable switches.} \TheSystem builds on programmable switch
architectures, leveraging their support for flexible
parsing~\cite{gibb_parsing}, forwarding~\cite{rmt, openflow} and stateful
processing~\cite{domino_sigcomm}. In addition, we design hardware for scalable
high-speed aggregation for a broad class of aggregation functions.

\TheSystem's vision and hardware primitive are similar to an earlier
position paper~\cite{marple-hotnets}. In addition to that work, \TheSystem
provides a functional query language (as opposed to SQL), a query compiler, a
formal characterization of which aggregation functions admit a scalable
implementation, a measurement of switch resources taken up by common queries,
and an end-to-end demonstration of its use.

%\begin{table}[t]
%\centering
%\small
%%\vspace{-0.15in}
%\begin{tabular}{|c|c|c|c|c|} \hline
%\bf{src $\rightarrow$ dst} & \bf{protocol} & \bf{\# Bursts} & \bf{Time ($\mu$s)} & \bf{\# Packets} \\ \hline
%h3:34573 $\rightarrow$ h4:4888 & UDP & 19 & 8969085 & 6090 \\
%h4:4888 $\rightarrow$ h3:34573 & UDP & 18 & 10558176 & 5820 \\
%h1:1777 $\rightarrow$ h2:58439 & TCP & 1 & 72196926 & 61584 \\
%h2:58439 $\rightarrow$ h1:1777 & TCP & 1 & 72248749 & 33074 \\ \hline
%\end{tabular}
%\caption{Per-flow burst statistics from \TheSystem.}
%\label{t:mininet-flowstats}
%\end{table}

%SK->NG: I don't think the Brown work is related to us.
%%\TheSystem shares the vision of developing new monitoring switch primitives with
%%the prior work of Nelson~\etal~\cite{stateful-switch-monitoring}. The semantic
%%switch features they propose pertain to stateful properties of network
%%protocols; our focus is on performance metrics. Notwithstanding, our key-value
%%store primitive can support a subset of their semantic features.
%% The key-value store we propose can be used to implement
%% some of the stateful property monitoring use cases in this paper, like the
%% stateful firewall and network address translation.

\Para{Network query languages.} Prior network query languages~\cite{gigascope,
frenetic, path_query, streaming-monitoring} allow users to ask questions
primarily about traffic volumes and count statistics, since their input data is
collected using NetFlow and match-action rule counters~\cite{openflow}. In
contrast, \TheSystem enables asking expressive {\em performance} questions on
data collected with purpose-built switch hardware. \TheSystem shares some
functional and relational constructs with Gigascope~\cite{gigascope} and
Sonata~\cite{streaming-monitoring}, but supports aggregations directly in the
switch.

\Para{Language-directed computer design.} %% The hardware design process
%% used to create \TheSystem is (i) creating a language that expresses the
%% kinds of switch monitoring we want and (ii) developing an efficient
%% hardware design that supports the language.
Our hardware design process is inspired by early
efforts on language-directed computer
design~\cite{language-directed-computer-design, ditzel_patterson, soar},
aimed at designing efficient hardware to support expressive high-level languages.

