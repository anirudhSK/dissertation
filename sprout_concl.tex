\section{Conclusion}
\label{s:comcl}

This paper presented Sprout, a transport protocol for real-time
interactive applications over Internet paths that traverse cellular
wireless networks. Sprout improves on the performance of current
approaches by modeling varying networks explicitly. Sprout has two
interesting ideas: the use of packet arrival times as a congestion
signal, and the use of probabilistic inference to make a cautious
forecast of packet deliveries, which the sender uses to pace its
transmissions. Our experiments show that forecasting is important to
controlling delay, providing an end-to-end rate control algorithm that
can react at time scales shorter than a round-trip time.

Our experiments conducted on traces from four commercial cellular
networks show many-fold reductions in delay, and increases in
throughput, over Skype, Facetime, and Hangout, as well as over Cubic,
Compound TCP, Vegas, and LEDBAT. Although Sprout is an end-to-end
scheme, in this setting it matched or exceeded the performance of
Cubic-over-CoDel, which requires modifications to network
infrastructure to be deployed.
